% +++++ DEFINITIONS +++++
\declaretheoremstyle[
	headfont=\bfseries\sffamily\color{definition-header-text}, bodyfont=\normalfont,
    postheadhook={},
	mdframed={
			linewidth=2pt,
			rightline=false, topline=false, bottomline=false,
            roundcorner=3pt,
			linecolor=ForestGreen, backgroundcolor=definition-background, fontcolor=definition-body-text,
			nobreak=false
		}
]{thmgreenbox}

\declaretheorem[style=thmgreenbox, name=Definition, numberwithin=section]{definitionbase}
\NewDocumentEnvironment{definition}{o} % This is a wrapper to get new line when a name is provided
{
    \IfNoValueTF{#1}
    {\begin{definitionbase}}
    {
      \begin{definitionbase}[#1]$\newline$
    }
}
{
  \end{definitionbase}%
}

% +++++ EXAMPLES +++++
\declaretheoremstyle[
	headfont=\bfseries\sffamily\color{example-header-text}, bodyfont=\normalfont,
	mdframed={
			linewidth=2pt,
			rightline=false, topline=false, bottomline=false,
			linecolor=NavyBlue, backgroundcolor=example-background, fontcolor=definition-body-text,
			nobreak=false
		}
]{thmbluebox}
\declaretheorem[style=thmbluebox, numbered=no, name=Example]{eg}

% +++++ THEOREMS +++++
\declaretheoremstyle[
	headfont=\bfseries\sffamily\color{theorem-header-text}, bodyfont=\normalfont,
	mdframed={
			linewidth=2pt,
			rightline=false, topline=false, bottomline=false,
			linecolor=RawSienna, backgroundcolor=theorem-background, fontcolor=theorem-body-text,
			nobreak=false
		}
]{thmredbox}
\declaretheorem[style=thmredbox, name=Theorem, numberwithin=section]{theorem}

% +++++ PROOFS +++++
% \makeatletter
% \renewenvironment{proof}[1][\proofname]{\par
% 	\pushQED{\qed}%
% 	\normalfont \topsep-2\p@\@plus6\p@\relax
% 	\trivlist
% 	\item[\hskip\labelsep
% 	            \color{RawSienna!70!black}\sffamily\bfseries
% 	            #1\@addpunct{.}]\ignorespaces
% 	\begin{mdframed}[linewidth=2pt,rightline=false, topline=false, bottomline=false,linecolor=RawSienna, backgroundcolor=RawSienna!1]
% 		}{%
% 		\popQED\endtrivlist\@endpefalse
% 	\end{mdframed}
% }
% \makeatother