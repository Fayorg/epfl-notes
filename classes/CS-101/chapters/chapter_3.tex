\chapter{Methods of Proof}
\begin{definition}[Argument]
    An argument consists of a set of premises and a conclusion. The premises are propositions that are assumed to be true, and the conclusion is a proposition that is inferred from the premises. An argument is often written in the following form:
    \[
        \begin{array}{rl}
            & P_1 \\
            & P_2 \\
            & ... \\
            & P_n \\
            \hline
            \therefore & Q
        \end{array}
    \]
    where $P_1, P_2, ..., P_n$ are the premises and $Q$ is the conclusion.
\end{definition}

\begin{definition}[Argument Form]
    An argument form is a template for an argument that uses propositional variables instead of specific propositions. For example, the argument form for modus ponens is:
    \[
        \begin{array}{rl}
            & p \to q \\
            & p \\
            \hline
            \therefore & q
        \end{array}
    \]
    where $p$ and $q$ are propositional variables that can be replaced with any propositions.
\end{definition}

\begin{definition}[Valid Argument]
    An argument is valid if and only if the conclusion is true whenever all the premises are true.
\end{definition}

% TODO: example with passed exams to go to second year
% \begin{eg}
    
% \end{eg}

\section{Inference Rules}
\begin{definition}[Inference Rule]
    An inference rule is a valid argument form that can be used to derive a conclusion from a set of premises. Some common inference rules are:
\end{definition}

\subsection{Rules of Inference for Propositions}
\begin{definition}[Modus Ponens]
    If $P \to Q$ is true and $P$ is true, then $Q$ must be true. It is often written as follows:
    \[
        \begin{array}{rl}
            & P \to Q \\
            & P \\
            \hline
            \therefore & Q
        \end{array}
    \]
    It can also be written as a tautology: $(P \land (P \to Q)) \to Q$.
\end{definition}

\begin{definition}[Conjunction]
    If $P$ is true and $Q$ is true, then $P \land Q$ is true. It is often written as follows:
    \[
        \begin{array}{rl}
            & P \\
            & Q \\
            \hline
            \therefore & P \land Q
        \end{array}
    \]
    It can also be written as a tautology: $((P) \land (Q)) \to (Q \land P)$.
\end{definition}

\begin{definition}[Modus Tollens]
    If $P \to Q$ is true and $\neg Q$ is true, then $\neg P$ must be true. It is often written as follows:
    \[
        \begin{array}{rl}
            & P \to Q \\
            & \neg Q \\
            \hline
            \therefore & \neg P
        \end{array}
    \]
    It can also be written as a tautology: $(\neg Q \land (P \to Q)) \to \neg P$.
\end{definition}

\begin{definition}[Hypothetical Syllogism]
    If $P \to Q$ is true and $Q \to R$ is true, then $P \to R$ must be true. It is often written as follows:
    \[
        \begin{array}{rl}
            & P \to Q \\
            & Q \to R \\
            \hline
            \therefore & P \to R
        \end{array}
    \]
    It can also be written as a tautology: $((P \to Q) \land (Q \to R)) \to (P \to R)$.
\end{definition}

\begin{eg}
    Let $p =$ "I have passed AICC", $q =$ "I can advance to year 2 of studies" and the premises be:
    \begin{itemize}[itemsep=1pt,label=$\circ$]
        \item "I have passed AICC. I can advance to year 2 of studies." ($p \to q$)
        \item "I cannot advanced to year 2 of studies." ($\neg q$)
    \end{itemize}
    We can represent the argument as follows:
    \[
        \begin{array}{rl}
            & p \to q \\
            & \neg q \\
            \hline
            \therefore & \neg p
        \end{array}
    \]
    We can conclude that "I have not passed AICC." ($\neg p$) by using modus tollens.
\end{eg}

\begin{definition}[Disjunctive Syllogism]
    If $P \lor Q$ is true and $\neg P$ is true, then $Q$ must be true. It is often written as follows:
    \[
        \begin{array}{rl}
            & P \lor Q \\
            & \neg P \\
            \hline
            \therefore & Q
        \end{array}
    \]
    It can also be written as a tautology: $((P \lor Q) \land \neg P) \to Q$.
\end{definition}

\begin{definition}[Addition]
    If $P$ is true, then $P \lor Q$ is true. It is often written as follows:
    \[
        \begin{array}{rl}
            & P \\
            \hline
            \therefore & P \lor Q
        \end{array}
    \]
    It can also be written as a tautology: $P \to (P \lor Q)$.
\end{definition}

\begin{definition}[Simplification]
    If $P \land Q$ is true, then $P$ is true. It is often written as follows:
    \[
        \begin{array}{rl}
            & P \land Q \\
            \hline
            \therefore & P
        \end{array}
    \]
    It can also be written as a tautology: $(P \land Q) \to P$.
\end{definition}

\begin{definition}[Resolution]
    If $P \lor Q$ is true and $\neg P \lor R$ is true, then $Q \lor R$ must be true. It is often written as follows:
    \[
        \begin{array}{rl}
            & P \lor Q \\
            & \neg P \lor R \\
            \hline
            \therefore & Q \lor R
        \end{array}
    \]
    It can also be written as a tautology: $((P \lor Q) \land (\neg P \lor R)) \to (Q \lor R)$.
\end{definition}

\begin{eg}
    In the sentence "It is below freezing now. Therefore, it is below freezing or raining now", let $p =$ "It is below freezing now" and $q =$ "It is raining now". We can represent the argument as follows:
    \[
        \begin{array}{rl}
            & p \\
            \hline
            \therefore & p \lor q
        \end{array}
    \]
    Since this argument follows the form of addition, we can conclude that "It is below freezing or raining now." ($p \lor q$).
\end{eg}

\begin{eg}
    In the sentence "If it rains today, the we will not have a barbecue today. If we do not have a barbecue today, then we will have a barbecue tomorrow", let $p =$ "It rains today", $q =$ "We will have a barbecue today" and $r =$ "We will have a barbecue tomorrow". We can represent the argument as follows:
    \[
        \begin{array}{rl}
            & p \to \neg q \\
            & \neg q \to r \\
            \hline
            \therefore & p \to r
        \end{array}
    \]
    Since this argument follows the form of hypothetical syllogism, we can conclude that "If it rains today, then we will have a barbecue tomorrow." ($p \to r$).
\end{eg}
Note that even seemingly "obvious" conclusions imply an argument.
\begin{eg}
    From $p \land (p \to q)$, we can conclude $q$:
    \[
        \begin{array}{rl}
            & p \land (p \to q) \\
            \hline
            \therefore & q
        \end{array}
    \]
    This can be simplified to:
    \[
        \begin{array}{rl}
            & p \\
            & p \to q \\
            \hline
            \therefore & q
        \end{array}
    \]
\end{eg}

\subsection{Rules of Inference for Quantified Statements}
\begin{definition}[Universal Instantiation]
    If $P(x)$ is a predicate and $c$ is an element in the domain of discourse, then from $\forall x P(x)$ we can conclude $P(c)$. It is often written as follows:
    \[
        \begin{array}{rl}
            & \forall x P(x) \\
            \hline
            \therefore & P(c)
        \end{array}
    \]
\end{definition}

\begin{definition}[Universal Generalization]
    If $P(c)$ is true for an arbitrary element $c$ in the domain of discourse, then we can conclude $\forall x P(x)$. It is often written as follows:
    \[
        \begin{array}{rl}
            & P(c) \\
            \hline
            \therefore & \forall x P(x)
        \end{array}
    \]
    Note that $c$ must be arbitrary, meaning that it cannot have any special properties that distinguish it from other elements in the domain.
\end{definition}

\begin{definition}[Existential Instantiation]
    If $P(x)$ is a predicate and $c$ is an element in the domain of discourse, then from $\exists x P(x)$ we can conclude $P(c)$. It is often written as follows:
    \[
        \begin{array}{rl}
            & \exists x P(x) \\
            \hline
            \therefore & P(c)
        \end{array}
    \]
    Note that $c$ must be a new element that does not appear elsewhere in the argument.
\end{definition}

\begin{definition}[Existential Generalization]
    If $P(c)$ is true for some element $c$ in the domain of discourse, then we can conclude $\exists x P(x)$. It is often written as follows:
    \[
        \begin{array}{rl}
            & P(c) \\
            \hline
            \therefore & \exists x P(x)
        \end{array}
    \]
\end{definition}

\begin{eg}
    Let the domain of discourse be all students in a class, let $A$ be a student in the class" and let the predicate $P(x) =$ "x has taken a course in Java". Given the premise "All students in the class have taken a course in Java." ($\forall x P(x)$), we can represent the argument as follows:
    \[
        \begin{array}{rl}
            & \forall x P(x) \\
            \hline
            \therefore & P(Sara)
        \end{array}
    \]
    where Sara is a student in the class. By using universal instantiation, we can conclude that "Sara has taken a course in Java." ($P(Sara)$).
\end{eg}

\begin{eg}
    Let's use the rules of inference to construct a valid argument showing that "Someone who passed the first exam has not read the book". Let's define some predicates as follows:
    \begin{itemize}[itemsep=1pt,label=$\circ$]
        \item $P(x)$: "x passed the first exam"
        \item $B(x)$: "x has read the book"
        \item $C(x)$: "x is in this class"
    \end{itemize}
    The the conslusion can be expressed as:
    \[
        \exists x (P(x) \land \neg B(x))
    \]
    The premises are:
    \begin{itemize}[itemsep=1pt,label=$\circ$]
        \item "Everyone in this class passed the first exam". ($\forall x (C(x) \to P(x))$)
        \item "A student in this class has not read the book". ($\exists x (C(x) \land \neg B(x))$)
    \end{itemize}
    We can represent the argument as follows:
    \[
        \begin{array}{rl}
            & \forall x (C(x) \to P(x)) \\
            & \exists x (C(x) \land \neg B(x)) \\
            \hline
            \therefore & \exists x (P(x) \land \neg B(x))
        \end{array}
    \]
    From the second premise, we can use existential instantiation to introduce a new constant $c$ such that:
    \[
        C(c) \land \neg B(c)
    \]
    From this, we can use simplification to obtain:
    \[
        C(c)
    \]
    From the first premise, we can use universal instantiation to obtain:
    \[
        C(c) \to P(c)
    \]
    From this and $C(c)$, we can use modus ponens to obtain:
    \[
        P(c)
    \]
    Finally, we can use conjunction to obtain:
    \[
        P(c) \land \neg B(c)
    \]
    And then we can use existential generalization to obtain the conclusion:
    \[
        \exists x (P(x) \land \neg B(x))
    \]
\end{eg}

\section{Exercices}
This section gathers a selection of exercises related to Chapter \thechapter, taken from weekly assignments, past exams, textbooks, and other sources. The origin of each exercise will be indicated at its beginning.