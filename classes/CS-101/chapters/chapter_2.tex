\chapter{Predicate Logic}
In propositional logic, some expression cannot be expressed easily or cannot be expressed at all. For example, the statement "x + 2 = 5.” is not a proposition because it contains a variable x. However, we can express this statement as a predicate.

\section{Propositional Functions}

\section{Quantifiers}

\subsection{Universal Quantifier}

\subsection{Existential Quantifier}

\subsection{Uniqueness Quantifier}

\subsection{Precedence, Scope and Binding}

\subsection{Validity and Satisfiability}

\section{Exercices}
This section gathers a selection of exercises related to Chapter \thechapter, taken from weekly assignments, past exams, textbooks, and other sources. The origin of each exercise will be indicated at its beginning.

\begin{eg} % from Explique AI (explique.epfl.ch)
    Given these two assemptions: 
    \begin{itemize}[itemsep=1pt,label=$\circ$]
        \item "Running is not difficult or not many students like running"
        \item "If jumping is easy, then running is not difficult"
    \end{itemize}
    How many of the following are valid conclusion of these assemptions?
    \begin{itemize}[itemsep=1pt,label=$\circ$]
        \item "Jumping is not easy, if many students like running"
        \item "Jumping is not easy or running is difficult"
        \item "If not many students like running, then either jumping is not easy or running is not difficult"
        \item "Running is not difficult or jumping is not easy"
    \end{itemize}
    $\newline$
    Answer: 3 (1, 3, 4)\\ \\
    Explaination: Let's translate the propositions as following:
    \begin{itemize}[itemsep=1pt,label=$\circ$]
        \item $D$: "Running is difficult"
        \item $M$: "Many student like running"
        \item $J$: "Jumping is easy"
    \end{itemize}
    And the premises translate to:
    \begin{itemize}[itemsep=1pt,label=$\circ$]
        \item $D \lor \neg M$
        \item $J \to \neg D \equiv \neg J \lor \neg D$ 
    \end{itemize}
    1. If $M$ holds, premise (1) forces $D$ (because $D \lor \neg M$ and $\neg M$ is false). From the contrapositive of (2) we have $D \to \neg J$. So $M \to D$ and $D \to \neg J$ gives $M \to \neg J$. \textbf{Valid.}
\end{eg}