\chapter{Representations of Numbers}

\section{Integers}

\begin{definition}[Decimal Notation]
    The decimal notation of an integer \( z \in \mathbb{Z} \) is a sequence of digits \( d_n d_{n-1} \ldots d_1 d_0 \) such that
    \[
        z = \sum_{i=0}^{n} d_i \cdot 10^i
    \]
    where each digit \( d_i \) is in the set \( \{0, 1, 2, \ldots, 9\} \).
\end{definition}

\begin{eg}
    The decimal notation of the integer \( 345 \) is given by the digits \( 3, 4, 5 \):
    \[
        345 = 3 \cdot 10^2 + 4 \cdot 10^1 + 5 \cdot 10^0
    \]
\end{eg}
Numbers can also be represented in other bases, the most important for computing are: binary (base 2), octal (base 8), and hexadecimal (base 16), but are not limited to these (ancient Mayans used base 20, ancient Babylonians used base 60).

\begin{definition}[Base $b$ Notation]
    The base \( b \) notation of an integer greater than $1$ is a sequence of digits \( d_n d_{n-1} \ldots d_1 d_0 \) such that \( 0 \leq d_i < b \), $n$ a non negative integer and \( d_n \neq 0 \).
    \[
        z = \sum_{i=0}^{n} d_i \cdot b^i
    \]
    where each digit \( d_i \) is in the set \( \{0, 1, 2, \ldots, b-1\} \).
\end{definition}
Remark that the base \( b \) notation is unique for each integer.
\begin{proof}
    Let \( z \in \mathbb{Z} \) be an integer and suppose there exist two different base \( b \) notations for \( z \):
    \[
        z = \sum_{i=0}^{n} d_i \cdot b^i = \sum_{i=0}^{m} e_i \cdot b^i
    \]
    where \( d_i, e_i \in \{0, 1, 2, \ldots, b-1\} \), \( d_n \neq 0 \), and \( e_m \neq 0 \). Without loss of generality, assume \( n \geq m \). We can rewrite the equation as:
    \[
        \sum_{i=0}^{n} d_i \cdot b^i - \sum_{i=0}^{m} e_i \cdot b^i = 0
    \]
    This implies:
    \[
        \sum_{i=0}^{m} (d_i - e_i) b^i + \sum_{i=m+1}^{n} d_i b^i = 0
    \]
    Since \( d_n \neq 0 \), the term \( d_n b^n \) is non-zero and dominates the sum for sufficiently large \( n \). Therefore, the only way for the entire sum to equal zero is if all coefficients are zero:
    \[
        d_i - e_i = 0 \quad \text{for } i = 0, 1, \ldots, m
    \]
    and
    \[
        d_i = 0 \quad \text{for } i = m+1, m+2, \ldots, n
    \]
    This leads to a contradiction since \( d_n \neq 0 \). Hence, our assumption that there exist two different base \( b \) notations for \( z \) is false. Therefore, the base \( b \) notation of an integer is unique.
\end{proof}

\begin{eg}
    Let's represent the integer $17$ in common bases:
    \begin{itemize}[itemsep=1pt,label=$\circ$]
        \item Binary (base 2): \( 17 = 1 \cdot 2^4 + 0 \cdot 2^3 + 0 \cdot 2^2 + 1 \cdot 2^1 + 1 \cdot 2^0 = 10001_2 \)
        \item Octal (base 8): \( 17 = 2 \cdot 8^1 + 1 \cdot 8^0 = 21_8 \)
        \item Hexadecimal (base 16): \( 17 = 1 \cdot 16^1 + 1 \cdot 16^0 = 11_{16} \)
    \end{itemize}
\end{eg}
Note that in hexadecimal notation, digits above 9 are represented using letters A to F (A=10, B=11, C=12, D=13, E=14, F=15) so all the numbers from $1$ to $15$ have a single digit representation.

\subsection{Contruction of Base $b$ Notation}
There are several algorithms to construct the base \( b \) notation of an integer \( n \geq 0 \).

\begin{eg}
    To construct a number \( n \) in base \( b \), we can use the repeated subtraction method:
    \begin{itemize}[itemsep=1pt,label=$\circ$]
        \item Find the largest power of \( b \), say \( b^k \), such that \( b^k \leq n \).
        \item Determine the coefficient \( d_k \) by calculating \( d_k = \lfloor n / b^k \rfloor \).
        \item Update \( n \) to \( n - d_k \cdot b^k \).
        \item Repeat the process for \( b^{k-1}, b^{k-2}, \ldots, b^0 \) until \( n \) becomes zero.
    \end{itemize}
    For example, to convert \( 45 \) to base \( 3 \):
    \[
        3^3 = 27 \leq 45 < 81 = 3^4 \quad \Rightarrow \quad d_3 = \lfloor 45 / 27 \rfloor = 1
    \]
    \[
        n = 45 - 1 \cdot 27 = 18
    \]
    \[
        3^2 = 9 \leq 18 < 27 = 3^3 \quad \Rightarrow \quad d_2 = \lfloor 18 / 9 \rfloor = 2
    \]
    \[
        n = 18 - 2 \cdot 9 = 0
    \]
    \[
        3^1 = 3 \leq 0 < 9 = 3^2 \quad \Rightarrow \quad d_1 = \lfloor 0 / 3 \rfloor = 0
    \]
    \[
        3^0 = 1 \leq 0 < 3 = 3^1 \quad \Rightarrow \quad d_0 = \lfloor 0 / 1 \rfloor = 0
    \]
    Thus, reading the coefficients from \( d_3 \) to \( d_0 \), we get \( 45 = 1200_3 \).
\end{eg}

\begin{theorem}[Division Remainder Method]
    If $a$ is an integer and $d$ a positive integer, then there are unique integers $q$ and $r$ such that $0 \leq r < d$ and:
    \[
        a = d \cdot q + r
    \]
    The integer $q$ is called the quotient, $d$ is called the divisor, $a$ is called the dividend and $r$ the remainder of the division of $a$ by $d$.
\end{theorem}

\begin{eg}
    For example, dividing \( 17 \) by \( 5 \):
    \[
        17 = 5 \cdot 3 + 2
    \]
    Here, the quotient \( q = 3 \) and the remainder \( r = 2 \).
\end{eg}

\begin{eg}
    Another example, dividing $-11$ by $3$:
    \[
        -11 = 3 \cdot (-4) + 1
    \]
    Here, the quotient \( q = -4 \) and the remainder \( r = 1 \).
\end{eg}

\begin{eg}
    To construct a number $n$ in base $b$, we can use the division-remainder method:
    \begin{itemize}[itemsep=1pt,label=$\circ$]
        \item Divide \( n \) by \( b \) to get a quotient \( q_0 \) and a remainder \( d_0 \) (the least significant digit).
        \item Set \( n = q_0 \) and repeat the division until the quotient is zero.
        \item The base \( b \) representation is obtained by reading the remainders in reverse order.
    \end{itemize}
    For example, to convert \( 45 \) to base \( 2 \):
    \[
        45 \div 2 = 22 \text{ remainder } 1 \quad (d_0 = 1)
    \]
    \[
        22 \div 2 = 11 \text{ remainder } 0 \quad (d_1 = 0)
    \]
    \[
        11 \div 2 = 5 \text{ remainder } 1 \quad (d_2 = 1)
    \]
    \[
        5 \div 2 = 2 \text{ remainder } 1 \quad (d_3 = 1)
    \]
    \[
        2 \div 2 = 1 \text{ remainder } 0 \quad (d_4 = 0)
    \]
    \[
        1 \div 2 = 0 \text{ remainder } 1 \quad (d_5 = 1)
    \]
    Reading the remainders in reverse order, we get \( 45 = 101101_2 \).
\end{eg}

\subsection{Operations on Base $b$ Notation}

\begin{definition}[Addition in Base $b$]
    To add two numbers in base \( b \), align the digits and add them column by column from right to left, carrying over any value that exceeds \( b-1 \) to the next column.
\end{definition}

\begin{eg}
    For example, adding \( 345_8 \) and \( 267_8 \) in base \( 8 \):
    \[
        \begin{array}{c@{}c@{}c@{}c}
          & 3 & 4 & 5_8 \\
        + & 2 & 6 & 7_8 \\
        \hline
          & 6 & 3 & 4_8 \\
        \end{array}
    \]
    Here, \( 5 + 7 = 12_{10} = 14_8 \) (write down \( 4 \), carry over \( 1 \)), \( 4 + 6 + 1 = 11_{10} = 13_8 \) (write down \( 3 \), carry over \( 1 \)), and \( 3 + 2 + 1 = 6_{10} = 6_8 \).
\end{eg}

\begin{definition}[Multiplication in Base $b$]
    To multiply two numbers in base \( b \), use the standard multiplication algorithm, multiplying each digit of the second number by the entire first number, shifting left for each digit position, and then summing all the partial products. The pseudo-code is as follows:
\end{definition}
Remark that mutiplying a number by the base \( b \) is equivalent to shifting the number one position to the left (adding a zero at the end).

\begin{eg}
    For example, multiplying \( 23_5 \) and \( 14_5 \) in base \( 5 \):
    \[
        \begin{array}{c@{}c@{}c@{}c}
          &   & 2 & 3_5 \\
        \times &   & 1 & 4_5 \\
        \hline
          & 2 & 0 & 2_5 \\ 
        + & 2 & 3 & 0_5 \\
        \hline
          & 4 & 3 & 2_5 \\
        \end{array}
    \]
    Here, \(3\times 4 = 12_{10} = 22_5\) (write down \(2\), carry \(2\)), then \(2\times 4 + 2 = 10_{10} = 20_5\) (write down \(0\), carry \(2\)), so the partial product is \(202_5\); the other partial product is \(23_5\) shifted to \(230_5\); adding these gives \(202_5 + 230_5 = 432_5\).
\end{eg}

\section{Counting}
\begin{definition}[Counting]
    Counting is ubiquitous in mathematics (e.g., combinatorics) and computer science. It involves determining the number of elements in a set or the number of ways to arrange or select items. Various techniques such as permutations, combinations, and the principle of inclusion-exclusion are used in counting problems.
\end{definition}
Let's introduce some notations that will be useful in counting:
\begin{itemize}[itemsep=1pt,label=$\circ$]
    \item Sequences are ordered.
    \item $X$ will be refered to as the alphabet.
    \item Often $s(1),s(2), \ldots, s(n)$ will be denoted as $s = s_1,s_2, \ldots, s_n$.
    \item Sequences will be used interchangeably with strings/words.
\end{itemize}

\begin{eg}
    Given the set of vowels \( X = \{a, e, i, o, u\} \), the number of possible 4-letter sequences (words) that can be formed is:
    \[
        |X|^4 = 5^4 = 625
    \]
    since each position in the sequence can be filled by any of the 5 vowels.
\end{eg}

\begin{theorem}[Product Rule]
    If a task can be broken down into \( k \) sequential steps, where the first step can be performed in \( n_1 \) ways, the second step in \( n_2 \) ways, and so on up to the \( k \)-th step which can be performed in \( n_k \) ways, then the total number of ways to perform the entire task is given by:
    \[
        N = n_1 \times n_2 \times \ldots \times n_k = \prod_{i=1}^{k} n_i
    \]
\end{theorem}
Remark that the set from which we choose the elements can change at each step but the set must not depend on the previous choices.

\begin{eg}
    If a license plate consists of 2 letters followed by 3 digits, and there are 26 letters in the alphabet and 10 digits (0-9), the total number of different license plates that can be formed is:
    \[
        N = 26^2 \times 10^3 = 676000
    \]
\end{eg}

\begin{theorem}
    The number of different subset of a set $S$ with \( n \) elements is \( 2^n \).
\end{theorem}
\begin{proof}
    When the element of $S$ are listed in an arbitrary order, there is a one-to-one correspondence between the subsets of $S$ and the binary sequences of length $n$: the $i$-th element of $S$ is in the subset if and only if the $i$-th digit of the sequence is $1$. Since there are \( 2^n \) binary sequences of length \( n \), there are \( 2^n \) subsets of \( S \).
\end{proof}
Remark that if the cardinality of a set is known and a bijection can be established between this set and another set, then the cardinality of the second set is also known and it is the same as the first set.

\subsection{Counting Functions}
\begin{definition}[Counting Functions]
    Given two finite sets \( A \) and \( B \) with cardinalities \( |A| = m \) and \( |B| = n \), the number of functions from \( A \) to \( B \) is given by:
    \[
        n^m
    \]
    since each element in \( A \) can be mapped to any of the \( n \) elements in \( B \).
\end{definition}
Remark that if instead of functions, only injective (one-to-one) functions are considered, the counting changes to:
\[
    \frac{n!}{(n-m)!}
\]
provided that \( n \geq m \).

\begin{eg}
    Rob has 4 blue socks, 7 red socks, 5 white socks and 3 black socks. He likes to wear either a red sock on his left foot with a blue sock on his right foot or a white sock on his left foot with a black sock on his right foot. How many different ways can Rob wear his socks?
    \[
        \text{Total ways} = (7 \times 4) + (5 \times 3) = 28 + 15 = 43
    \]
\end{eg}

\begin{theorem}[Sum Rule]
    If a task can be performed in \( n_1 \) ways or \( n_2 \) ways (but not both), then the total number of ways to perform the task is:
    \[
        N = n_1 + n_2
    \]
\end{theorem}

\begin{eg}
    Each user on a computer system has a password, which is $6$ to $8$ characters long, where each character is an uppercase letter (A-Z) or a digit (0-9). How many different passwords are possible?
    \[
        \text{Total passwords} = 36^6 + 36^7 + 36^8 = 2,901,650,853,888
    \]
\end{eg}

\begin{eg}
    Same example as before but now the password must contain at least one digit. How many different passwords are possible?
    \[
        \text{Total passwords} = (36^6 - 26^6) + (36^7 - 26^7) + (36^8 - 26^8) = 2,743,303,001,088
    \]
\end{eg}

\begin{theorem}[Substraction Rule]
    If a task can be performed in \( n \) ways, and \( m \) of these ways are not allowed, then the total number of ways to perform the task is:
    \[
        N = n - m
    \]
\end{theorem}

\begin{eg}
    How many bit strings of length $8$ either start with a 1 bit or end with the two bits 00?
    \[
        \text{Total bit strings} = 2^7 + 2^6 - 2^5 = 160
    \]
    Note that the bit strings that both start with a 1 and end with 00 have been subtracted once to avoid double counting. \\
    We could also view this example has two sets: $A$ the set of bit strings of length $8$ that start with a 1 and $B$ the set of bit strings of length $8$ that end with 00. The cardinality of the union of these two sets is given by:
    \[
        |A \cup B| = |A| + |B| - |A \cap B| = 2^7 + 2^6 - 2^5 = 160
    \]
\end{eg}

\begin{eg}
    How many integers from $1$ to $100$ are not divisible by $2$ or $5$?
    \[
        \text{Total integers} = 100 - (50 + 20 - 10) = 40
    \]
    Here, $50$ integers are divisible by $2$, $20$ integers are divisible by $5$ and $10$ integers are divisible by both $2$ and $5$.
\end{eg}

\section{Permutations and Combinations}
Let's assume that $S = \{1,2,3,4\}$. The number of ways to build a strings of length $2$ from the elements of $S$:
\vskip0.3cm
\begin{center}
    \begin{tabular}{p{0.30\textwidth} | p{0.30\textwidth} | p{0.30\textwidth}}
        & & \\
        & {\centering \textbf{Permutation} \par} & {\centering \textbf{Combination} \par} \\ 
        & & \\ \hline 
        & & \\ 
        {\centering \textbf{without repetition} \par} & { \centering
            $\begin{array}{cccc}
                & 12 & 13 & 14 \\
                21 & & 23 & 24 \\
                31 & 32 & & 34 \\
                41 & 42 & 43 &
            \end{array}$ \par
        } & {\centering
        $\begin{array}{cccc}
            & 12 & 13 & 14 \\
            & & 23 & 24 \\
            & & & 34 \\
            & & &
        \end{array}$\par} \\ 
        & & \\ \hline
        & & \\
        {\centering \textbf{with repetition} \par} & { \centering
            $\begin{array}{cccc}
                11 & 12 & 13 & 14 \\
                21 & 22 & 23 & 24 \\
                31 & 32 & 33 & 34 \\
                41 & 42 & 43 & 44
            \end{array}$ \par
        } & {\centering 
            $\begin{array}{cccc}
                11 & 12 & 13 & 14 \\
                 & 22 & 23 & 24 \\
                 &    & 33 & 34 \\
                 &    &    & 44
            \end{array}$
        \par} \\ 
        & & \\
    \end{tabular}
\end{center}
\subsection{Permutations}
\begin{definition}[Permutations]
    A permutation of a set of \( n \) distinct elements is an arrangement of all the elements in a specific order. The number of different permutations of \( n \) distinct elements is given by:
    \[
        n! = n \times (n-1) \times (n-2) \times \ldots \times 2 \times 1
    \]
\end{definition}
Remark that if only \( r \) elements are to be arranged from a set of \( n \) distinct elements, the number of different permutations is given by:
\[
    P(n, r) = \frac{n!}{(n-r)!}
\]

\begin{eg}
    Let's take two similar examples:
    \begin{itemize}[itemsep=1pt,label=$\circ$]
        \item A class of $100$ sutdents is electing a president, a vice-president and a secretary. How many different ways can these positions be filled?
        \[
            P(100, 3) = \frac{100!}{(100-3)!} = 970200
        \]
        \item A class of $100$ is electing $3$ representatives. How many different ways can these positions be filled? Let's denote the representatives as $R_1$, $R_2$ and $R_3$ to distinguish them, then the number of different ways to fill these positions is:
        \[
            \begin{array}{ccc}
                R_1 & R_2 & R_3 \\
                R_1 & R_3 & R_2 \\
                R_2 & R_1 & R_3 \\
                 & \vdots & \\
                R_3 & R_2 & R_1 \\
            \end{array}
        \]
        There are \( 3! = 6 \) ways to arrange the representatives for each selection of \( 3 \) students. Therefore, the total number of different ways to select the representatives is:
        \[
            C(n, k) = \frac{P(n, k)}{k!} = \begin{pmatrix}
                n \\ k
            \end{pmatrix}= \frac{P(100, 3)}{3!} = 161700
        \]
    \end{itemize}
\end{eg}

\subsection{Combinations}

\begin{definition}[Binomial Coefficient]
    The binomial coefficient is defined as:
    \[
        \begin{pmatrix}
            n \\ k
        \end{pmatrix} = \frac{n!}{k!(n-k)!}
    \]
\end{definition}

\begin{theorem}[Binomial Theorem]
    For any non-negative integer \( n \) and any real numbers \( x \) and \( y \):
    \[
        (x + y)^n = \sum_{k=0}^{n} \begin{pmatrix}
            n \\ k
        \end{pmatrix} x^{n-k} y^k
    \]
\end{theorem}
% Remark that this is the same coefficient used to expand the binomial expression \( (x + y)^n \) using the Binomial Theorem.
\begin{proof}
    Let's prove the Binomial Theorem using generating functions. Let from compute for a given $z$:
    \[
        F(z) = \sum_{n=0}^{\infty} (x + y)^n z^n
    \]
    which is a geometric series and thus can be rewritten as:
    \[
        F(z) = \frac{1}{1 - (x + y)z} = \frac{1}{1 - xz - yz}
    \]
    Remark that this series converges if $|z| < \frac{1}{|x+y|}$. Let's then compute:
    \[
        G(z) = \sum_{n=0}^{\infty} \left( \sum_{k=0}^{n} \begin{pmatrix}
            n \\ k
        \end{pmatrix} x^{n-k} y^k \right) z^n
    \]
    If it can be shown that $G(z) = F(z)$ for $|z| < \frac{1}{|x+y|}$, then the Binomial Theorem holds. Let's compute $G(z)$:
        \begin{align*}
            G(z) &= \sum_{n=0}^{\infty} \left( \sum_{k=0}^{n} \begin{pmatrix}
            n \\ k
        \end{pmatrix} x^{n-k} y^k \right) z^n = \sum_{n = 0}^{\infty} \sum_{k = 0}^{n} \frac{n(n-1) \ldots (n-k+1)}{k!} (xz)^{n-k} (yz)^k \\
        &= \sum_{k = 0}^{\infty} \sum_{n = k}^{\infty} \frac{n(n-1) \ldots (n-k+1)}{k!} (xz)^{n-k} (yz)^k \\
        &= \sum_{k = 0}^{\infty} \frac{(yz)^k}{k!} \sum_{n = k}^{\infty} \underbrace{n(n-1) \ldots (n - k + 1) (xz)^{n-k}}_{\frac{d^k}{du^k} u^k} \\
        &= \sum_{k = 0}^{\infty} \frac{(yz)^k}{k!} \frac{d^k}{du^k} \underbrace{\left( \sum_{n = k}^{\infty} u^n \right)}_{= \frac{1}{1 - u} - 1- u -u^2 - \ldots - u^{k-1}} \Bigg|_{u = xz} = \sum_{k = 0}^{\infty} \frac{(yz)^k}{k!} \frac{d^k}{du^k} \left( \frac{u^k}{1 - u} \right) \Bigg|_{u = xz}\\
        &= \sum_{k = 0}^{\infty} \frac{(yz)^k}{k!} \cdot \frac{k!}{(1-u)^{j + 1}} \Bigg|_{u = xz} = \sum_{k = 0}^{\infty} \frac{(yz)^k}{(1 - xz)^{k + 1}}  \\
        &= \frac{1}{1 - xz} \sum_{k = 0}^{\infty} \left( \frac{yz}{1 - xz} \right)^k = \frac{1}{1 - xz} \cdot \frac{1}{1 - \frac{yz}{1 - xz}}  \frac{1}{1 - xz - yz} = F(z)\\
        \end{align*}
\end{proof}

\begin{eg}
    Let's expend the binomial expression \( (x + y)^4 \) using the Binomial Theorem:
    \[
        (x + y)^4 = \sum_{k=0}^{4} \begin{pmatrix}
            4 \\ k
        \end{pmatrix} x^{4-k} y^k = \begin{pmatrix}
            4 \\ 0
        \end{pmatrix} x^4 + \begin{pmatrix}
            4 \\ 1
        \end{pmatrix} x^3 y + \begin{pmatrix}
            4 \\ 2
        \end{pmatrix} x^2 y^2 + \begin{pmatrix}
            4 \\ 3
        \end{pmatrix} x y^3 + \begin{pmatrix}
            4 \\ 4
        \end{pmatrix} y^4
    \]
    Calculating the binomial coefficients, we get:
    \[
        (x + y)^4 = 1 \cdot x^4 + 4 \cdot x^3 y + 6 \cdot x^2 y^2 + 4 \cdot x y^3 + 1 \cdot y^4
    \]
\end{eg}

\begin{theorem}[Pascal's Identity]
    For any non-negative integer \( n \) and any integer \( k \) such that \( 0 < k < n \):
    \[
        \begin{pmatrix}
            n \\ k
        \end{pmatrix} = \begin{pmatrix}
            n-1 \\ k-1
        \end{pmatrix} + \begin{pmatrix}
            n-1 \\ k
        \end{pmatrix}
    \]
\end{theorem}
\begin{proof}
    Let's directly compute:
    \[
        \begin{pmatrix}
            n-1 \\ k-1
        \end{pmatrix} + \begin{pmatrix}
            n-1 \\ k
        \end{pmatrix} = \frac{(n-1)!}{(k-1)!(n-k)!} + \frac{(n-1)!}{k!(n-1-k)!}
    \]
    Finding a common denominator, we get:
    \[
        = \frac{(n-1)!k + (n-1)!(n-k)}{k!(n-k)!} = \frac{(n-1)!(k + n - k)}{k!(n-k)!} = \frac{(n-1)!n}{k!(n-k)!} = \frac{n!}{k!(n-k)!} = \begin{pmatrix}
            n \\ k
        \end{pmatrix}
    \]
\end{proof}
Remark that this identity is more commonly visualized using Pascal's Triangle, where each entry is the sum of the two entries directly above it.

\begin{theorem}
    The sum of the binomial coefficients for a given \( n \) ($n\geq 0$) is equal to \( 2^n \):
    \[
        \sum_{k=0}^{n} \begin{pmatrix}
            n \\ k
        \end{pmatrix} = 2^n
    \]
\end{theorem}
\begin{proof}
    This can be proven using the Binomial Theorem. Setting \( x = 1 \) and \( y = 1 \) in the theorem gives:
    \[
        (1 + 1)^n = \sum_{k=0}^{n} \begin{pmatrix}
            n \\ k
        \end{pmatrix} 1^{n-k} 1^k = \sum_{k=0}^{n} \begin{pmatrix}
            n \\ k
        \end{pmatrix}
    \]
    Since \( (1 + 1)^n = 2^n \), we have:
    \[
        \sum_{k=0}^{n} \begin{pmatrix}
            n \\ k
        \end{pmatrix} = 2^n
    \]
\end{proof}

\begin{theorem}[Combinations]
    The number of ways to choose \( k \) elements from a set of \( n \) distinct elements, where the order of selection does not matter, is given by the binomial coefficient:
    \[
        C(n, k) = \begin{pmatrix}
            n \\ k
        \end{pmatrix} = \frac{n!}{k!(n-k)!}
    \]
\end{theorem}
Remark that \( C(n, k) = C(n, n-k) \) since choosing \( k \) elements to include is equivalent to choosing \( n-k \) elements to exclude.

\begin{eg}
    How many poker hands of $5$ cards can be dealt from a standard deck of $52$ cards?
    \[
        C(52, 5) = \begin{pmatrix}
            52 \\ 5
        \end{pmatrix} = \frac{52!}{5!(52-5)!} = 2,598,960
    \]
\end{eg}

\begin{eg}
    Based on the previous example, how many poker hands with a full house (can be three of a kind and a pair) can be dealt from a standard deck of $52$ cards?
    \[
        \text{Total full house hands} = 13 \times C(4, 3) \times 12 \times C(4, 2) = 3,744
    \]
    Here, \( 13 \) is the number of ranks for the three of a kind, \( C(4, 3) \) is the number of ways to choose \( 3 \) suits from \( 4 \), \( 12 \) is the number of remaining ranks for the pair, and \( C(4, 2) \) is the number of ways to choose \( 2 \) suits from \( 4 \).
\end{eg}

% TODO: add fruits example plus demo that was cutted from the video (w10c1 end and review w10c2 begining)
\begin{eg}
    A fruit shop offers \( 5 \) types of fruits: apples, bananas, cherries, dates, and elderberries. A customer wants to buy a fruit basket containing \( 8 \) pieces of fruit, where the order of selection does not matter and repetitions are allowed. How many different fruit baskets can the customer create? \\
    This problem can be solved using the "Stars and Bars" method. We represent the \( r=8 \) fruits as stars (\(*\)) and use \( n-1 \) bars (\(|\)) to separate the \( n=5 \) different types of fruit. We need \( 5-1 = 4 \) bars to create \( 5 \) compartments (bins). \\
    For example, if we select 2 apples, 3 bananas, 1 cherry, 0 dates, and 2 elderberries, this corresponds to the string:
    \[
        **|***|*||**
    \]
    Any arrangement of these \( 8 \) stars and \( 4 \) bars represents a unique fruit basket. The total length of the string is \( 8 + 4 = 12 \). The problem reduces to choosing which \( 8 \) of the \( 12 \) positions contain stars (or equivalently, which \( 4 \) positions contain bars). \\
    The formula for the number of combinations with repetition is thus given by:
    \[
        C(n + r - 1, r) = \begin{pmatrix}
            n + r - 1 \\ r
        \end{pmatrix}
    \]
    In this case, \( n = 5 \) and \( r = 8 \):
    \[
        \text{Total fruit baskets} = \begin{pmatrix}
            5 + 8 - 1 \\ 8
        \end{pmatrix} = \begin{pmatrix}
            12 \\ 8
        \end{pmatrix} = \frac{12!}{8!4!} = 495
    \]
    Therefore, the customer can create \( 495 \) different fruit baskets.
\end{eg}

% \subsection{Permutations vs Combinations}
% Let's assume that $S = \{1,2,3,4\}$. The number of ways to build a strings of length $2$ from the elements of $S$:
% \vskip0.3cm
% \begin{center}
%     \begin{tabular}{p{0.30\textwidth} | p{0.30\textwidth} | p{0.30\textwidth}}
%         & & \\
%         & {\centering \textbf{Permutation} \par} & {\centering \textbf{Combination} \par} \\ 
%         & & \\ \hline 
%         & & \\ 
%         {\centering \textbf{without repetition} \par} & { \centering
%             $\begin{array}{cccc}
%                 & 12 & 13 & 14 \\
%                 21 & & 23 & 24 \\
%                 31 & 32 & & 34 \\
%                 41 & 42 & 43 &
%             \end{array}$ \par
%         } & {\centering
%         $\begin{array}{cccc}
%             & 12 & 13 & 14 \\
%             & & 23 & 24 \\
%             & & & 34 \\
%             & & &
%         \end{array}$\par} \\ 
%         & & \\ \hline
%         & & \\
%         {\centering \textbf{with repetition} \par} & { \centering
%             $\begin{array}{cccc}
%                 11 & 12 & 13 & 14 \\
%                 21 & 22 & 23 & 24 \\
%                 31 & 32 & 33 & 34 \\
%                 41 & 42 & 43 & 44
%             \end{array}$ \par
%         } & {\centering 
%             $\begin{array}{cccc}
%                 11 & 12 & 13 & 14 \\
%                  & 22 & 23 & 24 \\
%                  &    & 33 & 34 \\
%                  &    &    & 44
%             \end{array}$
%         \par} \\ 
%         & & \\
%     \end{tabular}
% \end{center}

% \begin{eg}
%     How many different 4-digit even numbers can be formed? We assume that the digits can be repeated.
%     \[
%         \text{Total even numbers} = 9 \times 10 \times 10 \times 5 = 4500
%     \]
%     Here, the first digit can be any digit from \( 1 \) to \( 9 \) (9 options because it cannot be 0 otherwise the number would not be a 4-digit number), the second and third digits can be any digit from \( 0 \) to \( 9 \) (10 options each), and the last digit must be an even digit (0, 2, 4, 6, or 8) (5 options).
% \end{eg}

\subsection{Permutations with Indistinguishable Objects}
\begin{theorem}[Permutations with Indistinguishable Objects]
    The number of distinct permutations of \( n \) objects, where there are \( n_1 \) indistinguishable objects of type 1, \( n_2 \) indistinguishable objects of type 2, ..., and \( n_k \) indistinguishable objects of type \( k \) (with \( n_1 + n_2 + ... + n_k = n \)), is given by:
    \[
        \frac{n!}{n_1! \times n_2! \times ... \times n_k!}
    \]
\end{theorem}
\begin{proof}
    Consider a set of \( n \) objects, where \( n_1 \) are of type 1, \( n_2 \) are of type 2, ..., and \( n_k \) are of type \( k \). The total number of permutations of these \( n \) objects, if they were all distinguishable, would be \( n! \). However, since the objects of the same type are indistinguishable, we need to account for the overcounting that occurs when we permute the indistinguishable objects among themselves. \\
    For each type \( i \), there are \( n_i! \) ways to arrange the \( n_i \) indistinguishable objects of that type. Therefore, to find the number of distinct permutations, we divide the total permutations \( n! \) by the product of the factorials of the counts of each type:
    \[
        \text{Distinct permutations} = \frac{n!}{n_1! \times n_2! \times ... \times n_k!}
    \]
    This formula gives us the correct count of distinct arrangements by eliminating the overcounting due to indistinguishable objects.
\end{proof}
Remark that this theorem could also be proved using the product rule by considering the selection of positions for each type of indistinguishable object sequentially ($C(n,n_1) \cdot C(n-n_1,n_2) \cdot \ldots \cdot C(n-n_1-...-n_{k-1}, n_k)$). which would lead to the same result.

\begin{eg}
    How many different strings can be made by reordering the letters of the word "SUCCESS"?
    \[
        \text{Total strings} = \frac{7!}{3!2!1!1!} = 420
    \]
    Here, the total number of letters is \( 7 \), with the letter 'S' appearing \( 3 \) times, 'C' appearing \( 2 \) times, and 'U' and 'E' appearing \( 1 \) time each. The formula accounts for the indistinguishable letters by dividing by the factorial of their counts.
\end{eg}

\subsection{Repetition vs. Indistinguishable Objects}
\begin{eg}
    Let's suppose we have the following alphabet $S = \{s,u,c\}$ and we build strings of length $4$ from the elements of $S$. \\
    \textbf{If repetition is allowed}, the total number of strings is:
    \[
        |S|^4 = 3^4 = 81
    \]
    \textbf{If repetition are not allowed}, then the number of strings with the alphabet $S' = \{s,s,u,c\}$ is:
    \[
        \text{Total strings} = \frac{4!}{2!1!1!} = 12
    \]
    Here, the total number of letters is \( 4 \), with the letter 's' appearing \( 2 \) times, and 'u' and 'c' appearing \( 1 \) time each. The formula accounts for the indistinguishable letters by dividing by the factorial of their counts.
\end{eg}

\section{Pigeonhole Principle}
\begin{theorem}[Pigeonhole Principle]
    If \( n \) items are put into \( m \) containers, with \( n > m \), then at least one container must contain more than one item.
\end{theorem}
\begin{proof}
    Assume, for the sake of contradiction, that no container contains more than one item. This means that each container can hold at most one item. Therefore, the maximum number of items that can be placed in \( m \) containers is \( m \). However, since \( n > m \), this leads to a contradiction because we have more items than the maximum capacity of the containers. Hence, our assumption is false, and at least one container must contain more than one item.
\end{proof}

\begin{theorem}[Generalized Pigeonhole Principle]
    If \( n \) items are put into \( m \) containers, then at least one container must contain at least \( \lceil n/m \rceil \) items.
\end{theorem}
\begin{proof}
    Assume, for the sake of contradiction, that no container contains \( \lceil n/m \rceil \) or more items. This means that each container can hold at most \( \lceil n/m \rceil - 1 \) items. Therefore, the maximum number of items that can be placed in \( m \) containers is:
    \[
        m \times (\lceil n/m \rceil - 1)
    \]
    Since \( \lceil n/m \rceil - 1 < n/m \), we have:
    \[
        m \times (\lceil n/m \rceil - 1) < m \times (n/m) = n
    \]
    This leads to a contradiction because we have more items than the maximum capacity of the containers. Hence, our assumption is false, and at least one container must contain at least \( \lceil n/m \rceil \) items.
\end{proof}

\begin{eg}
    Let's show that among $100$ students, at least one month has at least $9$ students born in that month. \\
    There are $12$ months in a year, so we have $n = 100$ students and $m = 12$ months. Using the generalized pigeonhole principle:
    \[
        \text{At least one month has } \lceil 100/12 \rceil = 9 \text{ students.}
    \]
\end{eg}

\subsection{Minimum Number of Items to Guarantee a Certain Outcome}
\begin{definition}[Minimum Number of Items to Guarantee a Certain Outcome]
    To guarantee that at least \( k \) items are in the same container when \( n \) items are distributed among \( m \) containers, we need at least:
    \[
        n = m \times (k - 1) + 1
    \]
    items.
\end{definition}

\begin{eg}
    What is the minimum number of students required in a physics class to guarantee that at least $6$ students receive the same grade, assuming the possible grades are A, B, C, D, and F? \\
    There are $5$ possible grades, so we have $m = 5$ grades. To ensure that at least one grade has at least $6$ students, we can use the generalized pigeonhole principle:
    \[
        n = m \times (k - 1) + 1 = 5 \times (6 - 1) + 1 = 26
    \]
    Therefore, a minimum of $26$ students is required to guarantee that at least $6$ students receive the same grade.
\end{eg}

\begin{eg}
    How many numbers must be selected from the set $\{1,2,3,4,5,6\}$ to guarantee that at least one pair of these numbers add up to $7$? \\
    The pairs that add up to $7$ are: $(1,6)$, $(2,5)$, and $(3,4)$. Thus, we have $m = 3$ pairs. To ensure that at least one pair is selected, we can use the generalized pigeonhole principle:
    \[
        n = m \times (k - 1) + 1 = 3 \times (1 - 1) + 1 = 4
    \]
    Therefore, a minimum of $4$ numbers must be selected to guarantee that at least one pair adds up to $7$.
\end{eg}

\begin{eg}
    How many cards must be selected from a standard deck of $52$ cards to guarantee that at least $3$ cards of the same suit are selected? \\
    There are $4$ suits in a standard deck of cards (hearts, diamonds, clubs, spades), so we have $m = 4$ suits. To ensure that at least one suit has at least $3$ cards selected, we can use the generalized pigeonhole principle:
    \[
        n = m \times (k - 1) + 1 = 4 \times (3 - 1) + 1 = 9
    \]
    Therefore, a minimum of $9$ cards must be selected to guarantee that at least $3$ cards of the same suit are selected.
\end{eg}

\begin{eg}
    How many cards must be selected from a standard deck of $52$ cards to guarantee that at least $3$ hearts are chosen? \\
    There are $13$ hearts in a standard deck of cards, so to ensure that at least $3$ hearts are selected, we can consider the worst-case scenario where we select all non-heart cards first. There are $52 - 13 = 39$ non-heart cards. To guarantee that at least $3$ hearts are selected, we need to select:
    \[
        n = 39 + 3 = 42
    \]
    Therefore, a minimum of $42$ cards must be selected to guarantee that at least $3$ hearts are chosen.
\end{eg}

% TODO: maybe move this section to the example of the bijection between the number of subsets and the number of binary sequences of length n
\section{Combinatorial Proofs}
\begin{definition}[Combinatorial Proofs]
    A combinatorial proof is a method of proving mathematical identities by counting the same set of objects in two different ways. This approach often provides a more intuitive understanding of the identity being proved. There are two main steps in a combinatorial proof:
    \begin{itemize}[itemsep=1pt,label=$\circ$]
        \item Showing that there is a bijection between the set being counted by the two side of the identity.
        \item Prove that both sides of the identity count the same objects but in different ways.
    \end{itemize}
\end{definition}

\subsection{Bijection Principle}
\begin{theorem}[Bijection Principle]
    If there exists a bijection between two finite sets \( A \) and \( B \), then the cardinalities of the sets are equal, i.e., \( |A| = |B| \).
\end{theorem}
\begin{proof}
    A bijection is a one-to-one correspondence between the elements of two sets. This means that for every element in set \( A \), there is a unique element in set \( B \), and vice versa. Since each element in \( A \) can be paired with exactly one element in \( B \), the number of elements in both sets must be the same. Therefore, if there exists a bijection between \( A \) and \( B \), it follows that \( |A| = |B| \).
\end{proof}

\begin{eg}
    Prove that \( C(n, k) = C(n, n-k) \) using a combinatorial proof. \\
    Consider a set \( S \) with \( n \) elements. The left side of the identity, \( C(n, k) \), counts the number of ways to choose \( k \) elements from the set \( S \). The right side of the identity, \( C(n, n-k) \), counts the number of ways to choose \( n-k \) elements from the same set \( S \). \\
    There is a bijection between the two sets of choices: for every subset of \( k \) elements chosen from \( S \), there is a corresponding subset of \( n-k \) elements that are not chosen. This means that choosing \( k \) elements is equivalent to choosing which \( n-k \) elements to leave out. Therefore, both sides of the identity count the same number of subsets, leading to the conclusion that:
    \[
        C(n, k) = C(n, n-k)
    \]
\end{eg}

\subsection{Double Counting Principle}
\begin{theorem}[Double Counting Principle]
    If a set \( S \) can be counted in two different ways, then the two counts must be equal.
\end{theorem}
\begin{proof}
    Let \( S \) be a set that can be counted in two different ways, resulting in counts \( A \) and \( B \). Since both counts represent the same set \( S \), they must be equal. Therefore, we have:
    \[
        A = B
    \]
\end{proof}

\begin{eg}
    Let's prove the identity \( \sum_{k=0}^{n} C(n, k) = 2^n \) using a combinatorial proof. \\
    The left side of the identity, \( \sum_{k=0}^{n} C(n, k) \), counts the total number of subsets of a set with \( n \) elements. This is because \( C(n, k) \) represents the number of ways to choose \( k \) elements from the set, and summing over all possible values of \( k \) gives the total number of subsets. \\
    The right side of the identity, \( 2^n \), counts the same set of subsets by considering that each element in the set can either be included in a subset or not. Since there are \( n \) elements, and each element has \( 2 \) choices (to be included or not), the total number of subsets is \( 2^n \). \\
    Since both sides of the identity count the same set of subsets, we conclude that:
    \[
        \sum_{k=0}^{n} C(n, k) = 2^n
    \]
\end{eg}

\section{Counting with Recurrence Relations}
\begin{itemize}[itemsep=1pt,label=$\circ$]
    \item Some counting problems cannot (easily) be solved with the methods introduced so far (for example, How many bit strings of length $n$ do not contain two consecutive zeros?).
    \item In these cases, recurrence relations are a powerful tool to solve counting problems.
\end{itemize}
Remember that a recurrence relation is an equation that recursively defines a sequence, where each term is defined as a function of its preceding terms. A sequence is called a solution of the recurrence relation if it satisfies the relation and a recurrence relation recursively defines a sequence. \\
To solve a counting problem using recurrence relations, generally follow these steps:
\begin{itemize}[itemsep=1pt,label=$\circ$]
    \item Define a set $P_n$ depending on a parameter $n$.
    \item Describe $P_n$ in terms of $P_{n-1}, P_{n-2}, \ldots, P_{n - k}$.
    \item Derive a recurrence relation for $|P_n|$.
    \item Solve the recurrence relation.
\end{itemize}

\begin{eg}
    What is the number of permutations (whitout repetition) of a set with $n$ elements? \\
    Let's define the set \( P_n \) as the set of all permutations of a set with \( n \) elements. To form a permutation of \( n \) elements, we can start by choosing one of the \( n \) elements to be the first element in the permutation. Once we have chosen the first element, we are left with \( n-1 \) elements to arrange. The number of ways to arrange these \( n-1 \) elements is given by \( |P_{n-1}| \). \\
    Therefore, we can express \( |P_n| \) in terms of \( |P_{n-1}| \) as follows:
    \[
        |P_n| = n \times |P_{n-1}|
    \]
    This gives us the recurrence relation:
    \[
        |P_n| = n \times |P_{n-1}| \quad \text{for } n \geq 1
    \]
    with the base case:
    \[
        |P_1| = 1
    \]
    To solve this recurrence relation, we can expand it:
    \begin{align*}
        |P_n| &= n \times |P_{n-1}| = n \times (n-1) \times |P_{n-2}| \\
        &= n \times (n-1) \times (n-2) \times |P_{n-3}| \\
        &= \ldots = n \times (n-1) \times (n-2) \times \ldots \times 1 \times |P_1|
    \end{align*}
    Since \( |P_1| = 1 \), we have:
    \[
        |P_n| = n!
    \]
    Therefore, the number of permutations of a set with \( n \) elements is \( n! \).
\end{eg}

\begin{eg}
    Find a recursion relation for the number of way to climb a staircase with \( n \) steps, where at each step you can climb either \( 1 \) step or \( 2 \) steps. \\
    Let's define \( P_n \) as the number of ways to climb a staircase with \( n \) steps. To reach the \( n \)-th step, you can either:
    \begin{itemize}[itemsep=1pt,label=$\circ$]
        \item Climb \( 1 \) step from the \( (n-1) \)-th step. The number of ways to reach the \( (n-1) \)-th step is \( |P_{n-1}| \).
        \item Climb \( 2 \) steps from the \( (n-2) \)-th step. The number of ways to reach the \( (n-2) \)-th step is \( |P_{n-2}| \).
    \end{itemize}
    Therefore, we can express \( |P_n| \) in terms of \( |P_{n-1}| \) and \( |P_{n-2}| \) as follows:
    \[
        |P_n| = |P_{n-1}| + |P_{n-2}|
    \]
    This gives us the recurrence relation:
    \[
        |P_n| = |P_{n-1}| + |P_{n-2}| \quad \text{for } n \geq 2
    \]
    with the base cases:
    \[
        |P_0| = 1 \quad \text{(1 way to stay at the ground level)}
    \]
    \[
        |P_1| = 1 \quad \text{(1 way to climb 1 step)}
    \]
    This recurrence relation is similar to the Fibonacci sequence, and the number of ways to climb a staircase with \( n \) steps can be computed using this relation.
\end{eg}

\begin{eg}
    Let's find the minimum number of steps to solve the Tower of Hanoi problem with \( n \) disks. \\
    Let's define \( P_n \) as the minimum number of steps required to move \( n \) disks from one peg to another, following the rules of the Tower of Hanoi. To move \( n \) disks, we can follow these steps:
    \begin{itemize}[itemsep=1pt,label=$\circ$]
        \item Move the top \( n-1 \) disks from the source peg to an auxiliary peg. This requires \( |P_{n-1}| \) steps.
        \item Move the \( n \)-th disk (the largest disk) directly from the source peg to the destination peg. This requires \( 1 \) step.
        \item Move the \( n-1 \) disks from the auxiliary peg to the destination peg. This again requires \( |P_{n-1}| \) steps.
    \end{itemize}
    Therefore, we can express \( |P_n| \) in terms of \( |P_{n-1}| \) as follows:
    \[
        |P_n| = 2 \times |P_{n-1}| + 1
    \]
    This gives us the recurrence relation:
    \[
        |P_n| = 2 \times |P_{n-1}| + 1 \quad \text{for } n \geq 1
    \]
    with the base case:
    \[
        |P_0| = 0 \quad \text{(0 steps needed to move 0 disks)}
    \]
    To solve this recurrence relation, we can expand it:
    \begin{align*}
        |P_n| &= 2 \times |P_{n-1}| + 1 = 2 \times (2 \times |P_{n-2}| + 1) + 1 \\
        &= 2^2 \times |P_{n-2}| + 2 + 1 \\
        &= 2^3 \times |P_{n-3}| + 2^2 + 2 + 1 \\
        &= \ldots = 2^n \times |P_0| + (2^{n-1} + 2^{n-2} + \ldots + 2 + 1)
    \end{align*}
    Since \( |P_0| = 0 \), we have:
    \[
        |P_n| = 2^{n-1} + 2^{n-2} + \ldots + 2 + 1 = 2^n - 1
    \]
    Therefore, the minimum number of steps required to solve the Tower of Hanoi problem with \( n \) disks is \( 2^n - 1 \).
\end{eg}

\section{Advanced Counting}

\begin{eg}
    Let $A$ and $B$ be two finite sets. We want to find the cardinality of their union $A \cup B$. We showed previously that:
    \[
        |A \cup B| = |A| + |B| - |A \cap B|
    \]
    This formula accounts for the fact that when we add the cardinalities of sets \( A \) and \( B \), the elements that are in both sets (the intersection \( A \cap B \)) are counted twice. To correct for this double counting, we subtract the cardinality of the intersection \( |A \cap B| \) once.  
\end{eg}

\begin{eg}
    How many integers between $1$ and $100$ are divisible by $2$ or $5$? \\
    Let's define the sets:
    \begin{itemize}[itemsep=1pt,label=$\circ$]
        \item \( A \): the set of integers between \( 1 \) and \( 100 \) that are divisible by \( 2 \).
        \item \( B \): the set of integers between \( 1 \) and \( 100 \) that are divisible by \( 5 \).
    \end{itemize}
    We want to find the cardinality of the union \( A \cup B \), which represents the integers that are divisible by \( 2 \) or \( 5 \). \\
    First, we calculate the cardinalities of sets \( A \), \( B \), and their intersection \( A \cap B \):
    \[
        |A| = \left\lfloor \frac{100}{2} \right\rfloor = 50
    \]
    \[
        |B| = \left\lfloor \frac{100}{5} \right\rfloor = 20
    \]
    \[
        |A \cap B| = \left\lfloor \frac{100}{10} \right\rfloor = 10
    \]
    Now, we can use the formula for the cardinality of the union:
    \[
        |A \cup B| = |A| + |B| - |A \cap B| = 50 + 20 - 10 = 60
    \]
    Therefore, there are \( 60 \) integers between \( 1 \) and \( 100 \) that are divisible by \( 2 \) or \( 5 \).
\end{eg}

\begin{eg}
    How many integers between $1$ and $1000$ are divisible by $5$, $7$ and $11$? \\
    Let's define the sets:
    \begin{itemize}[itemsep=1pt,label=$\circ$]
        \item \( A \): the set of integers between \( 1 \) and \( 1000 \) that are divisible by \( 5 \).
        \item \( B \): the set of integers between \( 1 \) and \( 1000 \) that are divisible by \( 7 \).
        \item \( C \): the set of integers between \( 1 \) and \( 1000 \) that are divisible by \( 11 \).
    \end{itemize}
    We want to find the cardinality of the union \( A \cup B \cup C \), which represents the integers that are divisible by \( 5 \), \( 7 \), or \( 11 \). \\
    First, we calculate the cardinalities of sets \( A \), \( B \), \( C \), and their pairwise and triple intersections:
    \[
        |A| = \left\lfloor \frac{1000}{5} \right\rfloor = 200
    \]
    \[
        |B| = \left\lfloor \frac{1000}{7} \right\rfloor = 142
    \]
    \[
        |C| = \left\lfloor \frac{1000}{11} \right\rfloor = 90
    \]
    \[
        |A \cap B| = \left\lfloor \frac{1000}{35} \right\rfloor = 28
    \]
    \[
        |A \cap C| = \left\lfloor \frac{1000}{55} \right\rfloor = 18
    \]
    \[
        |B \cap C| = \left\lfloor \frac{1000}{77} \right\rfloor = 12
    \]
    \[
        |A \cap B \cap C| = \left\lfloor \frac{1000}{385} \right\rfloor = 2
    \]
    Now, we can use the Inclusion-Exclusion Principle to find the cardinality of the union:
    \begin{align*}
        |A \cup B \cup C| &= |A| + |B| + |C| - |A \cap B| - |A \cap C| - |B \cap C| + |A \cap B \cap C| \\
        &= 200 + 142 + 90 - 28 - 18 - 12 + 2 \\
        &= 374
    \end{align*}
\end{eg}
If generalized to \( n \) sets, the following can be observed:
\begin{itemize}[itemsep=1pt,label=$\circ$]
    \item Include the cardinalities of individual sets.
    \item Exclude the cardinalities of all pairwise intersections.
    \item Include the cardinalities of all triple intersections.
    \item Continue this process, alternating between inclusion and exclusion, until the intersection of all \( n \) sets is included or excluded based on whether \( n \) is odd or even.
\end{itemize}
\begin{theorem}[Inclusion-Exclusion Principle]
    For any finite sets \( A_1, A_2, \ldots, A_n \), the cardinality of their union is given by:
    \begin{align*}
        |A_1 \cup A_2 \cup \ldots \cup A_n| &= \sum_{i=1}^{n} |A_i| - \sum_{1 \leq i < j \leq n} |A_i \cap A_j| \\
        &+ \sum_{1 \leq i < j < k \leq n} |A_i \cap A_j \cap A_k| - \ldots + (-1)^{n+1} |A_1 \cap A_2 \cap \ldots \cap A_n|
    \end{align*}
\end{theorem}
\begin{proof}
    Assume that an element $a$ is in $r$ sets with $1 \leq r \leq n$ i.e. $a \in A_{i_1}, A_{i_2}, \ldots, A_{i_r}$. \\
    \textbf{Counting the contribution of $a$ to the right-hand side:}
    \begin{itemize}[itemsep=1pt,label=$\circ$]
        \item In the first sum, $a$ is counted $r$ times (once for each set it belongs to).
        \item In the second sum, $a$ is counted $\binom{r}{2}$ times (once for each pair of sets it belongs to).
        \item In the third sum, $a$ is counted $\binom{r}{3}$ times (once for each triplet of sets it belongs to).
        \item This pattern continues until the last term, where $a$ is counted $(-1)^{r+1} \binom{r}{r} = (-1)^{r+1}$ times.
    \end{itemize}
    Therefore, the total contribution of $a$ to the right-hand side is:
    \[
        r - \binom{r}{2} + \binom{r}{3} - \ldots + (-1)^{r + 1} \binom{r}{r - 1}
    \]
    Which can be rewritten as:
    \[
        \sum_{k=1}^{r} (-1)^{k-1} \binom{r}{k} = \sum_{k = 1}^{r} \binom{r}{k} (1)^{r-k} (-1)^{k - 1} = (1 - 1)^r - (-1) = 1
    \]
    This shows that each element $a$ is counted exactly once on the right-hand side, which matches its contribution to the left-hand side. Therefore, the two sides are equal, proving the Inclusion-Exclusion Principle.
\end{proof}

\begin{eg}
    How many permutations of the five digits $1,2,3,4,5$ begin with a $1$, contain the digit $45$ in second and third positionsm or end with a $3$? \\
    Let's define the sets:
    \begin{itemize}[itemsep=1pt,label=$\circ$]
        \item \( A \): the set of permutations that begin with a \( 1 \).
        \item \( B \): the set of permutations that contain the digits \( 4, 5 \) in the second and third positions.
        \item \( C \): the set of permutations that end with a \( 3 \).
    \end{itemize}
    We want to find the cardinality of the union \( A \cup B \cup C \), which represents the permutations that satisfy at least one of the conditions. \\
    First, we calculate the cardinalities of sets \( A \), \( B \), \( C \), and their pairwise and triple intersections:
    \[
        |A| = 4! = 24
    \]
    \[
        |B| = 3! = 6
    \]
    \[
        |C| = 4! = 24
    \]
    \[
        |A \cap B| = 2! = 2
    \]
    \[
        |A \cap C| = 3! = 6
    \]
    \[
        |B \cap C| = 2! = 2
    \]
    \[
        |A \cap B \cap C| = 1! = 1
    \]
    Now, we can use the Inclusion-Exclusion Principle to find the cardinality of the union:
    \begin{align*}
        |A \cup B \cup C| &= |A| + |B| + |C| - |A \cap B| - |A \cap C| - |B \cap C| + |A \cap B \cap C| \\
        &= 24 + 6 + 24 - 2 - 6 - 2 + 1 \\
        &= 45
    \end{align*}
    Therefore, there are \( 45 \) permutations of the digits \( 1, 2, 3, 4, 5 \) that begin with a \( 1 \), contain the digits \( 4, 5 \) in the second and third positions, or end with a \( 3 \).
\end{eg}

\subsection{Derangements}
\begin{definition}[Derangement]
    A derangement is a permutation of a set in which none of the elements appear in their original position. In other words, for a set of \( n \) elements, a derangement is a permutation \( \sigma \) such that \( \sigma(i) \neq i \) for all \( i \) from \( 1 \) to \( n \).
\end{definition}

\begin{theorem}
    The number of derangements of a set with \( n \) elements, denoted as \( D_n \), is given by the formula:
    \[
        D_n = n! \sum_{k=0}^{n} \frac{(-1)^k}{k!}
    \]
\end{theorem}
\begin{proof}
    Let $P_i$ be the set of permutations that fix the position of element $i$. The number of derangements is the number of permutations having none of the properties $P_i$ for $i = 1, 2, \ldots, n$. Thus:
    \[
        D_n = \left(n! - \underbrace{\sum_{i = 1}^{n}}_{= n} \underbrace{|P_i|}_{= (n-1)!} + \underbrace{\sum_{i \neq j}^{n}}_{= \binom{n}{2}} \underbrace{|P_i \cap P_j|}_{= (n-2)!} - \ldots + (-1)^n |P_1 \cap P_2 \cap \ldots \cap P_n|\right)
    \]
    Therefore:
    \begin{align*}
        D_n &= n! - n (n-1)! + \binom{n}{2} (n-2)! - \ldots + (-1)^n 1 \\
        &= n! - n! + \frac{n!}{2!} - \ldots + (-1)^n \frac{n!}{n!} \\
        &= n! \left(1 - \frac{1}{1!} + \frac{1}{2!} - \ldots + (-1)^n \frac{1}{n!}\right)
    \end{align*}
\end{proof}

\begin{eg}
    An employee checks the hats of $5$ people at a restaurant, but forget to put claim check numbers on the hats. When customers returns for their hats, the checker gives them back a hat chosen at random from the remaining hats. What is the change that no one receives their correct hat? \\
    We want to find the probability that no one receives their correct hat when $5$ hats are randomly distributed among $5$ people. This is equivalent to finding the number of derangements of $5$ elements, denoted as $D_5$, and dividing it by the total number of permutations of $5$ elements, which is $5!$. \\
    Using the formula for derangements, we have:
    \[
        D_5 = 5! \sum_{k=0}^{5} \frac{(-1)^k}{k!} = 120 \left(1 - 1 + \frac{1}{2!} - \frac{1}{3!} + \frac{1}{4!} - \frac{1}{5!}\right) = 120 \left(\frac{1}{2} - \frac{1}{6} + \frac{1}{24} - \frac{1}{120}\right) = 44
    \]
    Therefore, the probability that no one receives their correct hat is:
    \[
        \text{Probability} = \frac{D_5}{5!} = \frac{44}{120} = \frac{11}{30} \approx 0.3667 \approx \frac{1}{e}
    \]
\end{eg}

\section{Countability}
\begin{definition}[Cardinality]
    The cardinality of a set \( A \), denoted as \( |A| \), is a measure of the "number of elements" in the set. For finite sets, the cardinality is simply the count of distinct elements in the set. For infinite sets, cardinality is defined using concepts such as bijections and countability.
\end{definition}

\subsection{Finite Sets}
\begin{theorem}
    Let $A$ and $B$ be two sets. If there exists an injection from $A$ to $B$, then:
    \[
        |A| \leq |B|
    \]
\end{theorem}
\begin{proof}
    An injection from set \( A \) to set \( B \) is a function \( f: A \to B \) such that for every pair of distinct elements \( a_1, a_2 \in A \), we have \( f(a_1) \neq f(a_2) \). This means that each element in \( A \) is mapped to a unique element in \( B \), and no two elements in \( A \) map to the same element in \( B \). \\
    Since each element in \( A \) corresponds to a distinct element in \( B \), the number of elements in \( A \) cannot exceed the number of elements in \( B \). Therefore, we conclude that:
    \[
        |A| \leq |B|
    \]
\end{proof}

\begin{eg}
    Let $A$ be the set of students at EPFL and $B$ be the set of sciper numbers. We know that each student has a unique sciper number, which means there exists an injection from $A$ to $B$. Therefore, we can conclude that:
    \[
        |A| \leq |B|
    \]
\end{eg}

\begin{theorem}
    Let $A$ and $B$ be two sets. If there exists a surjection from $A$ to $B$, then:
    \[
        |A| \geq |B|
    \]
\end{theorem}
\begin{proof}
    A surjection from set \( A \) to set \( B \) is a function \( f: A \to B \) such that for every element \( b \in B \), there exists at least one element \( a \in A \) such that \( f(a) = b \). This means that every element in \( B \) is "covered" by at least one element in \( A \). \\
    Since every element in \( B \) has at least one corresponding element in \( A \), the number of elements in \( A \) must be at least as large as the number of elements in \( B \). Therefore, we conclude that:
    \[
        |A| \geq |B|
    \]
\end{proof}

\begin{eg}
    Let $A$ be the set of students at EPFL and $B$ be the set of sections offered at EPFL. We know that each section has at least one student enrolled and every student is enrolled in at least one section, which means there exists a surjection from $A$ to $B$. Therefore, we can conclude that:
    \[
        |A| \geq |B|
    \]
\end{eg}

\begin{theorem}
    Let $A$ and $B$ be two sets. If there exists a bijection from $A$ to $B$, then:
    \[
        |A| = |B|
    \]
\end{theorem}
\begin{proof}
    A bijection from set \( A \) to set \( B \) is a function \( f: A \to B \) that is both an injection and a surjection. This means that for every pair of distinct elements \( a_1, a_2 \in A \), we have \( f(a_1) \neq f(a_2) \) (injection), and for every element \( b \in B \), there exists at least one element \( a \in A \) such that \( f(a) = b \) (surjection). \\
    Since each element in \( A \) corresponds to a unique element in \( B \), and every element in \( B \) is "covered" by at least one element in \( A \), the number of elements in both sets must be the same. Therefore, we conclude that:
    \[
        |A| = |B|
    \]
\end{proof}

\subsection{Infinite Sets}
\begin{eg}
    Let's compare the cardinalities of the set of natural numbers \( \mathbb{N} \) and the set of even natural numbers \( 2\mathbb{N} \). \\
    We can define a bijection \( f: \mathbb{N} \to 2\mathbb{N} \) by the function \( f(n) = 2n \). This function maps each natural number \( n \) to its double, which is an even natural number. \\
    To show that \( f \) is a bijection, we need to demonstrate that it is both an injection and a surjection:
    \begin{itemize}[itemsep=1pt,label=$\circ$]
        \item Injection: If \( f(n_1) = f(n_2) \), then \( 2n_1 = 2n_2 \). Dividing both sides by \( 2 \), we get \( n_1 = n_2 \). Therefore, \( f \) is injective.
        \item Surjection: For every even natural number \( m \in 2\mathbb{N} \), there exists a natural number \( n = \frac{m}{2} \in \mathbb{N} \) such that \( f(n) = m \). Therefore, \( f \) is surjective.
    \end{itemize}
    Since \( f \) is both an injection and a surjection, it is a bijection. Therefore, we conclude that:
    \[
        |\mathbb{N}| = |2\mathbb{N}|
    \]
\end{eg}
Remark that this result in counter-intuitive, as $2\mathbb{N}$ is a proper subset of $\mathbb{N}$.

\begin{definition}[Infinite Set]
    A set \( A \) is called infinite if there exists a bijection between \( A \) and a proper subset of \( A \). Otherwise, the set is called finite.
\end{definition}

\subsection{Countable Sets}
\begin{definition}[Countable Set]
    A set \( A \) is called countable if there exists a bijection between \( A \) and the set of natural numbers \( \mathbb{N} \) or a subset of \( \mathbb{N} \). If no such bijection exists, the set is called uncountable.
\end{definition}
Remark that when an infinite set is countable (countably infinite), its cardinality is denoted by \( \aleph_0 \) (aleph-null).

\begin{theorem}
    An infinite set $S$ is countable if and only if it is possible to list the elements of the set in a sequence indexed by the positive integers, i.e., $S = \{s_1, s_2, s_3, \ldots\}$.
\end{theorem}
\begin{proof}
    (\( \Rightarrow \)) Assume that \( S \) is countable. By definition, there exists a bijection \( f: \mathbb{N} \to S \). We can list the elements of \( S \) in a sequence as follows:
    \[
        s_1 = f(1), \quad s_2 = f(2), \quad s_3 = f(3), \quad \ldots
    \]
    This gives us a sequence \( S = \{s_1, s_2, s_3, \ldots\} \).

    (\( \Leftarrow \)) Conversely, assume that we can list the elements of \( S \) in a sequence indexed by the positive integers:
    \[
        S = \{s_1, s_2, s_3, \ldots\}
    \]
    We can define a function \( f: \mathbb{N} \to S \) by \( f(n) = s_n \). This function is a bijection because:
    \begin{itemize}[itemsep=1pt,label=$\circ$]
        \item Injection: If \( f(n_1) = f(n_2) \), then \( s_{n_1} = s_{n_2} \). Since the elements are listed in a sequence, this implies that \( n_1 = n_2 \).
        \item Surjection: For every element \( s_k \in S \), there exists a natural number \( k \) such that \( f(k) = s_k \).
    \end{itemize}
    Therefore, we conclude that \( S \) is countable.
\end{proof}

\begin{eg}
    Let's show that the set of positive odd integers $F$ is a countable set. Let $f: \mathbb{Z}^+ \to F, f(n) = 2n -1$. \\
    We will show that $f$ is a bijection:
    \begin{itemize}[itemsep=1pt,label=$\circ$]
        \item Injection: Assume that $f(n_1) = f(n_2)$. Then, $2n_1 - 1 = 2n_2 - 1$. By adding $1$ to both sides and dividing by $2$, we get $n_1 = n_2$. Therefore, $f$ is injective.
        \item Surjection: Let $m \in F$. By definition of $F$, there exists $k \in \mathbb{Z}^+$ such that $m = 2k - 1$. Thus, $f(k) = m$. Therefore, $f$ is surjective.
    \end{itemize}
    Since $f$ is both injective and surjective, it is a bijection. Therefore, we conclude that:
    \[
        |\mathbb{Z}^+| = |F|
    \]
\end{eg}
Remark that by intuition, the cardinality of a set is the "size of" the set. However, in the case of infinite sets, this intuition does not always hold, as demonstrated by the example of the set of natural numbers and the set of even natural numbers, i.e. $A \subset B \nrightarrow |A| < |B|$.

\subsection{Properties of Countable Sets}
All of the following properties hold for countable sets:
\begin{itemize}[itemsep=1pt,label=$\circ$]
    \item Any subset of a countable set is countable.
    \item The union of a finite number of countable sets is countable.
    \item The cartesian product of a finite number of countable sets is countable.
    \item The set of all strings associated to a finite alphabet is countable.
\end{itemize}
Also:
\begin{itemize}[itemsep=1pt,label=$\circ$]
    \item If there is an injective function from $A$ to $B$ and $B$ is countable, then $A$ is countable.
    \item If there is a surjective function from $A$ to $B$ and $A$ is countable, then $B$ is countable.
\end{itemize}

\begin{eg}
    Let's show that the set of integers $\mathbb{Z}$ is countable. We can define a bijection $f: \mathbb{Z}^+ \to \mathbb{Z}$ as follows:
    \[
        f(n) = \begin{cases}
            \frac{n}{2} & \text{if } n \text{ is even} \\
            -\frac{n-1}{2} & \text{if } n \text{ is odd}
        \end{cases}
    \]
    We will show that $f$ is a bijection:
    \begin{itemize}[itemsep=1pt,label=$\circ$]
        \item Injection: Assume that $f(n_1) = f(n_2)$. We consider two cases:
        \begin{itemize}[itemsep=1pt,label=$\circ$]
            \item If both $n_1$ and $n_2$ are even, then $\frac{n_1}{2} = \frac{n_2}{2}$, which implies that $n_1 = n_2$.
            \item If both $n_1$ and $n_2$ are odd, then $-\frac{n_1 - 1}{2} = -\frac{n_2 - 1}{2}$, which implies that $n_1 = n_2$.
            \item If one of $n_1$ or $n_2$ is even and the other is odd, then $f(n_1)$ and $f(n_2)$ cannot be equal, since one is non-negative and the other is negative. Therefore, $f$ is injective.
        \end{itemize}
        \item Surjection: Let $m \in \mathbb{Z}$. We consider two cases:
        \begin{itemize}[itemsep=1pt,label=$\circ$]
            \item If $m \geq 0$, then we can choose $n = 2m$. Thus, $f(n) = m$.
            \item If $m < 0$, then we can choose $n = -2m - 1$. Thus, $f(n) = m$.
        \end{itemize}
        Therefore, $f$ is surjective.
    \end{itemize}
    Since $f$ is both injective and surjective, it is a bijection. Therefore, we conclude that:
    \[
        |\mathbb{Z}^+| = |\mathbb{Z}|
    \]
\end{eg}
Remark that this result could have been obtained using the property that the union of a finite number of countable sets is countable, since:
\[
    \mathbb{Z} = \{0\} \cup \mathbb{Z}^+ \cup (-\mathbb{Z}^+)
\]

\begin{eg}[Hilbert's Hotel]
    A hotel with a countably infinite number of rooms is fully occupied. A new guest arrives and wants a room. Is it possible to accommodate this new guest? \\
    Yes, it is possible to accommodate the new guest by shifting the current guests to different rooms. We can define a function \( f: \mathbb{Z}^+ \to \mathbb{Z}^+ \) that maps each current guest's room number to a new room number:
    \[
        f(n) = n + 1
    \]
    This function shifts each guest from room \( n \) to room \( n + 1 \). As a result, room \( 1 \) becomes available for the new guest. \\
    Since \( f \) is a bijection, it shows that the hotel can accommodate the new guest while still having a countably infinite number of rooms occupied. Therefore, the hotel can successfully accommodate the new guest. \\
    Now a bus with a countably infinite number of new guests arrives. Is it possible to accommodate all these new guests? \\
    Yes, it is possible to accommodate all the new guests by shifting the current guests to different rooms. We can define a function \( g: \mathbb{Z}^+ \to \mathbb{Z}^+ \) that maps each current guest's room number to a new room number:
    \[
        g(n) = 2n
    \]
    This function shifts each guest from room \( n \) to room \( 2n \). As a result, all odd-numbered rooms become available for the new guests. \\
    We can then assign the new guests to the odd-numbered rooms as follows:
    \[
        \text{New guest } k \text{ is assigned to room } 2k - 1
    \]
    Since both \( g \) and the assignment of new guests are bijections, it shows that the hotel can accommodate all the new guests while still having a countably infinite number of rooms occupied. Therefore, the hotel can successfully accommodate all the new guests.
\end{eg}

\subsection{Uncountable Sets}

\begin{theorem}[Cantor's Diagonal Argument]
    The set of real numbers \( \mathbb{R} \) is uncountable.
\end{theorem}
\begin{proof}
    Assume, for the sake of contradiction, that the set of real numbers \( \mathbb{R} \) is countable. This means that we can list all the real numbers in a sequence:
    \[
        r_1, r_2, r_3, \ldots
    \]
    where each \( r_i \) is a real number. We can represent each real number in its decimal expansion:
    \[
        r_1 = 0.d_{11}d_{12}d_{13}\ldots
    \]
    \[
        r_2 = 0.d_{21}d_{22}d_{23}\ldots
    \]
    \[
        r_3 = 0.d_{31}d_{32}d_{33}\ldots
    \]
    and so on. Now, we will construct a new real number \( r \) that is not in the list. We define \( r \) by choosing its decimal digits as follows:
    \[
        d_n = \begin{cases}
            1 & \text{if } d_{nn} \neq 1 \\
            2 & \text{if } d_{nn} = 1
        \end{cases}
    \]
    This means that the \( n \)-th digit of \( r \) is different from the \( n \)-th digit of \( r_n \). Therefore, \( r \) cannot be equal to any \( r_n \) in the list, since it differs from each \( r_n \) at least in the \( n \)-th digit. This contradicts our assumption that we have listed all real numbers. Hence, we conclude that the set of real numbers \( \mathbb{R} \) is uncountable.
\end{proof}
Remark that when constructing the new number $r$, one must be careful of same representations (e.g. $0.4999\ldots = 0.5000\ldots$). To avoid this, one can choose digits that are neither $0$ nor $9$. The same problem arise when wanting to prove that the interval $[0,1]$ is uncountable using binary representation (since $0.1111\ldots_2 = 1.0000\ldots_2$).

% \begin{eg}
%     Let's prove that the interval \( [0,1] \) represented in binary is uncountable using Cantor's Diagonal Argument. \\
%     Assume, for the sake of contradiction, that the interval \( [0,1] \) represented in binary is countable. This means that we can list all the binary representations of numbers in \( [0,1] \) in a sequence:
%     \[
%         b_1, b_2, b_3, \ldots
%     \]
%     where each \( b_i \) is a binary representation of a number in \( [0,1] \). We can represent each binary representation as follows:
%     \[
%         b_1 = 0.b_{11}b_{12}b_{13}\ldots
%     \]
%     \[
%         b_2 = 0.b_{21}b_{22}b_{23}\ldots
%     \]
%     \[
%         b_3 = 0.b_{31}b_{32}b_{33}\ldots
%     \]
%     and so on. Now, we will construct a new binary representation \( b \) that is not in the list. We define \( b \) by choosing its binary digits as follows:
%     \[
%         b_n = \begin{cases}
%             0 & \text{if } b_{nn} = 1 \\
%             1 & \text{if } b_{nn} = 0
%         \end{cases}
%     \]
%     This means that the \( n \)-th digit of \( b \) is different from the \( n \)-th digit of \( b_n \). Since 
% \end{eg}

\begin{eg}
    Let's show that the set of sequences of binary digits (0s and 1s) is uncountable. Assume, for the sake of contradiction, that the set of binary sequences is countable. This means that we can list all the binary sequences in a sequence:
    \[
        s_1, s_2, s_3, \ldots
    \]
    where each \( s_i \) is a binary sequence. We can represent each binary sequence as follows:
    \[
        s_1 = b_{11}b_{12}b_{13}\ldots
    \]
    \[
        s_2 = b_{21}b_{22}b_{23}\ldots
    \]
    \[
        s_3 = b_{31}b_{32}b_{33}\ldots
    \]
    and so on. Now, we will construct a new binary sequence \( s \) that is not in the list. We define \( s \) by choosing its digits as follows:
    \[
        b_n = \begin{cases}
            0 & \text{if } b_{nn} = 1 \\
            1 & \text{if } b_{nn} = 0
        \end{cases}
    \]
    This means that the \( n \)-th digit of \( s \) is different from the \( n \)-th digit of \( s_n \). Therefore, \( s \) cannot be equal to any \( s_n \) in the list, since it differs from each \( s_n \) at least in the \( n \)-th digit. This contradicts our assumption that we have listed all binary sequences. Hence, we conclude that the set of binary sequences is uncountable.
\end{eg}

\begin{eg}
    Let's show that the positive set of rational numbers \( \mathbb{Q}^+ \) is countable. We can represent each positive rational number as a fraction \( \frac{p}{q} \), where \( p \) and \( q \) are positive integers. We can arrange these fractions in a two-dimensional grid, where the rows represent the numerator \( p \) and the columns represent the denominator \( q \):
    \[
        \begin{array}{c|ccccc}
            \\ & q=1 & q=2 & q=3 & q=4 & \ldots \\ \\
            \hline \\
            p=1 & \frac{1}{1} & \frac{1}{2} & \frac{1}{3} & \frac{1}{4} & \ldots \\ \\
            p=2 & \frac{2}{1} & \frac{2}{2} & \frac{2}{3} & \frac{2}{4} & \ldots \\ \\
            p=3 & \frac{3}{1} & \frac{3}{2} & \frac{3}{3} & \frac{3}{4} & \ldots \\ \\
            p=4 & \frac{4}{1} & \frac{4}{2} & \frac{4}{3} & \frac{4}{4} & \ldots \\ \\
            \vdots & \vdots & \vdots & \vdots & \vdots & \ddots \\
        \end{array}
    \]
    We can then list the positive rational numbers by traversing this grid in a diagonal manner, starting from the top-left corner:
    \[
        \frac{1}{1}, \frac{1}{2}, \frac{2}{1}, \frac{3}{1}, \frac{2}{2}, \frac{1}{3}, \frac{1}{4}, \frac{2}{3}, \frac{3}{2}, \frac{4}{1}, \ldots
    \]
    To avoid counting duplicates (fractions that are equivalent), we can only include fractions in their simplest form (i.e., where the numerator and denominator are coprime). \\
    By this method, we can list all positive rational numbers in a sequence indexed by the positive integers. Therefore, we conclude that the set of positive rational numbers \( \mathbb{Q}^+ \) is countable.
\end{eg}
Remark the following regarding Cantor's Diagonal Argument:
\begin{itemize}[itemsep=1pt,label=$\circ$]
    \item If Cantor's Diagonal Argument fails to prove that a set is uncountable, it does not imply that the set is countable.
\end{itemize}

% \begin{eg}
%     Let's prove that the set of natural numbers $\mathbb{N}$ is not uncountable by using Cantor's Diagonal Argument. \\
%     Assume, for the sake of contradiction, that the set of natural numbers \( \mathbb{N} \) is uncountable. This means that we can list all the natural numbers in a sequence:
%     \[
%         n_1, n_2, n_3, \ldots
%     \]
%     where each \( n_i \) is a natural number. We can represent each natural number in its decimal expansion:
%     \[
%         n_1 = d_{11}d_{12}d_{13}\ldots
%     \]
%     \[
%         n_2 = d_{21}d_{22}d_{23}\ldots
%     \]
%     \[
%         n_3 = d_{31}d_{32}d_{33}\ldots
%     \]
%     and so on. Now, we will attempt to construct a new natural number \( n \) that is not in the list. We define \( n \) by choosing its decimal digits as follows:
%     \[
%         d_n = \begin{cases}
%             (d_{nn} + 1) \mod 10 & \text{if } d_{nn} \neq 9 \\
%             0 & \text{if } d_{nn} = 9
%         \end{cases}
%     \]
%     This means that the \( n \)-th digit of \( n \) is different from the \( n \)-th digit of \( n_n \). Therefore, \( n \) cannot be equal to any \( n_n \) in the list, since it differs from each \( n_n \) at least in the \( n \)-th digit. However, this construction does not guarantee that \( n \) is a natural number, as it may lead to an infinite decimal expansion. This contradicts our assumption that we have listed all natural numbers. Hence, we conclude that the set of natural numbers \( \mathbb{N} \) is not uncountable.
% \end{eg}

\begin{eg}
    Let's prove that the set of positive rational numbers \( \mathbb{Q}^+ \) is not uncountable by using Cantor's Diagonal Argument. \\
    Let's try to construct a new positive rational number \( r \) that is not in the list. We define \( r \) by choosing its decimal digits as follows:
    \[
        d_n = \begin{cases}
            1 & \text{if } d_{nn} \neq 1 \\
            2 & \text{if } d_{nn} = 1
        \end{cases}
    \]
    This means that the \( n \)-th digit of \( r \) is different from the \( n \)-th digit of \( r_n \). Therefore, \( r \) cannot be equal to any \( r_n \) in the list, since it differs from each \( r_n \) at least in the \( n \)-th digit. However, this construction does not guarantee that \( r \) is a positive rational number, as it may lead to an infinite decimal expansion that does not terminate or repeat. For example, $\frac{3}{7}$ has a decimal expansion that can be found by long division:
    \[
        \frac{30}{7} = 4 \quad remainder \quad 2
    \]
    \[
        \frac{20}{7} = 2 \quad remainder \quad 6
    \]
    \[        \frac{60}{7} = 8 \quad remainder \quad 4
    \]
    \[        \frac{40}{7} = 5 \quad remainder \quad 5
    \]
    \[        \frac{50}{7} = 7 \quad remainder \quad 1
    \]
    \[        \frac{10}{7} = 1 \quad remainder \quad 3
    \]
    Thus since the reminder $3$ appears again, the decimal expansion is $0.\overline{428571}$ which is a repetition. In contrast, nothing guarantees that the constructed number $r$ will have such a repeating or terminating decimal expansion. This contradict our assumption that we are constructing a positive rational number. Hence, we conclude that the set of positive rational numbers \( \mathbb{Q}^+ \) is not uncountable.
\end{eg}
Remark that a similar argument can be made to prove that the set of natural numbers \( \mathbb{N} \) is not uncountable.

\begin{theorem}
    If $A$ is uncountable and $A \subseteq B$, then $B$ is uncountable.
\end{theorem}
\begin{proof}
    Assume, for the sake of contradiction, that \( B \) is countable. Since \( A \subseteq B \), every element of \( A \) is also an element of \( B \). Therefore, we can define an injection \( f: A \to B \) that maps each element of \( A \) to itself in \( B \). \\
    Since \( B \) is countable, there exists a bijection \( g: \mathbb{N} \to B \). We can then compose the functions \( f \) and \( g^{-1} \) to obtain a function \( h: A \to \mathbb{N} \) defined by:
    \[
        h(a) = g^{-1}(f(a))
    \]
    This function \( h \) is an injection from \( A \) to \( \mathbb{N} \), which implies that \( A \) is countable. However, this contradicts our assumption that \( A \) is uncountable. Hence, we conclude that if \( A \) is uncountable and \( A \subseteq B \), then \( B \) must be uncountable.
\end{proof}
Remark that this theorem proves that the set of real numbers \( \mathbb{R} \) is uncountable, since $[0,1] \subset \mathbb{R}$ and $[0,1]$ is uncountable by Cantor's Diagonal Argument.

\section{Exercices}
This section gathers a selection of exercises related to Chapter \thechapter, taken from weekly assignments, past exams, textbooks, and other sources. The origin of each exercise will be indicated at its beginning.
% TODO: add in the exercise the quizz of w11c1 @ 40:40

\begin{exercise}[Week 11, Class 1, Bit Strings]
    Find a recurrence relation for the number of bit strings of length $n$ that contain a pair of consecutive zeros.
    \Answer
    Let's define \( P_n \) as the number of bit strings of length \( n \) that contain at least one pair of consecutive zeros. To form such a bit string, we can consider the following cases based on the last bit of the string:
    \begin{itemize}[itemsep=1pt,label=$\circ$]
        \item If the last bit is \( 1 \), then the first \( n-1 \) bits can be any bit string of length \( n-1 \) that contains a pair of consecutive zeros. The number of such strings is \( |P_{n-1}| \).
        \item If the last bit is \( 0 \), we need to consider the second last bit:
        \begin{itemize}[itemsep=1pt,label=$\circ$]
            \item If the second last bit is also \( 0 \), then we have a pair of consecutive zeros at the end of the string. The first \( n-2 \) bits can be any bit string of length \( n-2 \). The number of such strings is \( 2^{n-2} \).
            \item If the second last bit is \( 1 \), then the first \( n-2 \) bits must contain a pair of consecutive zeros. The number of such strings is \( |P_{n-2}| \).
        \end{itemize}
    \end{itemize}
    Therefore, we can express \( |P_n| \) in terms of \( |P_{n-1}| \) and \( |P_{n-2}| \) as follows:
    \[
        |P_n| = |P_{n-1}| + 2^{n-2} + |P_{n-2}|
    \]
    This gives us the recurrence relation:
    \[
        |P_n| = |P_{n-1}| + |P_{n-2}| + 2^{n-2} \quad \text{for } n \geq 2
    \]
    with the base cases:
    \[
        |P_1| = 0 \quad \text{(no bit strings of length 1 can contain consecutive zeros)}
    \]
    \[
        |P_2| = 1 \quad \text{(the only bit string of length 2 that contains consecutive zeros is "00")}
    \]
\end{exercise}