\chapter{Representations of Numbers}

\section{Integers}

\begin{definition}[Decimal Notation]
    The decimal notation of an integer \( z \in \mathbb{Z} \) is a sequence of digits \( d_n d_{n-1} \ldots d_1 d_0 \) such that
    \[
        z = \sum_{i=0}^{n} d_i \cdot 10^i
    \]
    where each digit \( d_i \) is in the set \( \{0, 1, 2, \ldots, 9\} \).
\end{definition}

\begin{eg}
    The decimal notation of the integer \( 345 \) is given by the digits \( 3, 4, 5 \):
    \[
        345 = 3 \cdot 10^2 + 4 \cdot 10^1 + 5 \cdot 10^0
    \]
\end{eg}
Numbers can also be represented in other bases, the most important for computing are: binary (base 2), octal (base 8), and hexadecimal (base 16), but are not limited to these (ancient Mayans used base 20, ancient Babylonians used base 60).

\begin{definition}[Base $b$ Notation]
    The base \( b \) notation of an integer greater than $1$ is a sequence of digits \( d_n d_{n-1} \ldots d_1 d_0 \) such that \( 0 \leq d_i < b \), $n$ a non negative integer and \( d_n \neq 0 \).
    \[
        z = \sum_{i=0}^{n} d_i \cdot b^i
    \]
    where each digit \( d_i \) is in the set \( \{0, 1, 2, \ldots, b-1\} \).
\end{definition}
Remark that the base \( b \) notation is unique for each integer.
\begin{proof}
    Let \( z \in \mathbb{Z} \) be an integer and suppose there exist two different base \( b \) notations for \( z \):
    \[
        z = \sum_{i=0}^{n} d_i \cdot b^i = \sum_{i=0}^{m} e_i \cdot b^i
    \]
    where \( d_i, e_i \in \{0, 1, 2, \ldots, b-1\} \), \( d_n \neq 0 \), and \( e_m \neq 0 \). Without loss of generality, assume \( n \geq m \). We can rewrite the equation as:
    \[
        \sum_{i=0}^{n} d_i \cdot b^i - \sum_{i=0}^{m} e_i \cdot b^i = 0
    \]
    This implies:
    \[
        \sum_{i=0}^{m} (d_i - e_i) b^i + \sum_{i=m+1}^{n} d_i b^i = 0
    \]
    Since \( d_n \neq 0 \), the term \( d_n b^n \) is non-zero and dominates the sum for sufficiently large \( n \). Therefore, the only way for the entire sum to equal zero is if all coefficients are zero:
    \[
        d_i - e_i = 0 \quad \text{for } i = 0, 1, \ldots, m
    \]
    and
    \[
        d_i = 0 \quad \text{for } i = m+1, m+2, \ldots, n
    \]
    This leads to a contradiction since \( d_n \neq 0 \). Hence, our assumption that there exist two different base \( b \) notations for \( z \) is false. Therefore, the base \( b \) notation of an integer is unique.
\end{proof}

\begin{eg}
    Let's represent the integer $17$ in common bases:
    \begin{itemize}[itemsep=1pt,label=$\circ$]
        \item Binary (base 2): \( 17 = 1 \cdot 2^4 + 0 \cdot 2^3 + 0 \cdot 2^2 + 1 \cdot 2^1 + 1 \cdot 2^0 = 10001_2 \)
        \item Octal (base 8): \( 17 = 2 \cdot 8^1 + 1 \cdot 8^0 = 21_8 \)
        \item Hexadecimal (base 16): \( 17 = 1 \cdot 16^1 + 1 \cdot 16^0 = 11_{16} \)
    \end{itemize}
\end{eg}
Note that in hexadecimal notation, digits above 9 are represented using letters A to F (A=10, B=11, C=12, D=13, E=14, F=15) so all the numbers from $1$ to $15$ have a single digit representation.

\subsection{Contruction of Base $b$ Notation}
There are several algorithms to construct the base \( b \) notation of an integer \( n \geq 0 \).

\begin{eg}
    To construct a number \( n \) in base \( b \), we can use the repeated subtraction method:
    \begin{itemize}[itemsep=1pt,label=$\circ$]
        \item Find the largest power of \( b \), say \( b^k \), such that \( b^k \leq n \).
        \item Determine the coefficient \( d_k \) by calculating \( d_k = \lfloor n / b^k \rfloor \).
        \item Update \( n \) to \( n - d_k \cdot b^k \).
        \item Repeat the process for \( b^{k-1}, b^{k-2}, \ldots, b^0 \) until \( n \) becomes zero.
    \end{itemize}
    For example, to convert \( 45 \) to base \( 3 \):
    \[
        3^3 = 27 \leq 45 < 81 = 3^4 \quad \Rightarrow \quad d_3 = \lfloor 45 / 27 \rfloor = 1
    \]
    \[
        n = 45 - 1 \cdot 27 = 18
    \]
    \[
        3^2 = 9 \leq 18 < 27 = 3^3 \quad \Rightarrow \quad d_2 = \lfloor 18 / 9 \rfloor = 2
    \]
    \[
        n = 18 - 2 \cdot 9 = 0
    \]
    \[
        3^1 = 3 \leq 0 < 9 = 3^2 \quad \Rightarrow \quad d_1 = \lfloor 0 / 3 \rfloor = 0
    \]
    \[
        3^0 = 1 \leq 0 < 3 = 3^1 \quad \Rightarrow \quad d_0 = \lfloor 0 / 1 \rfloor = 0
    \]
    Thus, reading the coefficients from \( d_3 \) to \( d_0 \), we get \( 45 = 1200_3 \).
\end{eg}

\begin{theorem}[Division Remainder Method]
    If $a$ is an integer and $d$ a positive integer, then there are unique integers $q$ and $r$ such that $0 \leq r < d$ and:
    \[
        a = d \cdot q + r
    \]
    The integer $q$ is called the quotient, $d$ is called the divisor, $a$ is called the dividend and $r$ the remainder of the division of $a$ by $d$.
\end{theorem}

\begin{eg}
    For example, dividing \( 17 \) by \( 5 \):
    \[
        17 = 5 \cdot 3 + 2
    \]
    Here, the quotient \( q = 3 \) and the remainder \( r = 2 \).
\end{eg}

\begin{eg}
    Another example, dividing $-11$ by $3$:
    \[
        -11 = 3 \cdot (-4) + 1
    \]
    Here, the quotient \( q = -4 \) and the remainder \( r = 1 \).
\end{eg}

\begin{eg}
    To construct a number $n$ in base $b$, we can use the division-remainder method:
    \begin{itemize}[itemsep=1pt,label=$\circ$]
        \item Divide \( n \) by \( b \) to get a quotient \( q_0 \) and a remainder \( d_0 \) (the least significant digit).
        \item Set \( n = q_0 \) and repeat the division until the quotient is zero.
        \item The base \( b \) representation is obtained by reading the remainders in reverse order.
    \end{itemize}
    For example, to convert \( 45 \) to base \( 2 \):
    \[
        45 \div 2 = 22 \text{ remainder } 1 \quad (d_0 = 1)
    \]
    \[
        22 \div 2 = 11 \text{ remainder } 0 \quad (d_1 = 0)
    \]
    \[
        11 \div 2 = 5 \text{ remainder } 1 \quad (d_2 = 1)
    \]
    \[
        5 \div 2 = 2 \text{ remainder } 1 \quad (d_3 = 1)
    \]
    \[
        2 \div 2 = 1 \text{ remainder } 0 \quad (d_4 = 0)
    \]
    \[
        1 \div 2 = 0 \text{ remainder } 1 \quad (d_5 = 1)
    \]
    Reading the remainders in reverse order, we get \( 45 = 101101_2 \).
\end{eg}

\subsection{Operations on Base $b$ Notation}

\begin{definition}[Addition in Base $b$]
    To add two numbers in base \( b \), align the digits and add them column by column from right to left, carrying over any value that exceeds \( b-1 \) to the next column.
\end{definition}

\begin{eg}
    For example, adding \( 345_8 \) and \( 267_8 \) in base \( 8 \):
    \[
        \begin{array}{c@{}c@{}c@{}c}
          & 3 & 4 & 5_8 \\
        + & 2 & 6 & 7_8 \\
        \hline
          & 6 & 3 & 4_8 \\
        \end{array}
    \]
    Here, \( 5 + 7 = 12_{10} = 14_8 \) (write down \( 4 \), carry over \( 1 \)), \( 4 + 6 + 1 = 11_{10} = 13_8 \) (write down \( 3 \), carry over \( 1 \)), and \( 3 + 2 + 1 = 6_{10} = 6_8 \).
\end{eg}

\begin{definition}[Multiplication in Base $b$]
    To multiply two numbers in base \( b \), use the standard multiplication algorithm, multiplying each digit of the second number by the entire first number, shifting left for each digit position, and then summing all the partial products. The pseudo-code is as follows:
\end{definition}
Remark that mutiplying a number by the base \( b \) is equivalent to shifting the number one position to the left (adding a zero at the end).

\begin{eg}
    For example, multiplying \( 23_5 \) and \( 14_5 \) in base \( 5 \):
    \[
        \begin{array}{c@{}c@{}c@{}c}
          &   & 2 & 3_5 \\
        \times &   & 1 & 4_5 \\
        \hline
          & 2 & 0 & 2_5 \\ 
        + & 2 & 3 & 0_5 \\
        \hline
          & 4 & 3 & 2_5 \\
        \end{array}
    \]
    Here, \(3\times 4 = 12_{10} = 22_5\) (write down \(2\), carry \(2\)), then \(2\times 4 + 2 = 10_{10} = 20_5\) (write down \(0\), carry \(2\)), so the partial product is \(202_5\); the other partial product is \(23_5\) shifted to \(230_5\); adding these gives \(202_5 + 230_5 = 432_5\).
\end{eg}

\section{Counting}
\begin{definition}[Counting]
    Counting is ubiquitous in mathematics (e.g., combinatorics) and computer science. It involves determining the number of elements in a set or the number of ways to arrange or select items. Various techniques such as permutations, combinations, and the principle of inclusion-exclusion are used in counting problems.
\end{definition}
Let's introduce some notations that will be useful in counting:
\begin{itemize}[itemsep=1pt,label=$\circ$]
    \item Sequences are ordered.
    \item $X$ will be refered to as the alphabet.
    \item Often $s(1),s(2), \ldots, s(n)$ will be denoted as $s = s_1,s_2, \ldots, s_n$.
    \item Sequences will be used interchangeably with strings/words.
\end{itemize}

\begin{eg}
    Given the set of vowels \( X = \{a, e, i, o, u\} \), the number of possible 4-letter sequences (words) that can be formed is:
    \[
        |X|^4 = 5^4 = 625
    \]
    since each position in the sequence can be filled by any of the 5 vowels.
\end{eg}

\begin{theorem}[Product Rule]
    If a task can be broken down into \( k \) sequential steps, where the first step can be performed in \( n_1 \) ways, the second step in \( n_2 \) ways, and so on up to the \( k \)-th step which can be performed in \( n_k \) ways, then the total number of ways to perform the entire task is given by:
    \[
        N = n_1 \times n_2 \times \ldots \times n_k = \prod_{i=1}^{k} n_i
    \]
\end{theorem}
Remark that the set from which we choose the elements can change at each step but the set must not depend on the previous choices.

\begin{eg}
    If a license plate consists of 2 letters followed by 3 digits, and there are 26 letters in the alphabet and 10 digits (0-9), the total number of different license plates that can be formed is:
    \[
        N = 26^2 \times 10^3 = 676000
    \]
\end{eg}

\begin{theorem}
    The number of different subset of a set $S$ with \( n \) elements is \( 2^n \).
\end{theorem}
\begin{proof}
    When the element of $S$ are listed in an arbitrary order, there is a one-to-one correspondence between the subsets of $S$ and the binary sequences of length $n$: the $i$-th element of $S$ is in the subset if and only if the $i$-th digit of the sequence is $1$. Since there are \( 2^n \) binary sequences of length \( n \), there are \( 2^n \) subsets of \( S \).
\end{proof}
Remark that if the cardinality of a set is known and a bijection can be established between this set and another set, then the cardinality of the second set is also known and it is the same as the first set.

\subsection{Counting Functions}
\begin{definition}[Counting Functions]
    Given two finite sets \( A \) and \( B \) with cardinalities \( |A| = m \) and \( |B| = n \), the number of functions from \( A \) to \( B \) is given by:
    \[
        n^m
    \]
    since each element in \( A \) can be mapped to any of the \( n \) elements in \( B \).
\end{definition}
Remark that if instead of functions, only injective (one-to-one) functions are considered, the counting changes to:
\[
    \frac{n!}{(n-m)!}
\]
provided that \( n \geq m \).

\begin{eg}
    Rob has 4 blue socks, 7 red socks, 5 white socks and 3 black socks. He likes to wear either a red sock on his left foot with a blue sock on his right foot or a white sock on his left foot with a black sock on his right foot. How many different ways can Rob wear his socks?
    \[
        \text{Total ways} = (7 \times 4) + (5 \times 3) = 28 + 15 = 43
    \]
\end{eg}

\begin{theorem}[Sum Rule]
    If a task can be performed in \( n_1 \) ways or \( n_2 \) ways (but not both), then the total number of ways to perform the task is:
    \[
        N = n_1 + n_2
    \]
\end{theorem}

\begin{eg}
    Each user on a computer system has a password, which is $6$ to $8$ characters long, where each character is an uppercase letter (A-Z) or a digit (0-9). How many different passwords are possible?
    \[
        \text{Total passwords} = 36^6 + 36^7 + 36^8 = 2,901,650,853,888
    \]
\end{eg}

\begin{eg}
    Same example as before but now the password must contain at least one digit. How many different passwords are possible?
    \[
        \text{Total passwords} = (36^6 - 26^6) + (36^7 - 26^7) + (36^8 - 26^8) = 2,743,303,001,088
    \]
\end{eg}

\begin{theorem}[Substraction Rule]
    If a task can be performed in \( n \) ways, and \( m \) of these ways are not allowed, then the total number of ways to perform the task is:
    \[
        N = n - m
    \]
\end{theorem}

\begin{eg}
    How many bit strings of length $8$ either start with a 1 bit or end with the two bits 00?
    \[
        \text{Total bit strings} = 2^7 + 2^6 - 2^5 = 160
    \]
    Note that the bit strings that both start with a 1 and end with 00 have been subtracted once to avoid double counting. \\
    We could also view this example has two sets: $A$ the set of bit strings of length $8$ that start with a 1 and $B$ the set of bit strings of length $8$ that end with 00. The cardinality of the union of these two sets is given by:
    \[
        |A \cup B| = |A| + |B| - |A \cap B| = 2^7 + 2^6 - 2^5 = 160
    \]
\end{eg}

\begin{eg}
    How many integers from $1$ to $100$ are not divisible by $2$ or $5$?
    \[
        \text{Total integers} = 100 - (50 + 20 - 10) = 40
    \]
    Here, $50$ integers are divisible by $2$, $20$ integers are divisible by $5$ and $10$ integers are divisible by both $2$ and $5$.
\end{eg}

\section{Permutations and Combinations}
Let's assume that $S = \{1,2,3,4\}$. The number of ways to build a strings of length $2$ from the elements of $S$:
\vskip0.3cm
\begin{center}
    \begin{tabular}{p{0.30\textwidth} | p{0.30\textwidth} | p{0.30\textwidth}}
        & & \\
        & {\centering \textbf{Permutation} \par} & {\centering \textbf{Combination} \par} \\ 
        & & \\ \hline 
        & & \\ 
        {\centering \textbf{without repetition} \par} & { \centering
            $\begin{array}{cccc}
                & 12 & 13 & 14 \\
                21 & & 23 & 24 \\
                31 & 32 & & 34 \\
                41 & 42 & 43 &
            \end{array}$ \par
        } & {\centering
        $\begin{array}{cccc}
            & 12 & 13 & 14 \\
            & & 23 & 24 \\
            & & & 34 \\
            & & &
        \end{array}$\par} \\ 
        & & \\ \hline
        & & \\
        {\centering \textbf{with repetition} \par} & { \centering
            $\begin{array}{cccc}
                11 & 12 & 13 & 14 \\
                21 & 22 & 23 & 24 \\
                31 & 32 & 33 & 34 \\
                41 & 42 & 43 & 44
            \end{array}$ \par
        } & {\centering 
            $\begin{array}{cccc}
                11 & 12 & 13 & 14 \\
                 & 22 & 23 & 24 \\
                 &    & 33 & 34 \\
                 &    &    & 44
            \end{array}$
        \par} \\ 
        & & \\
    \end{tabular}
\end{center}
\subsection{Permutations}
\begin{definition}[Permutations]
    A permutation of a set of \( n \) distinct elements is an arrangement of all the elements in a specific order. The number of different permutations of \( n \) distinct elements is given by:
    \[
        n! = n \times (n-1) \times (n-2) \times \ldots \times 2 \times 1
    \]
\end{definition}
Remark that if only \( r \) elements are to be arranged from a set of \( n \) distinct elements, the number of different permutations is given by:
\[
    P(n, r) = \frac{n!}{(n-r)!}
\]

\begin{eg}
    Let's take two similar examples:
    \begin{itemize}[itemsep=1pt,label=$\circ$]
        \item A class of $100$ sutdents is electing a president, a vice-president and a secretary. How many different ways can these positions be filled?
        \[
            P(100, 3) = \frac{100!}{(100-3)!} = 970200
        \]
        \item A class of $100$ is electing $3$ representatives. How many different ways can these positions be filled? Let's denote the representatives as $R_1$, $R_2$ and $R_3$ to distinguish them, then the number of different ways to fill these positions is:
        \[
            \begin{array}{ccc}
                R_1 & R_2 & R_3 \\
                R_1 & R_3 & R_2 \\
                R_2 & R_1 & R_3 \\
                 & \vdots & \\
                R_3 & R_2 & R_1 \\
            \end{array}
        \]
        There are \( 3! = 6 \) ways to arrange the representatives for each selection of \( 3 \) students. Therefore, the total number of different ways to select the representatives is:
        \[
            C(n, k) = \frac{P(n, k)}{k!} = \begin{pmatrix}
                n \\ k
            \end{pmatrix}= \frac{P(100, 3)}{3!} = 161700
        \]
    \end{itemize}
\end{eg}

\subsection{Combinations}
\begin{definition}[Binomial Coefficient]
    The binomial coefficient is defined as:
    \[
        \begin{pmatrix}
            n \\ k
        \end{pmatrix} = \frac{n!}{k!(n-k)!}
    \]
\end{definition}
Remark that this is the same coefficient used to expand the binomial expression \( (x + y)^n \) using the Binomial Theorem.

\begin{eg}
    Let's expend the binomial expression \( (x + y)^4 \) using the Binomial Theorem:
    \[
        (x + y)^4 = \sum_{k=0}^{4} \begin{pmatrix}
            4 \\ k
        \end{pmatrix} x^{4-k} y^k = \begin{pmatrix}
            4 \\ 0
        \end{pmatrix} x^4 + \begin{pmatrix}
            4 \\ 1
        \end{pmatrix} x^3 y + \begin{pmatrix}
            4 \\ 2
        \end{pmatrix} x^2 y^2 + \begin{pmatrix}
            4 \\ 3
        \end{pmatrix} x y^3 + \begin{pmatrix}
            4 \\ 4
        \end{pmatrix} y^4
    \]
    Calculating the binomial coefficients, we get:
    \[
        (x + y)^4 = 1 \cdot x^4 + 4 \cdot x^3 y + 6 \cdot x^2 y^2 + 4 \cdot x y^3 + 1 \cdot y^4
    \]
\end{eg}

\begin{theorem}[Combinations]
    The number of ways to choose \( k \) elements from a set of \( n \) distinct elements, where the order of selection does not matter, is given by the binomial coefficient:
    \[
        C(n, k) = \begin{pmatrix}
            n \\ k
        \end{pmatrix} = \frac{n!}{k!(n-k)!}
    \]
\end{theorem}
Remark that \( C(n, k) = C(n, n-k) \) since choosing \( k \) elements to include is equivalent to choosing \( n-k \) elements to exclude.

\begin{eg}
    How many poker hands of $5$ cards can be dealt from a standard deck of $52$ cards?
    \[
        C(52, 5) = \begin{pmatrix}
            52 \\ 5
        \end{pmatrix} = \frac{52!}{5!(52-5)!} = 2,598,960
    \]
\end{eg}

\begin{eg}
    Based on the previous example, how many poker hands with a full house (can be three of a kind and a pair) can be dealt from a standard deck of $52$ cards?
    \[
        \text{Total full house hands} = 13 \times C(4, 3) \times 12 \times C(4, 2) = 3,744
    \]
    Here, \( 13 \) is the number of ranks for the three of a kind, \( C(4, 3) \) is the number of ways to choose \( 3 \) suits from \( 4 \), \( 12 \) is the number of remaining ranks for the pair, and \( C(4, 2) \) is the number of ways to choose \( 2 \) suits from \( 4 \).
\end{eg}

% TODO: add fruits example plus demo that was cutted from the video (w10c1 end and review w10c2 begining)
\begin{eg}
    A fruit shop offers \( 5 \) types of fruits: apples, bananas, cherries, dates, and elderberries. A customer wants to buy a fruit basket containing \( 8 \) pieces of fruit, where the order of selection does not matter and repetitions are allowed. How many different fruit baskets can the customer create? \\
    This problem can be solved using the "Stars and Bars" method. We represent the \( r=8 \) fruits as stars (\(*\)) and use \( n-1 \) bars (\(|\)) to separate the \( n=5 \) different types of fruit. We need \( 5-1 = 4 \) bars to create \( 5 \) compartments (bins). \\
    For example, if we select 2 apples, 3 bananas, 1 cherry, 0 dates, and 2 elderberries, this corresponds to the string:
    \[
        **|***|*||**
    \]
    Any arrangement of these \( 8 \) stars and \( 4 \) bars represents a unique fruit basket. The total length of the string is \( 8 + 4 = 12 \). The problem reduces to choosing which \( 8 \) of the \( 12 \) positions contain stars (or equivalently, which \( 4 \) positions contain bars). \\
    The formula for the number of combinations with repetition is thus given by:
    \[
        C(n + r - 1, r) = \begin{pmatrix}
            n + r - 1 \\ r
        \end{pmatrix}
    \]
    In this case, \( n = 5 \) and \( r = 8 \):
    \[
        \text{Total fruit baskets} = \begin{pmatrix}
            5 + 8 - 1 \\ 8
        \end{pmatrix} = \begin{pmatrix}
            12 \\ 8
        \end{pmatrix} = \frac{12!}{8!4!} = 495
    \]
    Therefore, the customer can create \( 495 \) different fruit baskets.
\end{eg}

% \subsection{Permutations vs Combinations}
% Let's assume that $S = \{1,2,3,4\}$. The number of ways to build a strings of length $2$ from the elements of $S$:
% \vskip0.3cm
% \begin{center}
%     \begin{tabular}{p{0.30\textwidth} | p{0.30\textwidth} | p{0.30\textwidth}}
%         & & \\
%         & {\centering \textbf{Permutation} \par} & {\centering \textbf{Combination} \par} \\ 
%         & & \\ \hline 
%         & & \\ 
%         {\centering \textbf{without repetition} \par} & { \centering
%             $\begin{array}{cccc}
%                 & 12 & 13 & 14 \\
%                 21 & & 23 & 24 \\
%                 31 & 32 & & 34 \\
%                 41 & 42 & 43 &
%             \end{array}$ \par
%         } & {\centering
%         $\begin{array}{cccc}
%             & 12 & 13 & 14 \\
%             & & 23 & 24 \\
%             & & & 34 \\
%             & & &
%         \end{array}$\par} \\ 
%         & & \\ \hline
%         & & \\
%         {\centering \textbf{with repetition} \par} & { \centering
%             $\begin{array}{cccc}
%                 11 & 12 & 13 & 14 \\
%                 21 & 22 & 23 & 24 \\
%                 31 & 32 & 33 & 34 \\
%                 41 & 42 & 43 & 44
%             \end{array}$ \par
%         } & {\centering 
%             $\begin{array}{cccc}
%                 11 & 12 & 13 & 14 \\
%                  & 22 & 23 & 24 \\
%                  &    & 33 & 34 \\
%                  &    &    & 44
%             \end{array}$
%         \par} \\ 
%         & & \\
%     \end{tabular}
% \end{center}

% \begin{eg}
%     How many different 4-digit even numbers can be formed? We assume that the digits can be repeated.
%     \[
%         \text{Total even numbers} = 9 \times 10 \times 10 \times 5 = 4500
%     \]
%     Here, the first digit can be any digit from \( 1 \) to \( 9 \) (9 options because it cannot be 0 otherwise the number would not be a 4-digit number), the second and third digits can be any digit from \( 0 \) to \( 9 \) (10 options each), and the last digit must be an even digit (0, 2, 4, 6, or 8) (5 options).
% \end{eg}

\subsection{Permutations with Indistinguishable Objects}
\begin{theorem}[Permutations with Indistinguishable Objects]
    The number of distinct permutations of \( n \) objects, where there are \( n_1 \) indistinguishable objects of type 1, \( n_2 \) indistinguishable objects of type 2, ..., and \( n_k \) indistinguishable objects of type \( k \) (with \( n_1 + n_2 + ... + n_k = n \)), is given by:
    \[
        \frac{n!}{n_1! \times n_2! \times ... \times n_k!}
    \]
\end{theorem}
\begin{proof}
    Consider a set of \( n \) objects, where \( n_1 \) are of type 1, \( n_2 \) are of type 2, ..., and \( n_k \) are of type \( k \). The total number of permutations of these \( n \) objects, if they were all distinguishable, would be \( n! \). However, since the objects of the same type are indistinguishable, we need to account for the overcounting that occurs when we permute the indistinguishable objects among themselves. \\
    For each type \( i \), there are \( n_i! \) ways to arrange the \( n_i \) indistinguishable objects of that type. Therefore, to find the number of distinct permutations, we divide the total permutations \( n! \) by the product of the factorials of the counts of each type:
    \[
        \text{Distinct permutations} = \frac{n!}{n_1! \times n_2! \times ... \times n_k!}
    \]
    This formula gives us the correct count of distinct arrangements by eliminating the overcounting due to indistinguishable objects.
\end{proof}
Remark that this theorem could also be proved using the product rule by considering the selection of positions for each type of indistinguishable object sequentially ($C(n,n_1) \cdot C(n-n_1,n_2) \cdot \ldots \cdot C(n-n_1-...-n_{k-1}, n_k)$). which would lead to the same result.

\begin{eg}
    How many different strings can be made by reordering the letters of the word "SUCCESS"?
    \[
        \text{Total strings} = \frac{7!}{3!2!1!1!} = 420
    \]
    Here, the total number of letters is \( 7 \), with the letter 'S' appearing \( 3 \) times, 'C' appearing \( 2 \) times, and 'U' and 'E' appearing \( 1 \) time each. The formula accounts for the indistinguishable letters by dividing by the factorial of their counts.
\end{eg}

\subsection{Repetition vs. Indistinguishable Objects}
\begin{eg}
    Let's suppose we have the following alphabet $S = \{s,u,c\}$ and we build strings of length $4$ from the elements of $S$. \\
    \textbf{If repetition is allowed}, the total number of strings is:
    \[
        |S|^4 = 3^4 = 81
    \]
    \textbf{If repetition are not allowed}, then the number of strings with the alphabet $S' = \{s,s,u,c\}$ is:
    \[
        \text{Total strings} = \frac{4!}{2!1!1!} = 12
    \]
    Here, the total number of letters is \( 4 \), with the letter 's' appearing \( 2 \) times, and 'u' and 'c' appearing \( 1 \) time each. The formula accounts for the indistinguishable letters by dividing by the factorial of their counts.
\end{eg}

\section{Pigeonhole Principle}
\begin{theorem}[Pigeonhole Principle]
    If \( n \) items are put into \( m \) containers, with \( n > m \), then at least one container must contain more than one item.
\end{theorem}
\begin{proof}
    Assume, for the sake of contradiction, that no container contains more than one item. This means that each container can hold at most one item. Therefore, the maximum number of items that can be placed in \( m \) containers is \( m \). However, since \( n > m \), this leads to a contradiction because we have more items than the maximum capacity of the containers. Hence, our assumption is false, and at least one container must contain more than one item.
\end{proof}

\begin{theorem}[Generalized Pigeonhole Principle]
    If \( n \) items are put into \( m \) containers, then at least one container must contain at least \( \lceil n/m \rceil \) items.
\end{theorem}
\begin{proof}
    Assume, for the sake of contradiction, that no container contains \( \lceil n/m \rceil \) or more items. This means that each container can hold at most \( \lceil n/m \rceil - 1 \) items. Therefore, the maximum number of items that can be placed in \( m \) containers is:
    \[
        m \times (\lceil n/m \rceil - 1)
    \]
    Since \( \lceil n/m \rceil - 1 < n/m \), we have:
    \[
        m \times (\lceil n/m \rceil - 1) < m \times (n/m) = n
    \]
    This leads to a contradiction because we have more items than the maximum capacity of the containers. Hence, our assumption is false, and at least one container must contain at least \( \lceil n/m \rceil \) items.
\end{proof}

\begin{eg}
    Let's show that among $100$ students, at least one month has at least $9$ students born in that month. \\
    There are $12$ months in a year, so we have $n = 100$ students and $m = 12$ months. Using the generalized pigeonhole principle:
    \[
        \text{At least one month has } \lceil 100/12 \rceil = 9 \text{ students.}
    \]
\end{eg}

\subsection{Minimum Number of Items to Guarantee a Certain Outcome}
\begin{definition}[Minimum Number of Items to Guarantee a Certain Outcome]
    To guarantee that at least \( k \) items are in the same container when \( n \) items are distributed among \( m \) containers, we need at least:
    \[
        n = m \times (k - 1) + 1
    \]
    items.
\end{definition}

\begin{eg}
    What is the minimum number of students required in a physics class to guarantee that at least $6$ students receive the same grade, assuming the possible grades are A, B, C, D, and F? \\
    There are $5$ possible grades, so we have $m = 5$ grades. To ensure that at least one grade has at least $6$ students, we can use the generalized pigeonhole principle:
    \[
        n = m \times (k - 1) + 1 = 5 \times (6 - 1) + 1 = 26
    \]
    Therefore, a minimum of $26$ students is required to guarantee that at least $6$ students receive the same grade.
\end{eg}

\begin{eg}
    How many numbers must be selected from the set $\{1,2,3,4,5,6\}$ to guarantee that at least one pair of these numbers add up to $7$? \\
    The pairs that add up to $7$ are: $(1,6)$, $(2,5)$, and $(3,4)$. Thus, we have $m = 3$ pairs. To ensure that at least one pair is selected, we can use the generalized pigeonhole principle:
    \[
        n = m \times (k - 1) + 1 = 3 \times (1 - 1) + 1 = 4
    \]
    Therefore, a minimum of $4$ numbers must be selected to guarantee that at least one pair adds up to $7$.
\end{eg}

\begin{eg}
    How many cards must be selected from a standard deck of $52$ cards to guarantee that at least $3$ cards of the same suit are selected? \\
    There are $4$ suits in a standard deck of cards (hearts, diamonds, clubs, spades), so we have $m = 4$ suits. To ensure that at least one suit has at least $3$ cards selected, we can use the generalized pigeonhole principle:
    \[
        n = m \times (k - 1) + 1 = 4 \times (3 - 1) + 1 = 9
    \]
    Therefore, a minimum of $9$ cards must be selected to guarantee that at least $3$ cards of the same suit are selected.
\end{eg}

\begin{eg}
    How many cards must be selected from a standard deck of $52$ cards to guarantee that at least $3$ hearts are chosen? \\
    There are $13$ hearts in a standard deck of cards, so to ensure that at least $3$ hearts are selected, we can consider the worst-case scenario where we select all non-heart cards first. There are $52 - 13 = 39$ non-heart cards. To guarantee that at least $3$ hearts are selected, we need to select:
    \[
        n = 39 + 3 = 42
    \]
    Therefore, a minimum of $42$ cards must be selected to guarantee that at least $3$ hearts are chosen.
\end{eg}

% TODO: maybe move this section to the example of the bijection between the number of subsets and the number of binary sequences of length n
\section{Combinatorial Proofs}
\begin{definition}[Combinatorial Proofs]
    A combinatorial proof is a method of proving mathematical identities by counting the same set of objects in two different ways. This approach often provides a more intuitive understanding of the identity being proved. There are two main steps in a combinatorial proof:
    \begin{itemize}[itemsep=1pt,label=$\circ$]
        \item Showing that there is a bijection between the set being counted by the two side of the identity.
        \item Prove that both sides of the identity count the same objects but in different ways.
    \end{itemize}
\end{definition}

\subsection{Bijection Principle}
\begin{theorem}[Bijection Principle]
    If there exists a bijection between two finite sets \( A \) and \( B \), then the cardinalities of the sets are equal, i.e., \( |A| = |B| \).
\end{theorem}
\begin{proof}
    A bijection is a one-to-one correspondence between the elements of two sets. This means that for every element in set \( A \), there is a unique element in set \( B \), and vice versa. Since each element in \( A \) can be paired with exactly one element in \( B \), the number of elements in both sets must be the same. Therefore, if there exists a bijection between \( A \) and \( B \), it follows that \( |A| = |B| \).
\end{proof}

\begin{eg}
    Prove that \( C(n, k) = C(n, n-k) \) using a combinatorial proof. \\
    Consider a set \( S \) with \( n \) elements. The left side of the identity, \( C(n, k) \), counts the number of ways to choose \( k \) elements from the set \( S \). The right side of the identity, \( C(n, n-k) \), counts the number of ways to choose \( n-k \) elements from the same set \( S \). \\
    There is a bijection between the two sets of choices: for every subset of \( k \) elements chosen from \( S \), there is a corresponding subset of \( n-k \) elements that are not chosen. This means that choosing \( k \) elements is equivalent to choosing which \( n-k \) elements to leave out. Therefore, both sides of the identity count the same number of subsets, leading to the conclusion that:
    \[
        C(n, k) = C(n, n-k)
    \]
\end{eg}

\subsection{Double Counting Principle}
\begin{theorem}[Double Counting Principle]
    If a set \( S \) can be counted in two different ways, then the two counts must be equal.
\end{theorem}
\begin{proof}
    Let \( S \) be a set that can be counted in two different ways, resulting in counts \( A \) and \( B \). Since both counts represent the same set \( S \), they must be equal. Therefore, we have:
    \[
        A = B
    \]
\end{proof}

\begin{eg}
    Let's prove the identity \( \sum_{k=0}^{n} C(n, k) = 2^n \) using a combinatorial proof. \\
    The left side of the identity, \( \sum_{k=0}^{n} C(n, k) \), counts the total number of subsets of a set with \( n \) elements. This is because \( C(n, k) \) represents the number of ways to choose \( k \) elements from the set, and summing over all possible values of \( k \) gives the total number of subsets. \\
    The right side of the identity, \( 2^n \), counts the same set of subsets by considering that each element in the set can either be included in a subset or not. Since there are \( n \) elements, and each element has \( 2 \) choices (to be included or not), the total number of subsets is \( 2^n \). \\
    Since both sides of the identity count the same set of subsets, we conclude that:
    \[
        \sum_{k=0}^{n} C(n, k) = 2^n
    \]
\end{eg}