\chapter{Sequences of real numbers}

\begin{definition}[Sequences]
    A sequence is a function whose domain is the set of natural numbers. In other words, a sequence is an ordered list of numbers, typically denoted as \( (a_n)_{n \in \mathbb{N}} \), where each \( a_n \) is a real number.
\end{definition}
We often denote a sequence by \( (a_n) \), $(a_n)_{n \geq 0} = \{a_0, a_1, a_2, \ldots\}$ or simply \( a_n \), where \( n \) represents the position of the term in the sequence.

\begin{eg}
    Consider the sequences defined as:
    \begin{itemize}[itemsep=1pt,label=$\circ$]
        \item \( a_n = n^2 \) (the sequence of perfect squares)
        \item \( b_n = \frac{1}{n} \) (the sequence of reciprocals)
        \item \( c_n = (-1)^n \) (the alternating sequence)
        \item \( f_0 = 0, f_1 = 1, f_{n + 2} = f_{n + 1} + f_n \) (the Fibonacci sequence)
        \item $a_n = a \cdot n + b $ (an arithmetic sequence with common difference \( a \) and initial term \( b \))
        \item $a_n = a \cdot r^n$ (a geometric sequence with common ratio \( r \) and initial term \( a \))
    \end{itemize}
\end{eg}

\section{Properties of sequences}
\begin{definition}[Bounded sequences]
    A sequence \( (a_n) \) is said to be bounded above if there exists a real number \( M \) such that \( a_n \leq M \) for all \( n \).\\ 
    It is bounded below if there exists a real number \( m \) such that \( a_n \geq m \) for all \( n \). \\
    If a sequence is both bounded above and bounded below, it is called a bounded sequence.
\end{definition}

\begin{definition}[Absolute value]
    The absolute value of a real number \( x \), denoted by \( |x| \), is defined as:
    \[
    |x| = 
    \begin{cases} 
    x & \text{if } x \geq 0 \\ 
    -x & \text{if } x < 0 
    \end{cases}
    \]
\end{definition}
Let be \( (a_n) \) a sequence. We say that \( (a_n) \) is bounded if there exists \( M > 0 \) such that \( |a_n| \leq M \) for all \( n \).
\begin{definition}[Decreasing and increasing sequences]
    A sequence \( (a_n) \) is called decreasing if \( a_n \geq a_{n+1} \) for all \( n \). It is called increasing if \( a_n \leq a_{n+1} \) for all \( n \).  
\end{definition}
\begin{definition}[Monotonic sequences]
    A sequence \( (a_n) \) is called monotonic if it is either non-decreasing or non-increasing.
\end{definition}

\begin{eg}
    Let's consider the following sequences:
    \begin{itemize}[itemsep=1pt,label=$\circ$]
        \item $a_n = n$ is increasing ($a_{n + 1} - a_n = 1 > 0$) and unbounded ($\mathbb{N}$ is not bounded above).
        \item $a_n = \frac{1}{n + 1}$ is decreasing ($a_{n + 1} - a_n = -\frac{1}{(n + 1)(n + 2)} < 0$) and bounded (by $1$ and $0$).
        \item $a_n = (-1)^n$ is neither increasing nor decreasing, but it is bounded (by $1$ and $-1$).
        \item $a_n = a \cdot n + b$ is increasing if \( a > 0 \), decreasing if \( a < 0 \), and constant if \( a = 0 \) ($a_{n + 1} - a_n = a(n + 1) + b - a \cdot n - b = a$). It is unbounded unless \( a = 0 \) (if $a > 0$, $\forall S > 0, \forall b \in \mathbb{R}, \exists n \in \mathbb{N}: a_n > S - b \ (Archimedean) \iff an + b > S$).
        \item $f_0 = 0, f_1 = 1, f_{n + 2} = f_{n + 1} + f_n$ is increasing ($f_{n + 2} - f_{n + 1} = f_n \geq 0$) and unbounded (we can prove it by induction).
    \end{itemize}
\end{eg}

\section{Recurrence relations}
\begin{definition}[Recurrence relation]
    A recurrence relation is an equation that defines each term of a sequence in terms of one or more previous terms. It typically consists of:
    \begin{itemize}[itemsep=1pt,label=$\circ$]
        \item An initial condition or base case that specifies the value(s) of the first term(s) of the sequence.
        \item A formula that relates each term to one or more preceding terms.
    \end{itemize}
\end{definition}
% TODO: complete here
\begin{eg}
    The Fibonacci sequence is defined by the recurrence relation:
    \begin{itemize}[itemsep=1pt,label=$\circ$]
        \item Initial conditions: $(f_1)^2 - f_2 \cdot f_0 = 1^2 - 1 \cdot 0 = 1$
        \item Recurrence relation: $(f_{n + 1})^2 - f_{n + 2} \cdot f_n = f_{n+1}(f_{n -1} + f_n) - f_n (f_n + f_{n + 1}) = f_{n + 1} \cdot f_{n -1} = f_{n + 1} \cdot f_{n - 1} - (f_n)^2 = -((f_n)^2)$
    \end{itemize}
\end{eg}

\begin{eg}
    Let's find the sum of the first $n$ odd natural numbers.
    \begin{itemize}[itemsep=1pt,label=$\circ$]
        \item $S_1 = 1$
        \item $S_2 = 1 + 3 = 4$
        \item $S_3 = 1 + 3 + 5 = 9$
        \item $S_4 = 1 + 3 + 5 + 7 = 16$
    \end{itemize}
    Let hypothesize that $S_n = \sum_{k = 1}^{n} (2k -1) = n^2$. We can prove this by induction:
    \begin{itemize}[itemsep=1pt,label=$\circ$]
        \item Base case: $n = 1$, $S_1 = 1^2 = 1$ (true)
        \item Inductive step: Assume $S_n = n^2$ for some $n \geq 1$. We need to show that $S_{n + 1} = (n + 1)^2$.
        \[
        S_{n + 1} = S_n + (2(n + 1) - 1) = n^2 + (2n + 1) = n^2 + 2n + 1 = (n + 1)^2
        \]
    \end{itemize}
    Thus, by the principle of mathematical induction, the formula holds for all natural numbers \( n \).
\end{eg}

\section{Limits of sequences}
\begin{definition}[Limit of a sequence]
    A sequence \( (a_n) \) is said to converge to a limit \( L \) if for every \( \epsilon > 0 \), there exists a natural number \( N \) such that for all \( n \geq N \), the absolute difference between \( a_n \) and \( L \) is less than \( \epsilon \). In mathematical notation,
    \[\lim_{n \to \infty} a_n = L \iff \forall \epsilon > 0, \exists N \in \mathbb{N}, \forall n \geq N, |a_n - L| < \epsilon.\]

    \begin{center}
        \begin{tikzpicture}[scale=1.5]
            \draw[->] (-0.5, 2) -- (6, 2) node[right] {$n$};
            \draw[->] (0, 1.5) -- (0, 4) node[above] {$a_n$};
            \draw[thick, primary, domain=0:5, samples=100] plot (\x, {3 - exp(-\x)});
            \draw[secondary, dashed] (0, 3) -- (5, 3) node[right] {$L$};
            \draw[secondary, dashed] (1, 2) -- (1, 3.5);
            \draw[secondary, dashed] (0, 2.5) -- (5, 2.5);
            \draw[secondary, dashed] (0, 3.5) -- (5, 3.5);
            \node[secondary] at (2.5, 2.75) {$\epsilon$};
            \node[secondary] at (2.5, 3.25) {$\epsilon$};
            \node[secondary] at (1, 1.7) {$N$};
        \end{tikzpicture}
    \end{center}
\end{definition}
If a sequence does not converge to any limit, it is said to diverge.

\begin{eg}
    Let $a_n = \frac{1}{\sqrt{n + 1}}$. We will show that \( \lim_{n \to \infty} a_n = 0 \). \\
    Let \( \epsilon > 0 \). We need to find \( N \in \mathbb{N} \) such that for all \( n \geq N \), \( |a_n - 0| < \epsilon \). We have:
    \[|a_n - 0| = \left|\frac{1}{\sqrt{n + 1}} - 0\right| = \frac{1}{\sqrt{n + 1}} < \epsilon \]
    This inequality can be solved as follows:
    \[
        \frac{1}{\sqrt{n + 1}} < \epsilon \iff \sqrt{n + 1} > \frac{1}{\epsilon} \iff n + 1 > \frac{1}{\epsilon^2} \iff n > \frac{1}{\epsilon^2} - 1
    \]
    Thefore, we need \( n \) to be greater than \( \frac{1}{\epsilon^2} - 1 \). Since \( n \) must be a natural number, we can choose:
    \[ N = \left\lceil \frac{1}{\epsilon^2} - 1 \right\rceil \]
    where \( \lceil x \rceil \) denotes the smallest integer greater than or equal to \( x \). Thus, for all \( n \geq N \), we have:
    \[ |a_n - 0| = \frac{1}{\sqrt{n + 1}} < \epsilon \]
    Hence, by the definition of the limit of a sequence, we conclude that \( \lim_{n \to \infty} a_n = 0 \).
\end{eg}

\begin{eg}
    Let's study the limit of the sequence defined by \( (a_n) = (-1)^{8^n - 3^n} \). We first need to determine the behavior of the exponent \( 8^n - 3^n \) as \( n \) increases. We want to determine whether \( 8^n - 3^n \) is even or odd for large \( n \).
    \begin{itemize}[itemsep=1pt,label=$\circ$]
        \item For \( n = 0 \): \( 8^0 - 3^0 = 1 - 1 = 0 \) (even)
        \item For \( n = 1 \): \( 8^1 - 3^1 = 8 - 3 = 5 \) (odd)
        \item For \( n = 2 \): \( 8^2 - 3^2 = 64 - 9 = 55 \) (odd)
        \item For \( n = 3 \): \( 8^3 - 3^3 = 512 - 27 = 485 \) (odd)
    \end{itemize}
    We notice that for all \( n \geq 1 \), $8^n$ is even and $3^n$ is odd, thus \( 8^n - 3^n \) is always odd. Therefore, for all \( n \geq 1 \), \( a_n = (-1)^{\text{odd}} = -1 \). Thefore, the sequence \( (a_n) \) is constant and equal to -1 for all \( n \geq 1 \). Hence, we can conclude that:
    \[\lim_{n \to \infty} a_n = -1 \]
\end{eg}
