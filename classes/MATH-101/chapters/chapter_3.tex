\chapter{Sequences of real numbers}

\begin{definition}[Sequences]
    A sequence is a function whose domain is the set of natural numbers. In other words, a sequence is an ordered list of numbers, typically denoted as \( (a_n)_{n \in \mathbb{N}} \), where each \( a_n \) is a real number.
\end{definition}
We often denote a sequence by \( (a_n) \), $(a_n)_{n \geq 0} = \{a_0, a_1, a_2, \ldots\}$ or simply \( a_n \), where \( n \) represents the position of the term in the sequence.

\begin{eg}
    Consider the sequences defined as:
    \begin{itemize}[itemsep=1pt,label=$\circ$]
        \item \( a_n = n^2 \) (the sequence of perfect squares)
        \item \( b_n = \frac{1}{n} \) (the sequence of reciprocals)
        \item \( c_n = (-1)^n \) (the alternating sequence)
        \item \( f_0 = 0, f_1 = 1, f_{n + 2} = f_{n + 1} + f_n \) (the Fibonacci sequence)
        \item $a_n = a \cdot n + b $ (an arithmetic sequence with common difference \( a \) and initial term \( b \))
        \item $a_n = a \cdot r^n$ (a geometric sequence with common ratio \( r \) and initial term \( a \))
    \end{itemize}
\end{eg}

\section{Properties of sequences}
\begin{definition}[Bounded sequences]
    A sequence \( (a_n) \) is said to be bounded above if there exists a real number \( M \) such that \( a_n \leq M \) for all \( n \).\\ 
    It is bounded below if there exists a real number \( m \) such that \( a_n \geq m \) for all \( n \). \\
    If a sequence is both bounded above and bounded below, it is called a bounded sequence.
\end{definition}

\begin{definition}[Absolute value]
    The absolute value of a real number \( x \), denoted by \( |x| \), is defined as:
    \[
    |x| = 
    \begin{cases} 
    x & \text{if } x \geq 0 \\ 
    -x & \text{if } x < 0 
    \end{cases}
    \]
\end{definition}
Let be \( (a_n) \) a sequence. We say that \( (a_n) \) is bounded if there exists \( M > 0 \) such that \( |a_n| \leq M \) for all \( n \).
\begin{definition}[Decreasing and increasing sequences]
    A sequence \( (a_n) \) is called decreasing if \( a_n \geq a_{n+1} \) for all \( n \). It is called increasing if \( a_n \leq a_{n+1} \) for all \( n \).  
\end{definition}
\begin{definition}[Monotonic sequences]
    A sequence \( (a_n) \) is called monotonic if it is either non-decreasing or non-increasing.
\end{definition}

\begin{eg}
    Let's consider the following sequences:
    \begin{itemize}[itemsep=1pt,label=$\circ$]
        \item $a_n = n$ is increasing ($a_{n + 1} - a_n = 1 > 0$) and unbounded ($\mathbb{N}$ is not bounded above).
        \item $a_n = \frac{1}{n + 1}$ is decreasing ($a_{n + 1} - a_n = -\frac{1}{(n + 1)(n + 2)} < 0$) and bounded (by $1$ and $0$).
        \item $a_n = (-1)^n$ is neither increasing nor decreasing, but it is bounded (by $1$ and $-1$).
        \item $a_n = a \cdot n + b$ is increasing if \( a > 0 \), decreasing if \( a < 0 \), and constant if \( a = 0 \) ($a_{n + 1} - a_n = a(n + 1) + b - a \cdot n - b = a$). It is unbounded unless \( a = 0 \) (if $a > 0$, $\forall S > 0, \forall b \in \mathbb{R}, \exists n \in \mathbb{N}: a_n > S - b \ (Archimedean) \iff an + b > S$).
        \item $f_0 = 0, f_1 = 1, f_{n + 2} = f_{n + 1} + f_n$ is increasing ($f_{n + 2} - f_{n + 1} = f_n \geq 0$) and unbounded (we can prove it by induction).
    \end{itemize}
\end{eg}

\section{Recurrence relations}
\begin{definition}[Recurrence relation]
    A recurrence relation is an equation that defines each term of a sequence in terms of one or more previous terms. It typically consists of:
    \begin{itemize}[itemsep=1pt,label=$\circ$]
        \item An initial condition or base case that specifies the value(s) of the first term(s) of the sequence.
        \item A formula that relates each term to one or more preceding terms.
    \end{itemize}
\end{definition}
A recurrence relation can be visualized by dominoes. Each domino represents a term in the sequence. The initial condition is when we push the first domino and the recurrence relation is how each domino knocks over the next one making it fall.
\begin{eg}
    The Fibonacci sequence is defined by the recurrence relation:
    \begin{itemize}[itemsep=1pt,label=$\circ$]
        \item Initial conditions: $(f_1)^2 - f_2 \cdot f_0 = 1^2 - 1 \cdot 0 = 1$
        \item Recurrence relation: $(f_{n + 1})^2 - f_{n + 2} \cdot f_n = f_{n+1}(f_{n -1} + f_n) - f_n (f_n + f_{n + 1}) = f_{n + 1} \cdot f_{n -1} = f_{n + 1} \cdot f_{n - 1} - (f_n)^2 = -((f_n)^2)$
    \end{itemize}
\end{eg}

\begin{eg}
    Let's find the sum of the first $n$ odd natural numbers.
    \begin{itemize}[itemsep=1pt,label=$\circ$]
        \item $S_1 = 1$
        \item $S_2 = 1 + 3 = 4$
        \item $S_3 = 1 + 3 + 5 = 9$
        \item $S_4 = 1 + 3 + 5 + 7 = 16$
    \end{itemize}
    Let hypothesize that $S_n = \sum_{k = 1}^{n} (2k -1) = n^2$. We can prove this by induction:
    \begin{itemize}[itemsep=1pt,label=$\circ$]
        \item Base case: $n = 1$, $S_1 = 1^2 = 1$ (true)
        \item Inductive step: Assume $S_n = n^2$ for some $n \geq 1$. We need to show that $S_{n + 1} = (n + 1)^2$.
        \[
        S_{n + 1} = S_n + (2(n + 1) - 1) = n^2 + (2n + 1) = n^2 + 2n + 1 = (n + 1)^2
        \]
    \end{itemize}
    Thus, by the principle of mathematical induction, the formula holds for all natural numbers \( n \).
\end{eg}

\section{Limits of sequences}
\begin{definition}[Limit of a sequence]
    A sequence \( (a_n) \) is said to converge to a limit \( L \) if for every \( \epsilon > 0 \), there exists a natural number \( N \) such that for all \( n \geq N \), the absolute difference between \( a_n \) and \( L \) is less than \( \epsilon \). In mathematical notation,
    \[\lim_{n \to \infty} a_n = L \iff \forall \epsilon > 0, \exists N \in \mathbb{N}, \forall n \geq N, |a_n - L| < \epsilon.\]

    \begin{center}
        \begin{tikzpicture}[scale=1.5]
            \draw[->] (-0.5, 2) -- (6, 2) node[right] {$n$};
            \draw[->] (0, 1.5) -- (0, 4) node[above] {$a_n$};
            \draw[thick, primary, domain=0:5, samples=100] plot (\x, {3 - exp(-\x)});
            \draw[secondary, dashed] (0, 3) -- (5, 3) node[right] {$L$};
            \draw[secondary, dashed] (1, 2) -- (1, 3.5);
            \draw[secondary, dashed] (0, 2.5) -- (5, 2.5);
            \draw[secondary, dashed] (0, 3.5) -- (5, 3.5);
            \node[secondary] at (2.5, 2.75) {$\epsilon$};
            \node[secondary] at (2.5, 3.25) {$\epsilon$};
            \node[secondary] at (1, 1.7) {$N$};
        \end{tikzpicture}
    \end{center}
\end{definition}
If a sequence does not converge to any limit, it is said to diverge.

\begin{eg}
    Let $a_n = \frac{1}{\sqrt{n + 1}}$. We will show that \( \lim_{n \to \infty} a_n = 0 \). \\
    Let \( \epsilon > 0 \). We need to find \( N \in \mathbb{N} \) such that for all \( n \geq N \), \( |a_n - 0| < \epsilon \). We have:
    \[|a_n - 0| = \left|\frac{1}{\sqrt{n + 1}} - 0\right| = \frac{1}{\sqrt{n + 1}} < \epsilon \]
    This inequality can be solved as follows:
    \[
        \frac{1}{\sqrt{n + 1}} < \epsilon \iff \sqrt{n + 1} > \frac{1}{\epsilon} \iff n + 1 > \frac{1}{\epsilon^2} \iff n > \frac{1}{\epsilon^2} - 1
    \]
    Thefore, we need \( n \) to be greater than \( \frac{1}{\epsilon^2} - 1 \). Since \( n \) must be a natural number, we can choose:
    \[ N = \left\lceil \frac{1}{\epsilon^2} - 1 \right\rceil \]
    where \( \lceil x \rceil \) denotes the smallest integer greater than or equal to \( x \). Thus, for all \( n \geq N \), we have:
    \[ |a_n - 0| = \frac{1}{\sqrt{n + 1}} < \epsilon \]
    Hence, by the definition of the limit of a sequence, we conclude that \( \lim_{n \to \infty} a_n = 0 \).
\end{eg}

\begin{eg}
    Let's study the limit of the sequence defined by \( (a_n) = (-1)^{8^n - 3^n} \). We first need to determine the behavior of the exponent \( 8^n - 3^n \) as \( n \) increases. We want to determine whether \( 8^n - 3^n \) is even or odd for large \( n \).
    \begin{itemize}[itemsep=1pt,label=$\circ$]
        \item For \( n = 0 \): \( 8^0 - 3^0 = 1 - 1 = 0 \) (even)
        \item For \( n = 1 \): \( 8^1 - 3^1 = 8 - 3 = 5 \) (odd)
        \item For \( n = 2 \): \( 8^2 - 3^2 = 64 - 9 = 55 \) (odd)
        \item For \( n = 3 \): \( 8^3 - 3^3 = 512 - 27 = 485 \) (odd)
    \end{itemize}
    We notice that for all \( n \geq 1 \), $8^n$ is even and $3^n$ is odd, thus \( 8^n - 3^n \) is always odd. Therefore, for all \( n \geq 1 \), \( a_n = (-1)^{\text{odd}} = -1 \). Thefore, the sequence \( (a_n) \) is constant and equal to -1 for all \( n \geq 1 \). Hence, we can conclude that:
    \[\lim_{n \to \infty} a_n = -1 \]
\end{eg}

\begin{eg}
    Let $p \in \mathbb{Q}, p > 0$ and $a_0 = 1, a_n = \frac{1}{n^p}$. Let's show that \( \lim_{n \to \infty} a_n = 0 \). \\
    Let \( \epsilon > 0 \). We need to find \( N \in \mathbb{N} \) such that for all \( n \geq N \), \( |a_n - 0| < \epsilon \). We have:
    \[|a_n - 0| = \left|\frac{1}{n^p} - 0\right| = \frac{1}{n^p} < \epsilon \]
    This inequality can be solved as follows:
    \[
        \frac{1}{n^p} < \epsilon \iff n^p > \frac{1}{\epsilon} \iff n > \left(\frac{1}{\epsilon}\right)^{\frac{1}{p}}
    \]
    Thefore, we need \( n \) to be greater than \( \left(\frac{1}{\epsilon}\right)^{\frac{1}{p}} \). Since \( n \) must be a natural number, we can choose:
    \[ N = \left\lceil \left(\frac{1}{\epsilon}\right)^{\frac{1}{p}} \right\rceil \]
    where \( \lceil x \rceil \) denotes the smallest integer greater than or equal to \( x \). Thus, for all \( n \geq N \), we have:
    \[ |a_n - 0| = \frac{1}{n^p} < \epsilon \]
    Hence, by the definition of the limit of a sequence, we conclude that \( \lim_{n \to \infty} a_n = 0 \).
\end{eg}

\begin{definition}[Uniqueness of limits]
    If a sequence \( (a_n) \) converges to a limit \( L \), then this limit is unique. In other words, if \( \lim_{n \to \infty} a_n = L_1 \) and \( \lim_{n \to \infty} a_n = L_2 \), then \( L_1 = L_2 \).
\end{definition}

\begin{definition}[Triangle inequality]
    For any real numbers \( x \) and \( y \), the triangle inequality states that:
    \[ |x + y| \leq |x| + |y| \]
\end{definition}
\begin{proof}
    Let \( x, y \in \mathbb{R} \). We will prove the triangle inequality by considering two casses based on the sign of \( x + y \).
    \begin{itemize}[itemsep=1pt,label=$\circ$]
        \item Case 1: \( x + y \geq 0 \)
        \[ |x + y| = x + y \leq |x| + |y| \]
        \item Case 2: \( x + y < 0 \)
        \[ |x + y| = -(x + y) = -x - y \leq |x| + |y| \]
    \end{itemize}
    In both cases, we have shown that \( |x + y| \leq |x| + |y| \). Therefore, the triangle inequality holds for all real numbers \( x \) and \( y \).
\end{proof}

\begin{eg}
    Let's prove that the sequence $a_n = (-1)^n$ does not converge. Let's suppose, for the sake of contradiction, that the sequence converges to a limit \( L \). According to the definition of convergence, for every \( \epsilon > 0 \), there exists a natural number \( N \) such that for all \( n \geq N \), \( |a_n - L| < \epsilon \).
    Let's choose \( \epsilon = 0.5 \). Then, there exists \( N \in \mathbb{N} \) such that for all \( n \geq N \), \( |a_n - L| < 0.5 \).
    Now, consider the terms \( a_N \) and \( a_{N + 1} \):
    \begin{itemize}[itemsep=1pt,label=$\circ$]
        \item If \( N \) is even, then \( a_N = 1 \) and \( a_{N + 1} = -1 \).
        \item If \( N \) is odd, then \( a_N = -1 \) and \( a_{N + 1} = 1 \).
    \end{itemize}
    In either case, we have:
    \[ |a_N - a_{N + 1}| = |1 - (-1)| = 2 \]
    However, by the triangle inequality, we also have:
    \[ |a_N - a_{N + 1}| \leq |a_N - L| + |L - a_{N + 1}| < 0.5 + 0.5 = 1 \]
    This leads to a contradiction since \( 2 \leq 1 \) is false. Therefore, our initial assumption that the sequence \( (a_n) = (-1)^n \) converges must be incorrect. Hence, the sequence \( (a_n) = (-1)^n \) does not converge.
\end{eg}
Every sequence that converges is bounded. The converse is not true: a bounded sequence does not necessarily converge (e.g., $a_n = (-1)^n$).
\begin{proof}
    Let \( (a_n) \) be a sequence that converges to a limit \( L \). By the definition of convergence, for every \( \epsilon > 0 \), there exists a natural number \( N \) such that for all \( n \geq N \), \( |a_n - L| < \epsilon \).
    Let's choose \( \epsilon = 1 \). Then, there exists \( N \in \mathbb{N} \) such that for all \( n \geq N \), \( |a_n - L| < 1 \). This implies that:
    \[ -1 < a_n - L < 1 \]
    Adding \( L \) to all parts of the inequality, we get:
    \[ L - 1 < a_n < L + 1 \]
    This shows that for all \( n \geq N \), the terms of the sequence \( (a_n) \) are bounded between \( L - 1 \) and \( L + 1 \).
    Now, consider the finite set of terms \( a_0, a_1, a_2, \ldots, a_{N-1} \). Since this is a finite set, it has both a maximum and a minimum value. Let:
    \[ M_1 = \max\{a_0, a_1, a_2, \ldots, a_{N-1}\} \]
    and
    \[ m_1 = \min\{a_0, a_1, a_2, \ldots, a_{N-1}\} \]
    Now we can define:
    \[ M = \max(M_1, L + 1) \]
    and
    \[ m = \min(m_1, L - 1) \]
    Thus, for all \( n < N \), we have \( m_1 \leq a_n \leq M_1 \), and for all \( n \geq N \), we have \( L - 1 < a_n < L + 1 \). Therefore, for all \( n \in \mathbb{N} \), we have:
    \[ m < a_n < M \]
    This shows that the sequence \( (a_n) \) is bounded above by \( M \) and bounded below by \( m \). Hence, the sequence \( (a_n) \) is bounded.
\end{proof}

\subsection{Operations on limits}
\begin{theorem}
    Let \( (a_n) \) and \( (b_n) \) be two sequences that converge to limits \( L_1 \) and \( L_2 \) respectively. Then:
    \begin{itemize}[itemsep=1pt,label=$\circ$]
        \item The sequence \( (a_n \pm b_n) \) converges to \( L_1 \pm L_2 \).
        \item The sequence \( (a_n \cdot b_n) \) converges to \( L_1 \cdot L_2 \).
        \item If \( L_2 \neq 0 \), the sequence \( \left(\frac{a_n}{b_n}\right) \) converges to \( \frac{L_1}{L_2} \).
        \item The sequence \( p \cdot (a_n) \) converges to \( p \cdot L_1 \) for any constant \( p \in \mathbb{R} \).
    \end{itemize}
\end{theorem}
Note that if $(a_n + b_n)$ converges, then either both $(a_n)$ and $(b_n)$ converge, or both diverge. Also if $\lim_{n \to \infty} (a_n - b_n) = 0$, then $\lim_{n \to \infty} a_n = \lim_{n \to \infty} b_n$ or both diverge. \\
When $(a_n \cdot b_n)$ converges, we have:
\begin{itemize}[itemsep=1pt,label=$\circ$]
    \item $a_n \to L_1$ and $b_n \to L_2$. (e.g., $a_n = \frac{1}{n + 1}, b_n = 1$ and $a_n \cdot b_n = \frac{1}{n + 1} \to 0$)
    \item $a_n$ and $b_n$ diverge. (e.g., $a_n = (-1)^n, b_n = (-1)^n + \frac{1}{n + 1}$ and $a_n \cdot b_n = 1 + \frac{(-1)^n}{n + 1}$)
    \item $a_n \to L$ and $b_n$ diverges. (e.g., $a_n = (n + 1), b_n = \frac{1}{(n + 1)^2}$ and $a_n \cdot b_n = \frac{1}{n + 1}$)
\end{itemize}
But if $b_n \to L \neq 0$ then $a_n$ must converge. \\
 
\subsection{Quotients of Polynomial Sequences}
\begin{theorem}
    Let \( (a_n) \) and \( (b_n) \) be sequences defined by polynomials \( P(n) \) and \( Q(n) \) respectively, where:
    \[
    a_n = P(n) = a_k n^k + a_{k-1} n^{k-1} + \ldots + a_1 n + a_0
    \]
    \[
    b_n = Q(n) = b_m n^m + b_{m-1} n^{m-1} + \ldots + b_1 n + b_0
    \]
    with \( a_k, b_m \neq 0 \). The limit of the quotient of these polynomial sequences as \( n \) approaches infinity can be determined based on the degrees of the polynomials \( k \) and \( m \):
    \begin{itemize}[itemsep=1pt,label=$\circ$]
        \item If \( k < m \), then \( \lim_{n \to \infty} \frac{a_n}{b_n} = 0 \).
        \item If \( k = m \), then \( \lim_{n \to \infty} \frac{a_n}{b_n} = \frac{a_k}{b_m} \).
        \item If \( k > m \), then \( \frac{a_n}{b_n} \) diverges.
    \end{itemize}
\end{theorem}
\begin{proof}
    Let \( (a_n) \) and \( (b_n) \) be sequences defined by polynomials \( P(n) \) and \( Q(n) \) respectively, where:
    \[
    a_n = P(n) = a_k n^k + a_{k-1} n^{k-1} + \ldots + a_1 n + a_0
    \]
    \[
    b_n = Q(n) = b_m n^m + b_{m-1} n^{m-1} + \ldots + b_1 n + b_0
    \]
    with \( a_k, b_m \neq 0 \). We analyze the limit of the quotient \( \frac{a_n}{b_n} \) as \( n \) approaches infinity based on the degrees of the polynomials \( k \) and \( m \):
    \[
        \frac{a_n}{b_n} = \frac{a_k n^k + a_{k-1} n^{k-1} + \ldots + a_1 n + a_0}{b_m n^m + b_{m-1} n^{m-1} + \ldots + b_1 n + b_0} = \frac{n^k}{n^m} \cdot \frac{a_k + \frac{a_{k-1}}{n} + \ldots + \frac{a_1}{n^{k-1}} + \frac{a_0}{n^k}}{b_m + \frac{b_{m-1}}{n} + \ldots + \frac{b_1}{n^{m-1}} + \frac{b_0}{n^m}}
    \]
    We easily see that:
    \[
        \lim_{n \to \infty} \frac{a_k + \frac{a_{k-1}}{n} + \ldots + \frac{a_1}{n^{k-1}} + \frac{a_0}{n^k}}{b_m + \frac{b_{m-1}}{n} + \ldots + \frac{b_1}{n^{m-1}} + \frac{b_0}{n^m}} = \frac{a_k}{b_m}
    \]
    Thus we have three cases:
    \begin{itemize}[itemsep=1pt,label=$\circ$]
        \item If \( k < m \), then \( \frac{n^k}{n^m} = \frac{1}{n^{m-k}} \to 0 \) as \( n \to \infty \). Therefore, \( \lim_{n \to \infty} \frac{a_n}{b_n} = 0 \).
        \item If \( k = m \), then \( \frac{n^k}{n^m} = 1 \). Therefore, \( \lim_{n \to \infty} \frac{a_n}{b_n} = \frac{a_k}{b_m} \).
        \item If \( k > m \), then \( \frac{n^k}{n^m} = n^{k-m} \to \infty \) as \( n \to \infty \). Therefore, the quotient \( \frac{a_n}{b_n} \) diverges.
    \end{itemize}
    This completes the proof.
\end{proof}

\begin{eg}
    Let $a_n = \frac{3n^2 + 5n + 1}{2n^2 + n + 10}$. Then we have:
    \[
        \lim_{n \to \infty} a_n = \lim_{n \to \infty} \frac{3n^2 + 5n + 1}{2n^2 + n + 10} = \lim_{n \to \infty} \frac{n^2}{n^2} \cdot \frac{3 + \frac{5}{n} + \frac{1}{n^2}}{2 + \frac{1}{n} + \frac{10}{n^2}} = \frac{3}{2}
    \]
\end{eg}

\subsection{Squeeze theorem}
\begin{theorem}
    Let $(a_n)$ and $(b_n)$ be two sequences such that $(a_n) \to a$ and $(b_n) \to b$. If there exists a $m_0 \in \mathbb{N} : \forall n \geq m_0 \implies a_n \geq b_n$, then \( a \geq b \).
\end{theorem}
\begin{proof}
    Let's prove the theorem by contradiction. Suppose that \( a < b \). Then, we can choose \( \epsilon = \frac{b - a}{2} > 0 \). By the definition of convergence, there exist natural numbers \( N_1 \) and \( N_2 \) such that for all \( n \geq N_1 \), \( |a_n - a| < \epsilon \) and for all \( n \geq N_2 \), \( |b_n - b| < \epsilon \).
    Let \( N = \max(N_1, N_2, m_0) \). Then, for all \( n \geq N \), we have:
    \[
        |a_n - a| < \epsilon \implies a_n < a + \epsilon = a + \frac{b - a}{2} = \frac{a + b}{2}
    \]
    and
    \[
        |b_n - b| < \epsilon \implies b_n > b - \epsilon = b - \frac{b - a}{2} = \frac{a + b}{2}
    \]
    Therefore, for all \( n \geq N \), we have:
    \[
        a_n > \frac{a + b}{2} > b_n
    \]
    This contradicts our initial assumption that there exists \( m_0 \in \mathbb{N} : \forall n \geq m_0, a_n \geq b_n \). Hence, our assumption that \( a < b \) must be false. Therefore, we conclude that \( a \geq b \).
\end{proof}

\begin{theorem}[Squeeze theorem]
    Let \( (a_n) \), \( (b_n) \), and \( (c_n) \) be three sequences such that \( a_n \leq b_n \leq c_n \) for all \( n \) greater than some index \( N \). If \( \lim_{n \to \infty} a_n = L \) and \( \lim_{n \to \infty} c_n = L \), then \( \lim_{n \to \infty} b_n = L \).
\end{theorem}
\begin{proof}
    Let \( (a_n) \), \( (b_n) \), and \( (c_n) \) be three sequences such that \( a_n \leq b_n \leq c_n \) for all \( n \) greater than some index \( N \). We are given that \( \lim_{n \to \infty} a_n = L \) and \( \lim_{n \to \infty} c_n = L \). We want to show that \( \lim_{n \to \infty} b_n = L \).
    By the definition of limits, for every \( \epsilon > 0 \), there exist natural numbers \( N_1 \) and \( N_2 \) such that for all \( n \geq N_1 \), \( |a_n - L| < \epsilon \) and for all \( n \geq N_2 \), \( |c_n - L| < \epsilon \).
    Let \( N' = \max(N, N_1, N_2) \). Then, for all \( n \geq N' \), we have:
    \[
        |a_n - L| < \epsilon \implies L - \epsilon < a_n < L + \epsilon
    \]
    and
    \[
        |c_n - L| < \epsilon \implies L - \epsilon < c_n < L + \epsilon
    \]
    Since \( a_n \leq b_n \leq c_n \), it follows that:
    \[
        L - \epsilon < a_n \leq b_n \leq c_n < L + \epsilon
    \]
    Therefore, for all \( n \geq N' \), we have:
    \[
        |b_n - L| < \epsilon
    \]
    This shows that for every \( \epsilon > 0 \), there exists a natural number \( N' \) such that for all \( n \geq N' \), \( |b_n - L| < \epsilon \). By the definition of limits, we conclude that \( \lim_{n \to \infty} b_n = L \).
\end{proof}

\begin{eg}
    Let's find the limit of the sequences defined by $(a_n) = \frac{5n^2 + 6n \cos (n)}{4n^2 + 3}$ and $(b_n) = \frac{\sqrt{3n^2 + 1}}{5n + 2}$. We will use the squeeze theorem to find these limits. We have:
    \[ -1 \leq \cos(n) \leq 1 \implies 5n^2 - 6n \leq 5n^2 + 6n \cos(n) \leq 5n^2 + 6n \]
    Thus:
    \[ \frac{5n^2 - 6n}{4n^2 + 3} \leq a_n \leq \frac{5n^2 + 6n}{4n^2 + 3} \]
    We can find the limits of the sequences on the left and right:
    \[ \lim_{n \to \infty} \frac{5n^2 - 6n}{4n^2 + 3} = \lim_{n \to \infty} \frac{n^2}{n^2} \cdot \frac{5 - \frac{6}{n}}{4 + \frac{3}{n^2}} = \frac{5}{4} \]
    \[ \lim_{n \to \infty} \frac{5n^2 + 6n}{4n^2 + 3} = \lim_{n \to \infty} \frac{n^2}{n^2} \cdot \frac{5 + \frac{6}{n}}{4 + \frac{3}{n^2}} = \frac{5}{4} \]
    By the squeeze theorem, we conclude that:
    \[ \lim_{n \to \infty} a_n = \frac{5}{4} \]
    Now, let's find the limit of the sequence \( (b_n) = \frac{\sqrt{3n^2 + 1}}{5n + 2} \). We have:
    \[
        \frac{\sqrt{3n^2 + 1}}{5n + 2} = \frac{\sqrt{3n^2} \sqrt{1 + \frac{1}{3n^2}}}{5n + 2} = \frac{n \sqrt{3} \sqrt{1 + \frac{1}{3n^2}}}{5n + 2} = \frac{\sqrt{3} \sqrt{1 + \frac{1}{3n^2}}}{5 + \frac{2}{n}}
    \]
    We can find bounds for \( b_n \):
    \[
        1 \leq \sqrt{1 + \frac{1}{3n^2}} \leq \sqrt{1 + \frac{1}{3n^2} + (\frac{1}{6n^2})^2} = \sqrt{(1 + \frac{1}{6n^2})2} = 1 + \frac{1}{6n^2}
    \]
    Thus:
    \[ \frac{\sqrt{3}}{5 + \frac{2}{n}} \leq b_n \leq \frac{\sqrt{3}(1 + \frac{1}{6n^2})}{5 + \frac{2}{n}} \]
    We can find the limits of the sequences on the left and right:
    \[ \lim_{n \to \infty} \frac{\sqrt{3}}{5 + \frac{2}{n}} = \frac{\sqrt{3}}{5} \]
    \[ \lim_{n \to \infty} \frac{\sqrt{3}(1 + \frac{1}{6n^2})}{5 + \frac{2}{n}} = \frac{\sqrt{3}}{5} \]
    By the squeeze theorem, we conclude that:
    \[ \lim_{n \to \infty} b_n = \frac{\sqrt{3}}{5} \]
\end{eg}

\begin{eg}
    Let's prove the limit of $a_n = \sqrt[n]{a}, \ a > 0$ is $1$. \\
    Let $a > 1$ then $\forall x \in \mathbb{R} \implies (x-1)(x^{n-1} + x^{n-2} + \ldots + x + 1) = x^n -1, \ \forall n \in \mathbb{N}^*$. Thus, we have:
    \[
        x-1 = \frac{x^n - 1}{x^{n-1} + x^{n-2} + \ldots + x + 1}
    \]
    Let $x = \sqrt[n]{a} > 1$ then we have:
    \[
        0 < \sqrt[n]{a} - 1 = \frac{a - 1}{\sqrt[n]{a^{n-1}} + \sqrt[n]{a^{n-2}} + \ldots + \sqrt[n]{a} + 1} < \frac{a - 1}{n}
    \]
    By the squeeze theorem, we conclude that:
    \[ \lim_{n \to \infty} a_n = 1 \]
    If $a = 1$ then $a_n = 1$ thus $\lim_{n \to \infty} a_n = 1$. \\
    If $0 < a < 1$ then we can write $a_n = \sqrt[n]{\frac{1}{b}}$ with $b = \frac{1}{a} > 1$. Thus, we have:
    \[ \lim_{n \to \infty} a_n = \lim_{n \to \infty} \sqrt[n]{\frac{1}{b}} = \frac{1}{\lim_{n \to \infty} \sqrt[n]{b}} = \frac{1}{1} = 1 \]
\end{eg}

\subsection{Geometric Sequences}
\begin{definition}[Geometric sequence]
    A geometric sequence is a sequence of numbers where each term after the first is found by multiplying the previous term by a constant called the common ratio. The general form of a geometric sequence can be expressed as:
    \[
        a_n = a_0 \cdot r^{n}
    \]
    for \( n \in \mathbb{N} \), \( a_0 \in \mathbb{R}, \ a_0 \neq 0\) and \( r \in \mathbb{R} \). We then have the following limites:
    \begin{itemize}[itemsep=1pt,label=$\circ$]
        \item If \( |r| < 1 \), then \( \lim_{n \to \infty} a_n = 0 \).
        \item If \( r = 1 \), then \( \lim_{n \to \infty} a_n = a_0 \).
        \item If $|r| > 1$ or \( r = -1 \), then the sequence diverges.
    \end{itemize}
\end{definition}
\begin{proof}
    Let \( (a_n) \) be a geometric sequence defined by:
    \[
        a_n = a_0 \cdot r^{n}
    \]
    for \( n \in \mathbb{N} \), \( a_0 \in \mathbb{R}, \ a_0 \neq 0\) and \( r \in \mathbb{R} \). We analyze the limit of the sequence \( (a_n) \) as \( n \) approaches infinity based on the value of the common ratio \( r \): \\
    \textbf{1. If $r > 1$}: \\
    We can rewrite \( r \) as \( r = 1 + x \) where \( x > 0 \). Then we have:
    \[
        r^n = (1 + x)^n = 1 + \begin{pmatrix}
            n \\ 1
        \end{pmatrix}x + \begin{pmatrix}
            n \\ 2
        \end{pmatrix}x^2 + \ldots + \begin{pmatrix}
            n \\ n
        \end{pmatrix}x^n \geq 1 + \begin{pmatrix}
            n \\ 1
        \end{pmatrix}x = 1 + nx \quad \forall n \geq 1
    \]
    Then by the Archimedean property, $\forall M > 0$, $\exists n \in \mathbb{N}$ : $(1 + nx) > M$:
    \[
        |a_0r^n| = |a_0(1+x)^n| \geq |a_0||1 + nx| > |a_0|M
    \]
    Thus, the sequence is not bounded and diverges. \\
    \textbf{2. If $0 < r < 1$}: \\
    We can rewrite as \( q = \frac{1}{r} > 1 \) and we have $\forall M > 0$, $\exists n_0 \in \mathbb{N}$ : $q^n > M$ for all $n \geq n_0$. Let $\epsilon > 0$ and $M = \frac{|a_0|}{\epsilon}$ we then have:
    \[
        q^n > \frac{|a_0|}{\epsilon} \implies \frac{1}{q^n} = r^n < \frac{\epsilon}{|a_0|} \quad \forall n \geq n_0
    \]
    Thus, we have:
    \[
        |a_0 r^n - 0| < \epsilon \quad \forall n \geq n_0
    \]
    By the definition of limits, we conclude that:
    \[ \lim_{n \to \infty} a_n = 0 \]
    \textbf{3. If $r = 1$}: \\
    We have $a_0 r^n = a_0$ for all $n \in \mathbb{N}$. Thus, we have:
    \[
        \lim_{n \to \infty} a_0 r^n = a_0
    \]
\end{proof}

\begin{eg}
    Let $a_n = (-3)^{-3n}$. Then we have:
    \[ a_n = ((-3)^{-3})^n = ((-1)^{-3} \cdot 3^{-3})^n = (-1 \cdot \frac{1}{27})^n = (-\frac{1}{27})^n \]
    Then $r = -\frac{1}{27}$ thus $|r| < 1$. By the definition of geometric sequence, we conclude that:
    \[ \lim_{n \to \infty} a_n = 0 \]
\end{eg}

\begin{eg}
    Let $a_n = \sqrt{\frac{2^{2n + 1}}{3^{n + 8}}}$. Then we have:
    \[ a_n = \sqrt{\frac{2 \cdot 4^n}{3^8 \cdot 3^n}} = \frac{\sqrt{2}}{\sqrt{3^8}} \cdot \left(\frac{4}{3}\right)^{n/2} = \frac{\sqrt{2}}{81} \left(\sqrt{\frac{4}{3}}\right)^n \]
    Then $r = \sqrt{\frac{4}{3}}$ thus $|r| > 1$. By the definition of geometric sequence, we conclude that the sequence diverges.
\end{eg}

\subsection{Properties of Limits}
Some properties of limits:
\begin{itemize}[itemsep=1pt,label=$\circ$]
    \item If $\lim_{n \to \infty} x_n = l \in \mathbb{R}$, then $\lim_{n \to \infty} |x_n| = |l|$.
    \item If $\lim_{n \to \infty} |x_n| = 0$, then $\lim_{n \to \infty} x_n = 0$.
    \item If $\lim_{n \to \infty} |x_n| = l \in \mathbb{R}^*$, then $\lim_{n \to \infty} x_n$ does not exist.
    \item If $a_n$ is bounded and $b_n \to 0$, then $a_n b_n \to 0$.
\end{itemize}
\begin{proof}
    Let's prove that if $a_n$ is bounded and $b_n \to 0$, then $a_n b_n \to 0$. \\
    Since \( (a_n) \) is bounded, there exists a constant \( M > 0 \) such that \( |a_n| \leq M \) for all \( n \in \mathbb{N} \). Since \( (b_n) \) converges to 0, for every \( \epsilon > 0 \), there exists a natural number \( N \) such that for all \( n \geq N \), \( |b_n - 0| < \frac{\epsilon}{M} \). This implies that:
    \[ |b_n| < \frac{\epsilon}{M} \]
    Now, for all \( n \geq N \), we have:
    \[ |a_n b_n| = |a_n| \cdot |b_n| \leq M \cdot |b_n| < M \cdot \frac{\epsilon}{M} = \epsilon \]
    This shows that for every \( \epsilon > 0 \), there exists a natural number \( N \) such that for all \( n \geq N \), \( |a_n b_n - 0| < \epsilon \). By the definition of limits, we conclude that \( \lim_{n \to \infty} a_n b_n = 0 \).
\end{proof}

\subsection{Alembert's test}
\begin{theorem}[Alembert's test]
    Let \( (a_n) \) be a sequence such that \( a_n \neq 0 \) for all \( n \in \mathbb{N} \) and \( \lim_{n \to \infty} \left| \frac{a_{n+1}}{a_n} \right| = L \geq 0 \). Then:
    \begin{itemize}[itemsep=1pt,label=$\circ$]
        \item If \( L < 1 \), then \( \lim_{n \to \infty} a_n = 0 \).
        \item If \( L > 1 \), then the sequence \( (a_n) \) diverges.
        \item If \( L = 1 \), the test is inconclusive.
    \end{itemize}
\end{theorem}
\begin{proof}
    Let $L < 1$ Then we have:
    \[
        \lim_{n \to \infty} \left| \frac{a_{n+1}}{a_n} \right| = L < 1 \implies \forall \epsilon > 0, \ \exists N \in \mathbb{N} : \forall n \geq N, \left| \frac{a_{n+1}}{a_n} \right| \leq L + \epsilon < 1
    \]
    Let $m > N$, we have:
    \[
        \underbrace{\left| \frac{a_m}{a_{m -1}} \right|}_{\leq L + \epsilon} \cdot \underbrace{\left| \frac{a_{m - 1}}{a_{m - 2}} \right|}_{\leq L + \epsilon} \cdot \ldots \cdot \underbrace{\left| \frac{a_{N + 1}}{a_N} \right|}_{\leq L + \epsilon} = \left| \frac{a_m}{a_N} \right| \leq (L + \epsilon)^{m - N}
    \]
    We then have:
    \[ 0  \leq |a_m| \leq |a_N| (L + \epsilon)^{m - N} \to 0 \ \text{(Geometric Sequence)}\]
    By the squeeze theorem, we conclude that:
    \[ \lim_{m \to \infty} a_m = 0 \]
    Let $L > 1$. Then we have:
    \[ \lim_{n \to \infty} \left| \frac{a_{n+1}}{a_n} \right| = L > 1 \implies \forall \epsilon > 0, \ \exists N \in \mathbb{N} : \forall n \geq N, \left| \frac{a_{n+1}}{a_n} \right| \geq L - \epsilon > 1 \]
    Let $m > N$, we have:
    \[ \underbrace{\left| \frac{a_m}{a_{m -1}} \right|}_{\geq L - \epsilon} \cdot \underbrace{\left| \frac{a_{m - 1}}{a_{m - 2}} \right|}_{\geq L - \epsilon} \cdot \ldots \cdot \underbrace{\left| \frac{a_{N + 1}}{a_N} \right|}_{\geq L - \epsilon} = \left| \frac{a_m}{a_N} \right| \geq (L - \epsilon)^{m - N}
    \]
    We then have:
    \[ |a_m| \geq |a_N| (L - \epsilon)^{m - N} \to \infty \ \text{(Geometric Sequence)}\]
    Thus, the sequence diverges.
\end{proof}
If $L = 1$ the test is inconclusive.
\begin{eg}
    Let $a_n = n$ then we have:
    \[ \left| \frac{a_{n+1}}{a_n} \right| = \frac{n + 1}{n} = 1 + \frac{1}{n} \to 1 \]
    The sequence diverges.
\end{eg}
\begin{eg}
    Let's compute $\lim_{n \to \infty} \frac{n^3}{5^n}$. By applying Alembert's test, we have:
    \[ \left| \frac{a_{n+1}}{a_n} \right| = \frac{(n + 1)^3}{5^{n + 1}} \cdot \frac{5^n}{n^3} = \frac{(n + 1)^3}{5 n^3} = \frac{1}{5} \left(1 + \frac{1}{n}\right)^3 \to \frac{1}{5} < 1 \]
    By the definition of Alembert's test, we conclude that:
    \[ \lim_{n \to \infty} \frac{n^3}{5^n} = 0 \]
\end{eg}
\begin{eg}
    Let's compute $\lim_{n \to \infty} \frac{5^n}{n!}$. By applying Alembert's test, we have:
    \[ \left| \frac{a_{n+1}}{a_n} \right| = \frac{5^{n + 1}}{(n + 1)!} \cdot \frac{n!}{5^n} = \frac{5}{n + 1} \to 0 < 1 \]
    By the definition of Alembert's test, we conclude that:
    \[ \lim_{n \to \infty} \frac{5^n}{n!} = 0 \]
\end{eg}
Note that $\lim_{n \to \infty} \frac{S^n}{n!} = 0$ since $n!$ grows faster than $S^n$ for any constant $S$.

\section{Infinite Limits}
\begin{definition}[Infinite limit]
    A sequence \( (a_n) \) is said to diverge to infinity (or negative infinity) if for every real number \( M > 0 \) (or \( M < 0 \)), there exists a natural number \( N \) such that for all \( n \geq N \), \( a_n > M \) (or \( a_n < M \)). In this case, we write:
    \[ \lim_{n \to \infty} a_n = +\infty \quad \text{or} \quad \lim_{n \to \infty} a_n = -\infty \]
\end{definition}
Note that not all divergent sequences diverge to infinity. For example, the sequence defined by \( a_n = (-1)^n \) oscillates between $-1$ and $1$ and does not diverge to infinity or negative infinity.

\subsection{Properties of infinite limits and indeterminate forms}
\begin{itemize}[itemsep=1pt,label=$\circ$]
    \item $\lim_{n \to \infty} a_n = \infty = \lim_{n \to \infty} b_n \implies \lim_{n \to \infty} (a_n + b_n) = \infty$
    \item $\lim_{n \to \infty} a_n = \pm \infty$ and $b_n$ is bounded $\implies \lim_{n \to \infty} (a_n + b_n) = \pm \infty$
    \item $\lim_{n \to \infty} b_n = \infty$ and $a_n \geq b_n \quad (\forall n \geq n_0) \implies \lim_{n \to \infty} a_n = \infty$
    \item If $a_n$ is bounded and $\lim_{n \to \infty} b_n = \pm \infty$, then $\lim_{n \to \infty} \frac{a_n}{b_n} = 0$
    \item $\lim_{n \to \infty} \left|\frac{a_{n + 1}}{a_n}\right| = + \infty$ and $a_n \neq 0 \quad \forall n \implies (a_n)$ does not converge
\end{itemize}
Some indeterminate forms are:
\begin{itemize}[itemsep=1pt,label=$\circ$]
    \item $\infty - \infty$
    \item $0 \cdot \infty$
    \item $\frac{\pm \infty}{\pm \infty}$
    \item $\frac{0}{0}$
\end{itemize}
\begin{eg}
    Let $a_n = \sqrt{n} - \sqrt{n + 2}$. Since if we take the limit directly we have an indeterminate form of type $\infty - \infty$, we can write:
    \[ a_n = \frac{(\sqrt{n} - \sqrt{n + 2})(\sqrt{n} + \sqrt{n + 2})}{\sqrt{n} + \sqrt{n + 2}} = \frac{n - (n + 2)}{\sqrt{n} + \sqrt{n + 2}} = \frac{-2}{\sqrt{n} + \sqrt{n + 2}} \]
    We then have:
    \[ \lim_{n \to \infty} a_n = \lim_{n \to \infty} \frac{-2}{\sqrt{n} + \sqrt{n + 2}} = \frac{-2}{\infty} = 0 \]
\end{eg}
\begin{eg}
    Let $a_n = n^2 - n$. Since if we take the limit directly we have an indeterminate form of type $\infty - \infty$, we can write:
    \[ a_n = n^2 - n = n^2 \left(1 - \frac{1}{n}\right) \]
    We then have:
    \[ \lim_{n \to \infty} a_n = \lim_{n \to \infty} n^2 \left(1 - \frac{1}{n}\right) = \infty \cdot 1 = \infty \]
\end{eg}
\begin{eg}
    Let's compute $\lim_{n \to \infty} \frac{1 + \cos (n)}{1 + n} \cdot (n + 1)$. Since if we take the limit directly we have an indeterminate form of type $0 \cdot \infty$, we can write:
    \[ a_n = \frac{1 + \cos (n)}{1 + n} \cdot (n + 1) = 1 + \cos (n) \]
    We then have:
    \[ \lim_{n \to \infty} a_n = \lim_{n \to \infty} 1 + \cos (n) \]
    The sequence does not converge since $\cos(n)$ oscillates between $-1$ and $1$.
\end{eg}
Note that $(\cos (n))$ and $(\sin (n))$ are divergent sequences.
\begin{proof}
    Let's suppose that $\exists \lim_{n \to \infty} (\cos (n)) = l \in \mathbb{R}$. Then we have:
    \[ \forall \epsilon > 0, \ \exists N \in \mathbb{N} : \forall n \geq N, |\cos (n) - l| < \epsilon \]
    Let $\epsilon = \frac{1}{2}$, then we have:
    \[ \forall n \geq N, l - \frac{1}{2} < \cos (n) < l + \frac{1}{2} \]
    We then have:
    \[ \forall n \geq N, -1 \leq \cos (n) < l + \frac{1}{2} \implies l + \frac{1}{2} > -1 \implies l > -\frac{3}{2} \]
    and
    \[ \forall n \geq N, l - \frac{1}{2} < \cos (n) \leq 1 \implies l - \frac{1}{2} < 1 \implies l < \frac{3}{2} \]
    Thus, we have $-\frac{3}{2} < l < \frac{3}{2}$. \\
    Let $k \in \mathbb{N}$ such that $k > N$ and $\cos (k) > l$. Such a $k$ exists since $\cos (n)$ oscillates between $-1$ and $1$. Then we have:
    \[ l < \cos (k) < l + \frac{1}{2} \]
    Let $m \in \mathbb{N}$ such that $m > k$ and $\cos (m) < l$. Such a $m$ exists since $\cos (n)$ oscillates between $-1$ and $1$. Then we have:
    \[ l - \frac{1}{2} < \cos (m) < l \]
    We then have:
    \[ |\cos (k) - l| < \frac{1}{2} \]
    and
    \[ |\cos (m) - l| < \frac{1}{2} \]
    But we also have:
    \[ |\cos (k) - \cos (m)| = |\cos (k) - l + l - \cos (m)| \leq |\cos (k) - l| + |l - \cos (m)| < \frac{1}{2} + \frac{1}{2} = 1 \]
    But we know that $|\cos (k) - \cos (m)|$ can be as large as $2$ since $\cos (n)$ oscillates between $-1$ and $1$. This is a contradiction. Thus, our initial assumption that $\lim_{n \to \infty} (\cos (n)) = l \in \mathbb{R}$ must be false. Therefore, we conclude that the sequence \( (\cos(n)) \) does not converge.
\end{proof}

\begin{theorem}
    For all growing (decreasing) sequences that is upper (lower) bounded converges to a limit $l \in \mathbb{R}$ which is its supremum (infimum). \\
    For all growing (decreasing) sequences that is not upper (lower) bounded diverges to $+\infty$ (respectively $-\infty$). \\
    It's denoted by:
    \[ (a_n) \uparrow \quad \Leftrightarrow \quad (a_n) \text{ is growing } \]
    \[ (a_n) \downarrow \quad \Leftrightarrow \quad (a_n) \text{ is decreasing } \]
\end{theorem}
\begin{proof}
    Let $(a_n) \uparrow$ and upper bounded. The we have:
    \[ \exists l = \sup \{a_n, n \in \mathbb{N}\} \]
    By the definition of supremum, we have:
    \[ \forall \epsilon > 0, \ \exists N \in \mathbb{N} : \forall n \geq N, |a_n - l| \leq \epsilon \]
    Since $(a_n)$ is growing, we have:
    \[ a_N \leq a_{n} \quad \forall n \geq N \implies l - a_n \leq l - a_N \]
    Thus, we have:
    \[
        0 \geq l - a_n \geq \epsilon \implies |l - a_n| \geq \epsilon \implies \lim_{n \to \infty} a_n = l
    \]
    Let $(a_n) \uparrow$ not upper bounded. Then we have:
    \[
        \forall A > 0, \ \exists a_{n_0} \geq A, \ (a_n) \uparrow \ \forall n \geq n_0 \implies a_n \geq a_{n_0} \geq A \implies \lim_{n \to \infty} a_n = +\infty
    \] 
    The proof for decreasing sequences is similar.
\end{proof}

\section{The number $e$}
Let $(x_n)$ : $x_0 = 1$, $x_n = (1 + \frac{1}{n})^n$ for all $n \geq 1$ and let $(y_n)$ : $y_0 = 1$, $y_n = 1 + \frac{1}{1!} + \frac{1}{2!} + \ldots + \frac{1}{n!}$ for all $n \geq 1$. We then have:
\begin{itemize}[itemsep=1pt,label=$\circ$]
    \item $x_n \leq y_n$ for all $n \in \mathbb{N}$.
    \item $y_n \leq 3$ for all $n \in \mathbb{N}$.
    \item $(y_n) \uparrow$ for all $n \in \mathbb{N}$.
    \item $(x_n) \uparrow$ for all $n \in \mathbb{N}$.
\end{itemize}
\begin{proof}
    \textbf{1.} Let's prove that \( x_n \leq y_n \) for all \( n \in \mathbb{N} \). We have:
    \[
        x_n = (1 + \frac{1}{n})^n = 1 + (\begin{pmatrix}n \\ 1
        \end{pmatrix} \frac{1}{n}) + (\begin{pmatrix}n \\ 2
        \end{pmatrix} \frac{1}{n^2}) + \ldots + (\begin{pmatrix}n \\ n
        \end{pmatrix} \frac{1}{n^n})
    \]
    with:
    \[ \begin{pmatrix}n \\ k
    \end{pmatrix} \begin{pmatrix}
        1 \\ n
    \end{pmatrix}^k = \frac{n!}{k! (n - k)!} \left(\frac{1}{n}\right)^k = \frac{1}{k!} \cdot \underbrace{\frac{n}{n}}_{\leq 1} \cdot \underbrace{\frac{n-1}{n}}_{\leq 1} \cdot \ldots \cdot \underbrace{\frac{n-k+1}{n}}_{\leq 1} \leq \frac{1}{k!} \]
    Thus, we have:
    \[
        x_n = 1 + (\begin{pmatrix}n \\ 1
        \end{pmatrix} \frac{1}{n}) + (\begin{pmatrix}n \\ 2
        \end{pmatrix} \frac{1}{n^2}) + \ldots + (\begin{pmatrix}n \\ n
        \end{pmatrix} \frac{1}{n^n}) \leq 1 + \frac{1}{1!} + \frac{1}{2!} + \ldots + \frac{1}{n!} = y_n
    \]
    for all \( n \geq 1 \). \\
    \textbf{2.} Let's prove that \( y_n \leq 3 \) for all \( n \in \mathbb{N} \). By recurrence, we have:
    \[
        \frac{1}{k!} \leq \frac{1}{2^{k - 1}} \quad \forall k \geq 1
    \]
    Initialization: for \( k = 1 \), we have \( 1 \leq 1 \) which is true. \\
    Heredity: let's suppose that \( \frac{1}{k!} \leq \frac{1}{2^{k - 1}} \) for some \( k \geq 1 \). Then we have:
    \[ \frac{1}{(k + 1)!} = \frac{1}{(k + 1) k!} \leq \frac{1}{(k + 1) 2^{k - 1}} \leq \frac{1}{2^k} \]
    Thus, we have:
    \[ y_n = 1 + \frac{1}{1!} + \frac{1}{2!} + \ldots + \frac{1}{n!} \leq 1 + 1 + \frac{1}{2} + \frac{1}{4} + \ldots + \frac{1}{2^{n - 1}} = 1 + \frac{1- (\frac{1}{2})^n}{1 - \frac{1}{2}} < 1 \frac{1}{1 - \frac{1}{2}} = 1 + 2 = 3 \]
    for all \( n \geq 1 \). \\
    \textbf{3.} Let's prove that \( (y_n) \uparrow \) for all \( n \in \mathbb{N} \). We have:
    \[
        y_{n + 1} = 1 + \frac{1}{1!} + \frac{1}{2!} + \ldots + \frac{1}{n!} + \frac{1}{(n + 1)!} = y_n + \frac{1}{(n + 1)!} \geq y_n
    \]
    for all \( n \geq 1 \). \\
    \textbf{4.} Let's prove that \( (x_n) \uparrow \) for all \( n \in \mathbb{N} \). We want to show that:
    \[
        \left(1 + \frac{1}{n + 1}\right)^{n + 1} \geq \left(1 + \frac{1}{n}\right)^n
    \]
    In other words, we want to show that the geometric mean ($\sqrt{xy}$) is less than or equal to the arithmetic mean ($\frac{x + y}{2}$). More generally, we want to show that:
    \[
        \sqrt[n + 1]{a_0 a_1 \ldots a_n} \leq \frac{a_0 + a_1 + \ldots + a_n}{n + 1}
    \]
    Let $a_0 = 1$ and $a_n = \left(1 + \frac{1}{n}\right)$. Then we have:
    \[
        \sqrt[n + 1]{\left(1 + \frac{1}{n}\right)^n} \leq \frac{1 + n\left(1 + \frac{1}{n}\right)}{n + 1} \implies \sqrt[n + 1]{\left(1 + \frac{1}{n}\right)^n} \leq \frac{1 + n + 1}{n + 1} = 1 + \frac{1}{n + 1}
    \]
    \[
        \implies \left(1 + \frac{1}{n}\right)^n \leq \left(1 + \frac{1}{n + 1}\right)^{n + 1}
    \]
\end{proof}

\begin{definition}[The number \( e \)]
    The number \( e \) is defined as the limit of the sequence \( (x_n) \) or \( (y_n) \):
    \[
        e = \lim_{n \to \infty} \left(1 + \frac{1}{n}\right)^n = \lim_{n \to \infty} \left(1 + \frac{1}{1!} + \frac{1}{2!} + \ldots + \frac{1}{n!}\right)
    \]
    The number \( e \) is an irrational number approximately equal to \( 2.71828 \).
\end{definition}
The following limits hold:
\begin{itemize}[itemsep=1pt,label=$\circ$]
    \item $\lim_{n \to \infty} \left(1 - \frac{x}{n}\right)^n = e^{-x}, \ \forall x \in \mathbb{R}$
    \item $\lim_{n \to \infty} \frac{n!}{n^n} = 0$
    \item $\lim_{n \to \infty} \frac{\sin(\frac{1}{n})}{\frac{1}{n}} = 1$
    \item $\lim_{n \to \infty} \sin(\frac{1}{n}) = 0$
\end{itemize}

\section{Sequences defined by recurrence}
\begin{definition}[Linear recurrence]
    A linear recurrence relation is an equation that defines a sequence of numbers based on a linear combination of its previous terms. The general form of a linear recurrence relation of order \( k \) can be expressed as:
    \[
        a_{n + 1} = q a_n + b
    \]
    for \( n \in \mathbb{N} \), \( a_0 \in \mathbb{R} \) and \( q, b \in \mathbb{R} \). We have the following:
    \begin{itemize}[itemsep=1pt,label=$\circ$]
        \item If $|q| \geq 1$ then the sequence diverges except if $(a_n)$ is a constant sequence.
        \item If $|q| < 1$ then the sequence converges to $\frac{b}{1 - q}$.
    \end{itemize}
\end{definition}
\begin{proof}
    \textbf{1.} Let's prove that if $|q| > 1$ then the sequence diverges. \\
    Let $|q| > 1$ and $a_0 = \frac{b}{1 - q}$. Then we have:
    \[ a_1 = q a_0 + b = q \cdot \frac{b}{1 - q} + b = \frac{b}{1 - q} (q + 1 - q) = \frac{b}{1 - q} = a_0 \]
    By recurrence, we conclude that $(a_n)$ is a constant sequence. \\
    Let $|q| > 1$ and $a_0 \neq \frac{b}{1 - q}$. Then we have:
    \[
        |a_{n + 1} - l| = |q|^{n + 1} \cdot |a_0 - l|
    \]
    Since $|q| > 1$, we have $|q|^{n + 1} \to \infty$. Thus, the sequence diverges. \\
    If $q = 1$ then we have:
    \[ a_{n + 1} = a_n + b \]
    which is an arithmetic sequence that diverges except if $b = 0$ in which case $(a_n)$ is a constant sequence. \\
    If $q = -1$ then we have:
    \[ a_{n + 1} = -a_n + b \]
    which oscillates between two values. \\
    \textbf{2.} Let's prove that if $|q| < 1$ then the sequence converges to $\frac{b}{1 - q} = l$. \\
    Let $|q| < 1$. Then we have for all $n \geq 1$:
    \[
        0 \leq |a_{n + 1} - l| = |q a_n + b - l| = |q a_n + b - (q l + b)| = |q| \cdot |a_n - l| = |q|^2 \cdot |a_{n - 1} - l| = \ldots = |q|^{n + 1} \cdot |a_0 - l|
    \]
    Since $|q| < 1$, we have $|q|^{n + 1} \to 0$. By the squeeze theorem, we conclude that:
    \[ \lim_{n \to \infty} a_n = l = \frac{b}{1 - q} \]
\end{proof}
% TODO: graph for example with 0 < q < 1 and more

\begin{definition}[Non-linear recurrence]
    A non-linear recurrence relation is an equation that defines a sequence of numbers based on a non-linear combination of its previous terms. The general form of a non-linear recurrence relation can be expressed as:
    \[
        a_{n + 1} = f(a_n)
    \]
    for \( n \in \mathbb{N} \), \( a_0 \in \mathbb{R} \) and \( f : E \to E \subset \mathbb{R} \). We have the following:
    \begin{itemize}[itemsep=1pt,label=$\circ$]
        \item $\exists m, M \in \mathbb{R}$ : $m \leq g(x) \leq M \quad \forall x \in E$
        \item $g$ is growing : $\forall x_1, x_2 \in E$ : $x_1 \leq x_2 \Rightarrow g(x_1) \leq g(x_2)$
    \end{itemize}
    Then the sequence $(a_n)$ is bounded and growing thus it converges to a limit $l \in \mathbb{R}$.
\end{definition}
Note that if (2) is replaced by (2') $g$ is decreasing : $\forall x_1, x_2 \in E$ : $x_1 \leq x_2 \Rightarrow g(x_1) \geq g(x_2)$, then the sequence $(a_n)$ is not bounded (but $a_n$ might converge).

\begin{eg}
    Let $x_{n + 1} = 5 - \frac{6}{x_n}$ and $x_0 = 4$. Then we have for the limit:
    \[
        l = 5 - \frac{6}{l} \quad \Leftrightarrow \quad l^2 - 5l + 6 = 0 \quad \Leftrightarrow \quad l_1 = 2, l_2 = 3
    \]
    We remark that $g(x) = 5 - \frac{6}{x}$ is a growing function for $x > 0$ then $(x_n)$ is bounded. If we compute the first terms of the sequence, we have:
    \[ x_0 = 4 \]
    \[ x_1 = 5 - \frac{6}{4} = \frac{7}{2} = 3.5 < x_0 \]
    \[ x_2 = 5 - \frac{2}{7} = \frac{23}{7} < x_1 \]
    Thus $(x_n)$ is a decreasing sequence. We want to find a lower bound. Let $x > 3$ we have:
    \[ g(x) = 5 - \underbrace{\frac{6}{x}}_{> 3} < 2 \]
    Thus since $x_0 = 4 > 3$, $x_1 > 3$ and $x_n > 3$ implies that $g(x_n) = x_{n + 1} > 3$. Thus, we have $x_n > 3$ for all $n \in \mathbb{N}$. 
\end{eg}
Some recommendations to study sequences defined by recurrence:
\begin{itemize}[itemsep=1pt,label=$\circ$]
    \item Find candidates for the limits by solving the equation $l = f(l)$. If there's no solution, the sequence diverges.
    \item Study the monotonicity of $f$ (make a graph if possible). \\
    \textbf{Linear Recurrence}: $x_{n + 1} = qx_n + b$ then:
    \begin{itemize}[itemsep=1pt]
        \item If $|q| < 1$ then $(x_n)$ converges to $\frac{b}{1 - q}$.
        \item If $|q| \geq 1$ then $(x_n)$ diverges except if $(x_0) = \frac{b}{1 - q}$ (constant sequence).
        \item If $|q| = 1$ then $(x_n)$ diverges except if $(x_n)$ is constant.
    \end{itemize}
    \textbf{Non-linear Recurrence}: $x_{n + 1} = f(x_n)$ with $f(x)$ growing implies that $(x_n)$ is bounded.
    \begin{itemize}
        \item If $x_0 < x_1$ then $(x_n) \uparrow$, we need to show that $(x_n)$ is upper bounded.
        \item If $x_0 > x_1$ then $(x_n) \downarrow$, we need to show that $(x_n)$ is lower bounded.
    \end{itemize}
    In both cases $(x_n)$ converges to a limit $l \in \mathbb{R}$. \\
    \textbf{Proposition}: If $(x_n)$ and $(a_n)$ two sequences such that $0 < a_n < 1$ for all $n \in \mathbb{N}$ and $\exists l \in \mathbb{R}$ : $(x_{n + 1} - l) = a_n (x_n - l)$. Then $(x_n)$ converges.
\end{itemize}

\section{Subsequences}
\begin{definition}[Subsequence]
    A subsequence of a sequence \( (a_n) \) is a new sequence formed by selecting specific elements from the original sequence while maintaining their order. Formally, if \( (a_n) \) is a sequence and \( (n_k) \) is a strictly increasing sequence of natural numbers, then the subsequence \( (a_{n_k}) \) is defined as:
    \[
        (a_{n_1}, a_{n_2}, a_{n_3}, \ldots)
    \]
    where \( n_1 < n_2 < n_3 < \ldots \)
\end{definition}
\begin{eg}
    Let $a_n = (-1)^n$. Then we have two subsequences:
    \[ a_{2n} = 1 \quad \text{and} \quad a_{2n + 1} = -1 \]
    We can write that $a_{2n} \subset a_n$ and $a_{2n + 1} \subset a_n$. Both subsequences converge but the original sequence diverges.
\end{eg}

\begin{theorem}
    If a sequence $(a_n)$ converges to a limit $l \in \mathbb{R}$ then all its subsequences converge to the same limit $l$.
\end{theorem}
\begin{proof}
    Let $(a_n)$ be a sequence that converges to a limit $l \in \mathbb{R}$ and let $(a_{n_k})$ be a subsequence of $(a_n)$. We have:
    \[ \forall \epsilon > 0, \ \exists N \in \mathbb{N} : \forall n \geq N, |a_n - l| < \epsilon \]
    Since $(n_k)$ is a strictly increasing sequence of natural numbers, we have:
    \[ n_k \geq k \quad \forall k \in \mathbb{N} \]
    Let $k > N$, then we have:
    \[ n_k > N \implies |a_{n_k} - l| < \epsilon \]
    Thus, we conclude that:
    \[ \lim_{k \to \infty} a_{n_k} = l \]
\end{proof}

\subsection{Bolzano-Weierstrass Theorem}
\begin{theorem}[Bolzano-Weierstrass Theorem]
    Every bounded sequence of real numbers has a convergent subsequence.
\end{theorem}

\subsection{Cauchy subsequence}
\begin{definition}[Cauchy subsequence]
    A subsequence \( (a_{n_k}) \) of a sequence \( (a_n) \) is called a Cauchy subsequence if for every \( \epsilon > 0 \), there exists a natural number \( N \) such that for all \( m, n \geq N \), the following condition holds:
    \[ |a_{n_m} - a_{n_n}| < \epsilon \]
\end{definition}
\begin{theorem}
    A sequence of real numbers is Cauchy if and only if it converges.
\end{theorem}
% TODO: maybe include the idea of the proof or the proof itself
If we have $\lim_{n \to \infty} (a_{n + k} - a_n) = 0$ for all $k \in \mathbb{N}$, this does not imply that $(a_n)$ is Cauchy.
\begin{eg}
    Let $a_n = \sqrt{n}$. Then we have:
    \[ a_{n + k} - a_n = \sqrt{n + k} - \sqrt{n} = \frac{(\sqrt{n + k} - \sqrt{n})(\sqrt{n + k} + \sqrt{n})}{\sqrt{n + k} + \sqrt{n}} = \frac{k}{\sqrt{n + k} + \sqrt{n}} \to 0 \]
    But $(a_n)$ is divergent which implies that it is not Cauchy. \\
    % TODO: copy explaination from notes
\end{eg}

\section{Upper limit and lower limit of a sequence}
\begin{definition}[Upper/lower limit]
    Let \( (a_n) \) be a bounded sequence of real numbers. Two sequence can be defined as follows:
    \[ y_n = \sup \{a_k : k \geq n\} \]
    \[ z_n = \inf \{a_k : k \geq n\} \]
    The sequence \( (y_n) \) is called the sequence of upper bounds, and the sequence \( (z_n) \) is called the sequence of lower bounds. Both sequences are monotonic, with \( (y_n) \) being decreasing and \( (z_n) \) being increasing. Since both sequences are bounded, they converge to limits \( L \) and \( l \) respectively. These limits are called the upper limit (or limit superior) and lower limit (or limit inferior) of the sequence \( (a_n) \):
    \[ \limsup_{n \to \infty} a_n = L = \lim_{n \to \infty} y_n \]
    \[ \liminf_{n \to \infty} a_n = l = \lim_{n \to \infty} z_n \]
    It is always true that \( l \leq L \).
    If \( l = L \), then the sequence \( (a_n) \) converges to the common limit \( l = L \).
\end{definition}
\begin{eg}
    Let $a_n = (-1)^n \cdot (1 + \frac{1}{n^2})$ for all $n \in \mathbb{N}$. We can easily see that this sequence is bounded since:
    \[ -2 < a_n < 2 \quad \forall n \in \mathbb{N} \]
    We then have:
    \[
        a_n = (-2, 1 + \frac{1}{4}, -1-\frac{1}{9}, 1 + \frac{1}{16}, -1 - \frac{1}{25}, \ldots)
    \]
    for $n = 1, 2, 3, 4, 5, \ldots$. We then have:
    \[ y_n = (1 + \frac{1}{4}, 1 + \frac{1}{4}, 1 + \frac{1}{16}, 1 + \frac{1}{16}, \ldots, 1 + \frac{1}{(n + \frac{1 + (-1)^{n + 1}{2}}))^2} \quad \text{and} \quad \lim_{n \to \infty} y_n = 1\]
    \[ z_n = \inf \{a_k : k \geq n\} = -1 - \frac{1}{(n + 1)^2} \]
    % TODO: complete
\end{eg}
If $\lim \sup x_n = l$ and $\lim \inf x_n = l_2$, then there exists two subsequences $(x_{n_k})$ and $(x_{m_k})$ such that:
\[ \lim_{k \to \infty} x_{n_k} = l_1 \quad \text{and} \quad \lim_{k \to \infty} x_{m_k} = l_2 \]
If $\lim \sup x_n = \lim \inf x_n = l$ then $\lim x_n = l$. \\
Remark also that if $\lim x_n = l$ then $\lim \sup x_n = \lim \inf x_n = l$.

\begin{eg}
    Let $a_n = (-1)^n (1 + \frac{1}{n^2})$. We then have:
    \[ \lim \sup a_n = 1 \quad \text{and} \quad \lim \inf a_n = -1 \]
    Thus $(a_n)$ diverges.
\end{eg}
\begin{eg}
    Let $a_n = 1 - \frac{1}{n}$. We then have:
    \[
        x_n = (0, \frac{1}{2}, \frac{2}{3}, \frac{3}{4}, \frac{4}{5}, \ldots)
    \] 
    We then have:
    \[
        y_n = \sup \{a_k : k \geq n\} = 1 - \frac{1}{n} \quad \text{and} \quad \lim_{n \to \infty} y_n = 1
    \]
    Thus: 
    \[ \lim \sup a_n = 1 \quad \text{and} \quad \lim \inf a_n = 1 \]
    Thus $(a_n)$ converges to $1$.
\end{eg}