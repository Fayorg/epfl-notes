\chapter{Sequences of real numbers}

\begin{definition}[Sequences]
    A sequence is a function whose domain is the set of natural numbers. In other words, a sequence is an ordered list of numbers, typically denoted as \( (a_n)_{n \in \mathbb{N}} \), where each \( a_n \) is a real number.
\end{definition}
We often denote a sequence by \( (a_n) \), $(a_n)_{n \geq 0} = \{a_0, a_1, a_2, \ldots\}$ or simply \( a_n \), where \( n \) represents the position of the term in the sequence.

\begin{eg}
    Consider the sequences defined as:
    \begin{itemize}[itemsep=1pt,label=$\circ$]
        \item \( a_n = n^2 \) (the sequence of perfect squares)
        \item \( b_n = \frac{1}{n} \) (the sequence of reciprocals)
        \item \( c_n = (-1)^n \) (the alternating sequence)
        \item \( f_0 = 0, f_1 = 1, f_{n + 2} = f_{n + 1} + f_n \) (the Fibonacci sequence)
        \item $a_n = a \cdot n + b $ (an arithmetic sequence with common difference \( a \) and initial term \( b \))
        \item $a_n = a \cdot r^n$ (a geometric sequence with common ratio \( r \) and initial term \( a \))
    \end{itemize}
\end{eg}

\section{Properties of sequences}
\begin{definition}[Bounded sequences]
    A sequence \( (a_n) \) is said to be bounded above if there exists a real number \( M \) such that \( a_n \leq M \) for all \( n \).\\ 
    It is bounded below if there exists a real number \( m \) such that \( a_n \geq m \) for all \( n \). \\
    If a sequence is both bounded above and bounded below, it is called a bounded sequence.
\end{definition}

\begin{definition}[Absolute value]
    The absolute value of a real number \( x \), denoted by \( |x| \), is defined as:
    \[
    |x| = 
    \begin{cases} 
    x & \text{if } x \geq 0 \\ 
    -x & \text{if } x < 0 
    \end{cases}
    \]
\end{definition}
Let be \( (a_n) \) a sequence. We say that \( (a_n) \) is bounded if there exists \( M > 0 \) such that \( |a_n| \leq M \) for all \( n \).
\begin{definition}[Decreasing and increasing sequences]
    A sequence \( (a_n) \) is called decreasing if \( a_n \geq a_{n+1} \) for all \( n \). It is called increasing if \( a_n \leq a_{n+1} \) for all \( n \).  
\end{definition}
\begin{definition}[Monotonic sequences]
    A sequence \( (a_n) \) is called monotonic if it is either non-decreasing or non-increasing.
\end{definition}

\begin{eg}
    Let's consider the following sequences:
    \begin{itemize}[itemsep=1pt,label=$\circ$]
        \item $a_n = n$ is increasing ($a_{n + 1} - a_n = 1 > 0$) and unbounded ($\mathbb{N}$ is not bounded above).
        \item $a_n = \frac{1}{n + 1}$ is decreasing ($a_{n + 1} - a_n = -\frac{1}{(n + 1)(n + 2)} < 0$) and bounded (by $1$ and $0$).
        \item $a_n = (-1)^n$ is neither increasing nor decreasing, but it is bounded (by $1$ and $-1$).
        \item $a_n = a \cdot n + b$ is increasing if \( a > 0 \), decreasing if \( a < 0 \), and constant if \( a = 0 \) ($a_{n + 1} - a_n = a(n + 1) + b - a \cdot n - b = a$). It is unbounded unless \( a = 0 \) (if $a > 0$, $\forall S > 0, \forall b \in \mathbb{R}, \exists n \in \mathbb{N}: a_n > S - b \ (Archimedean) \iff an + b > S$).
        \item $f_0 = 0, f_1 = 1, f_{n + 2} = f_{n + 1} + f_n$ is increasing ($f_{n + 2} - f_{n + 1} = f_n \geq 0$) and unbounded (we can prove it by induction).
    \end{itemize}
\end{eg}

\section{Recurrence relations}
\begin{definition}[Recurrence relation]
    A recurrence relation is an equation that defines each term of a sequence in terms of one or more previous terms. It typically consists of:
    \begin{itemize}[itemsep=1pt,label=$\circ$]
        \item An initial condition or base case that specifies the value(s) of the first term(s) of the sequence.
        \item A formula that relates each term to one or more preceding terms.
    \end{itemize}
\end{definition}
% TODO: complete here
\begin{eg}
    The Fibonacci sequence is defined by the recurrence relation:
    \begin{itemize}[itemsep=1pt,label=$\circ$]
        \item Initial conditions: $(f_1)^2 - f_2 \cdot f_0 = 1^2 - 1 \cdot 0 = 1$
        \item Recurrence relation: $(f_{n + 1})^2 - f_{n + 2} \cdot f_n = f_{n+1}(f_{n -1} + f_n) - f_n (f_n + f_{n + 1}) = f_{n + 1} \cdot f_{n -1} = f_{n + 1} \cdot f_{n - 1} - (f_n)^2 = -((f_n)^2)$
    \end{itemize}
\end{eg}

\begin{eg}
    Let's find the sum of the first $n$ odd natural numbers.
    \begin{itemize}[itemsep=1pt,label=$\circ$]
        \item $S_1 = 1$
        \item $S_2 = 1 + 3 = 4$
        \item $S_3 = 1 + 3 + 5 = 9$
        \item $S_4 = 1 + 3 + 5 + 7 = 16$
    \end{itemize}
    Let hypothesize that $S_n = \sum_{k = 1}^{n} (2k -1) = n^2$. We can prove this by induction:
    \begin{itemize}[itemsep=1pt,label=$\circ$]
        \item Base case: $n = 1$, $S_1 = 1^2 = 1$ (true)
        \item Inductive step: Assume $S_n = n^2$ for some $n \geq 1$. We need to show that $S_{n + 1} = (n + 1)^2$.
        \[
        S_{n + 1} = S_n + (2(n + 1) - 1) = n^2 + (2n + 1) = n^2 + 2n + 1 = (n + 1)^2
        \]
    \end{itemize}
    Thus, by the principle of mathematical induction, the formula holds for all natural numbers \( n \).
\end{eg}

\section{Limits of sequences}
\begin{definition}[Limit of a sequence]
    A sequence \( (a_n) \) is said to converge to a limit \( L \) if for every \( \epsilon > 0 \), there exists a natural number \( N \) such that for all \( n \geq N \), the absolute difference between \( a_n \) and \( L \) is less than \( \epsilon \). In mathematical notation,
    \[\lim_{n \to \infty} a_n = L \iff \forall \epsilon > 0, \exists N \in \mathbb{N}, \forall n \geq N, |a_n - L| < \epsilon.\]

    \begin{center}
        \begin{tikzpicture}[scale=1.5]
            \draw[->] (-0.5, 2) -- (6, 2) node[right] {$n$};
            \draw[->] (0, 1.5) -- (0, 4) node[above] {$a_n$};
            \draw[thick, primary, domain=0:5, samples=100] plot (\x, {3 - exp(-\x)});
            \draw[secondary, dashed] (0, 3) -- (5, 3) node[right] {$L$};
            \draw[secondary, dashed] (1, 2) -- (1, 3.5);
            \draw[secondary, dashed] (0, 2.5) -- (5, 2.5);
            \draw[secondary, dashed] (0, 3.5) -- (5, 3.5);
            \node[secondary] at (2.5, 2.75) {$\epsilon$};
            \node[secondary] at (2.5, 3.25) {$\epsilon$};
            \node[secondary] at (1, 1.7) {$N$};
        \end{tikzpicture}
    \end{center}
\end{definition}
If a sequence does not converge to any limit, it is said to diverge.

\begin{eg}
    Let $a_n = \frac{1}{\sqrt{n + 1}}$. We will show that \( \lim_{n \to \infty} a_n = 0 \). \\
    Let \( \epsilon > 0 \). We need to find \( N \in \mathbb{N} \) such that for all \( n \geq N \), \( |a_n - 0| < \epsilon \). We have:
    \[|a_n - 0| = \left|\frac{1}{\sqrt{n + 1}} - 0\right| = \frac{1}{\sqrt{n + 1}} < \epsilon \]
    This inequality can be solved as follows:
    \[
        \frac{1}{\sqrt{n + 1}} < \epsilon \iff \sqrt{n + 1} > \frac{1}{\epsilon} \iff n + 1 > \frac{1}{\epsilon^2} \iff n > \frac{1}{\epsilon^2} - 1
    \]
    Thefore, we need \( n \) to be greater than \( \frac{1}{\epsilon^2} - 1 \). Since \( n \) must be a natural number, we can choose:
    \[ N = \left\lceil \frac{1}{\epsilon^2} - 1 \right\rceil \]
    where \( \lceil x \rceil \) denotes the smallest integer greater than or equal to \( x \). Thus, for all \( n \geq N \), we have:
    \[ |a_n - 0| = \frac{1}{\sqrt{n + 1}} < \epsilon \]
    Hence, by the definition of the limit of a sequence, we conclude that \( \lim_{n \to \infty} a_n = 0 \).
\end{eg}

\begin{eg}
    Let's study the limit of the sequence defined by \( (a_n) = (-1)^{8^n - 3^n} \). We first need to determine the behavior of the exponent \( 8^n - 3^n \) as \( n \) increases. We want to determine whether \( 8^n - 3^n \) is even or odd for large \( n \).
    \begin{itemize}[itemsep=1pt,label=$\circ$]
        \item For \( n = 0 \): \( 8^0 - 3^0 = 1 - 1 = 0 \) (even)
        \item For \( n = 1 \): \( 8^1 - 3^1 = 8 - 3 = 5 \) (odd)
        \item For \( n = 2 \): \( 8^2 - 3^2 = 64 - 9 = 55 \) (odd)
        \item For \( n = 3 \): \( 8^3 - 3^3 = 512 - 27 = 485 \) (odd)
    \end{itemize}
    We notice that for all \( n \geq 1 \), $8^n$ is even and $3^n$ is odd, thus \( 8^n - 3^n \) is always odd. Therefore, for all \( n \geq 1 \), \( a_n = (-1)^{\text{odd}} = -1 \). Thefore, the sequence \( (a_n) \) is constant and equal to -1 for all \( n \geq 1 \). Hence, we can conclude that:
    \[\lim_{n \to \infty} a_n = -1 \]
\end{eg}

\begin{eg}
    Let $p \in \mathbb{Q}, p > 0$ and $a_0 = 1, a_n = \frac{1}{n^p}$. Let's show that \( \lim_{n \to \infty} a_n = 0 \). \\
    Let \( \epsilon > 0 \). We need to find \( N \in \mathbb{N} \) such that for all \( n \geq N \), \( |a_n - 0| < \epsilon \). We have:
    \[|a_n - 0| = \left|\frac{1}{n^p} - 0\right| = \frac{1}{n^p} < \epsilon \]
    This inequality can be solved as follows:
    \[
        \frac{1}{n^p} < \epsilon \iff n^p > \frac{1}{\epsilon} \iff n > \left(\frac{1}{\epsilon}\right)^{\frac{1}{p}}
    \]
    Thefore, we need \( n \) to be greater than \( \left(\frac{1}{\epsilon}\right)^{\frac{1}{p}} \). Since \( n \) must be a natural number, we can choose:
    \[ N = \left\lceil \left(\frac{1}{\epsilon}\right)^{\frac{1}{p}} \right\rceil \]
    where \( \lceil x \rceil \) denotes the smallest integer greater than or equal to \( x \). Thus, for all \( n \geq N \), we have:
    \[ |a_n - 0| = \frac{1}{n^p} < \epsilon \]
    Hence, by the definition of the limit of a sequence, we conclude that \( \lim_{n \to \infty} a_n = 0 \).
\end{eg}

\begin{definition}[Uniqueness of limits]
    If a sequence \( (a_n) \) converges to a limit \( L \), then this limit is unique. In other words, if \( \lim_{n \to \infty} a_n = L_1 \) and \( \lim_{n \to \infty} a_n = L_2 \), then \( L_1 = L_2 \).
\end{definition}

\begin{definition}[Triangle inequality]
    For any real numbers \( x \) and \( y \), the triangle inequality states that:
    \[ |x + y| \leq |x| + |y| \]
\end{definition}
\begin{proof}
    Let \( x, y \in \mathbb{R} \). We will prove the triangle inequality by considering two casses based on the sign of \( x + y \).
    \begin{itemize}[itemsep=1pt,label=$\circ$]
        \item Case 1: \( x + y \geq 0 \)
        \[ |x + y| = x + y \leq |x| + |y| \]
        \item Case 2: \( x + y < 0 \)
        \[ |x + y| = -(x + y) = -x - y \leq |x| + |y| \]
    \end{itemize}
    In both cases, we have shown that \( |x + y| \leq |x| + |y| \). Therefore, the triangle inequality holds for all real numbers \( x \) and \( y \).
\end{proof}

\begin{eg}
    Let's prove that the sequence $a_n = (-1)^n$ does not converge. Let's suppose, for the sake of contradiction, that the sequence converges to a limit \( L \). According to the definition of convergence, for every \( \epsilon > 0 \), there exists a natural number \( N \) such that for all \( n \geq N \), \( |a_n - L| < \epsilon \).
    Let's choose \( \epsilon = 0.5 \). Then, there exists \( N \in \mathbb{N} \) such that for all \( n \geq N \), \( |a_n - L| < 0.5 \).
    Now, consider the terms \( a_N \) and \( a_{N + 1} \):
    \begin{itemize}[itemsep=1pt,label=$\circ$]
        \item If \( N \) is even, then \( a_N = 1 \) and \( a_{N + 1} = -1 \).
        \item If \( N \) is odd, then \( a_N = -1 \) and \( a_{N + 1} = 1 \).
    \end{itemize}
    In either case, we have:
    \[ |a_N - a_{N + 1}| = |1 - (-1)| = 2 \]
    However, by the triangle inequality, we also have:
    \[ |a_N - a_{N + 1}| \leq |a_N - L| + |L - a_{N + 1}| < 0.5 + 0.5 = 1 \]
    This leads to a contradiction since \( 2 \leq 1 \) is false. Therefore, our initial assumption that the sequence \( (a_n) = (-1)^n \) converges must be incorrect. Hence, the sequence \( (a_n) = (-1)^n \) does not converge.
\end{eg}
Every sequence that converges is bounded. The converse is not true: a bounded sequence does not necessarily converge (e.g., $a_n = (-1)^n$).
\begin{proof}
    Let \( (a_n) \) be a sequence that converges to a limit \( L \). By the definition of convergence, for every \( \epsilon > 0 \), there exists a natural number \( N \) such that for all \( n \geq N \), \( |a_n - L| < \epsilon \).
    Let's choose \( \epsilon = 1 \). Then, there exists \( N \in \mathbb{N} \) such that for all \( n \geq N \), \( |a_n - L| < 1 \). This implies that:
    \[ -1 < a_n - L < 1 \]
    Adding \( L \) to all parts of the inequality, we get:
    \[ L - 1 < a_n < L + 1 \]
    This shows that for all \( n \geq N \), the terms of the sequence \( (a_n) \) are bounded between \( L - 1 \) and \( L + 1 \).
    Now, consider the finite set of terms \( a_0, a_1, a_2, \ldots, a_{N-1} \). Since this is a finite set, it has both a maximum and a minimum value. Let:
    \[ M_1 = \max\{a_0, a_1, a_2, \ldots, a_{N-1}\} \]
    and
    \[ m_1 = \min\{a_0, a_1, a_2, \ldots, a_{N-1}\} \]
    Now we can define:
    \[ M = \max(M_1, L + 1) \]
    and
    \[ m = \min(m_1, L - 1) \]
    Thus, for all \( n < N \), we have \( m_1 \leq a_n \leq M_1 \), and for all \( n \geq N \), we have \( L - 1 < a_n < L + 1 \). Therefore, for all \( n \in \mathbb{N} \), we have:
    \[ m < a_n < M \]
    This shows that the sequence \( (a_n) \) is bounded above by \( M \) and bounded below by \( m \). Hence, the sequence \( (a_n) \) is bounded.
\end{proof}

\subsection{Operations on limits}
\begin{theorem}
    Let \( (a_n) \) and \( (b_n) \) be two sequences that converge to limits \( L_1 \) and \( L_2 \) respectively. Then:
    \begin{itemize}[itemsep=1pt,label=$\circ$]
        \item The sequence \( (a_n \pm b_n) \) converges to \( L_1 \pm L_2 \).
        \item The sequence \( (a_n \cdot b_n) \) converges to \( L_1 \cdot L_2 \).
        \item If \( L_2 \neq 0 \), the sequence \( \left(\frac{a_n}{b_n}\right) \) converges to \( \frac{L_1}{L_2} \).
        \item The sequence \( p \cdot (a_n) \) converges to \( p \cdot L_1 \) for any constant \( p \in \mathbb{R} \).
    \end{itemize}
\end{theorem}
Note that if $(a_n + b_n)$ converges, then either both $(a_n)$ and $(b_n)$ converge, or both diverge. Also if $\lim_{n \to \infty} (a_n - b_n) = 0$, then $\lim_{n \to \infty} a_n = \lim_{n \to \infty} b_n$ or both diverge. \\
When $(a_n \cdot b_n)$ converges, we have:
\begin{itemize}[itemsep=1pt,label=$\circ$]
    \item $a_n \to L_1$ and $b_n \to L_2$. (e.g., $a_n = \frac{1}{n + 1}, b_n = 1$ and $a_n \cdot b_n = \frac{1}{n + 1} \to 0$)
    \item $a_n$ and $b_n$ diverge. (e.g., $a_n = (-1)^n, b_n = (-1)^n + \frac{1}{n + 1}$ and $a_n \cdot b_n = 1 + \frac{(-1)^n}{n + 1}$)
    \item $a_n \to L$ and $b_n$ diverges. (e.g., $a_n = (n + 1), b_n = \frac{1}{(n + 1)^2}$ and $a_n \cdot b_n = \frac{1}{n + 1}$)
\end{itemize}
But if $b_n \to L \neq 0$ then $a_n$ must converge. \\
 
\subsection{Quotients of Polynomial Sequences}
\begin{theorem}
    Let \( (a_n) \) and \( (b_n) \) be sequences defined by polynomials \( P(n) \) and \( Q(n) \) respectively, where:
    \[
    a_n = P(n) = a_k n^k + a_{k-1} n^{k-1} + \ldots + a_1 n + a_0
    \]
    \[
    b_n = Q(n) = b_m n^m + b_{m-1} n^{m-1} + \ldots + b_1 n + b_0
    \]
    with \( a_k, b_m \neq 0 \). The limit of the quotient of these polynomial sequences as \( n \) approaches infinity can be determined based on the degrees of the polynomials \( k \) and \( m \):
    \begin{itemize}[itemsep=1pt,label=$\circ$]
        \item If \( k < m \), then \( \lim_{n \to \infty} \frac{a_n}{b_n} = 0 \).
        \item If \( k = m \), then \( \lim_{n \to \infty} \frac{a_n}{b_n} = \frac{a_k}{b_m} \).
        \item If \( k > m \), then \( \frac{a_n}{b_n} \) diverges.
    \end{itemize}
\end{theorem}
\begin{proof}
    Let \( (a_n) \) and \( (b_n) \) be sequences defined by polynomials \( P(n) \) and \( Q(n) \) respectively, where:
    \[
    a_n = P(n) = a_k n^k + a_{k-1} n^{k-1} + \ldots + a_1 n + a_0
    \]
    \[
    b_n = Q(n) = b_m n^m + b_{m-1} n^{m-1} + \ldots + b_1 n + b_0
    \]
    with \( a_k, b_m \neq 0 \). We analyze the limit of the quotient \( \frac{a_n}{b_n} \) as \( n \) approaches infinity based on the degrees of the polynomials \( k \) and \( m \):
    \[
        \frac{a_n}{b_n} = \frac{a_k n^k + a_{k-1} n^{k-1} + \ldots + a_1 n + a_0}{b_m n^m + b_{m-1} n^{m-1} + \ldots + b_1 n + b_0} = \frac{n^k}{n^m} \cdot \frac{a_k + \frac{a_{k-1}}{n} + \ldots + \frac{a_1}{n^{k-1}} + \frac{a_0}{n^k}}{b_m + \frac{b_{m-1}}{n} + \ldots + \frac{b_1}{n^{m-1}} + \frac{b_0}{n^m}}
    \]
    We easily see that:
    \[
        \lim_{n \to \infty} \frac{a_k + \frac{a_{k-1}}{n} + \ldots + \frac{a_1}{n^{k-1}} + \frac{a_0}{n^k}}{b_m + \frac{b_{m-1}}{n} + \ldots + \frac{b_1}{n^{m-1}} + \frac{b_0}{n^m}} = \frac{a_k}{b_m}
    \]
    Thus we have three cases:
    \begin{itemize}[itemsep=1pt,label=$\circ$]
        \item If \( k < m \), then \( \frac{n^k}{n^m} = \frac{1}{n^{m-k}} \to 0 \) as \( n \to \infty \). Therefore, \( \lim_{n \to \infty} \frac{a_n}{b_n} = 0 \).
        \item If \( k = m \), then \( \frac{n^k}{n^m} = 1 \). Therefore, \( \lim_{n \to \infty} \frac{a_n}{b_n} = \frac{a_k}{b_m} \).
        \item If \( k > m \), then \( \frac{n^k}{n^m} = n^{k-m} \to \infty \) as \( n \to \infty \). Therefore, the quotient \( \frac{a_n}{b_n} \) diverges.
    \end{itemize}
    This completes the proof.
\end{proof}

\begin{eg}
    Let $a_n = \frac{3n^2 + 5n + 1}{2n^2 + n + 10}$. Then we have:
    \[
        \lim_{n \to \infty} a_n = \lim_{n \to \infty} \frac{3n^2 + 5n + 1}{2n^2 + n + 10} = \lim_{n \to \infty} \frac{n^2}{n^2} \cdot \frac{3 + \frac{5}{n} + \frac{1}{n^2}}{2 + \frac{1}{n} + \frac{10}{n^2}} = \frac{3}{2}
    \]
\end{eg}

\subsection{Squeeze theorem}
\begin{theorem}
    Let $(a_n)$ and $(b_n)$ be two sequences such that $(a_n) \to a$ and $(b_n) \to b$. If there exists a $m_0 \in \mathbb{N} : \forall n \geq m_0 \implies a_n \geq b_n$, then \( a \geq b \).
\end{theorem}
\begin{proof}
    Let's prove the theorem by contradiction. Suppose that \( a < b \). Then, we can choose \( \epsilon = \frac{b - a}{2} > 0 \). By the definition of convergence, there exist natural numbers \( N_1 \) and \( N_2 \) such that for all \( n \geq N_1 \), \( |a_n - a| < \epsilon \) and for all \( n \geq N_2 \), \( |b_n - b| < \epsilon \).
    Let \( N = \max(N_1, N_2, m_0) \). Then, for all \( n \geq N \), we have:
    \[
        |a_n - a| < \epsilon \implies a_n < a + \epsilon = a + \frac{b - a}{2} = \frac{a + b}{2}
    \]
    and
    \[
        |b_n - b| < \epsilon \implies b_n > b - \epsilon = b - \frac{b - a}{2} = \frac{a + b}{2}
    \]
    Therefore, for all \( n \geq N \), we have:
    \[
        a_n > \frac{a + b}{2} > b_n
    \]
    This contradicts our initial assumption that there exists \( m_0 \in \mathbb{N} : \forall n \geq m_0, a_n \geq b_n \). Hence, our assumption that \( a < b \) must be false. Therefore, we conclude that \( a \geq b \).
\end{proof}

\begin{theorem}[Squeeze theorem]
    Let \( (a_n) \), \( (b_n) \), and \( (c_n) \) be three sequences such that \( a_n \leq b_n \leq c_n \) for all \( n \) greater than some index \( N \). If \( \lim_{n \to \infty} a_n = L \) and \( \lim_{n \to \infty} c_n = L \), then \( \lim_{n \to \infty} b_n = L \).
\end{theorem}
\begin{proof}
    Let \( (a_n) \), \( (b_n) \), and \( (c_n) \) be three sequences such that \( a_n \leq b_n \leq c_n \) for all \( n \) greater than some index \( N \). We are given that \( \lim_{n \to \infty} a_n = L \) and \( \lim_{n \to \infty} c_n = L \). We want to show that \( \lim_{n \to \infty} b_n = L \).
    By the definition of limits, for every \( \epsilon > 0 \), there exist natural numbers \( N_1 \) and \( N_2 \) such that for all \( n \geq N_1 \), \( |a_n - L| < \epsilon \) and for all \( n \geq N_2 \), \( |c_n - L| < \epsilon \).
    Let \( N' = \max(N, N_1, N_2) \). Then, for all \( n \geq N' \), we have:
    \[
        |a_n - L| < \epsilon \implies L - \epsilon < a_n < L + \epsilon
    \]
    and
    \[
        |c_n - L| < \epsilon \implies L - \epsilon < c_n < L + \epsilon
    \]
    Since \( a_n \leq b_n \leq c_n \), it follows that:
    \[
        L - \epsilon < a_n \leq b_n \leq c_n < L + \epsilon
    \]
    Therefore, for all \( n \geq N' \), we have:
    \[
        |b_n - L| < \epsilon
    \]
    This shows that for every \( \epsilon > 0 \), there exists a natural number \( N' \) such that for all \( n \geq N' \), \( |b_n - L| < \epsilon \). By the definition of limits, we conclude that \( \lim_{n \to \infty} b_n = L \).
\end{proof}

\begin{eg}
    Let's find the limit of the sequences defined by $(a_n) = \frac{5n^2 + 6n \cos (n)}{4n^2 + 3}$ and $(b_n) = \frac{\sqrt{3n^2 + 1}}{5n + 2}$. We will use the squeeze theorem to find these limits. We have:
    \[ -1 \leq \cos(n) \leq 1 \implies 5n^2 - 6n \leq 5n^2 + 6n \cos(n) \leq 5n^2 + 6n \]
    Thus:
    \[ \frac{5n^2 - 6n}{4n^2 + 3} \leq a_n \leq \frac{5n^2 + 6n}{4n^2 + 3} \]
    We can find the limits of the sequences on the left and right:
    \[ \lim_{n \to \infty} \frac{5n^2 - 6n}{4n^2 + 3} = \lim_{n \to \infty} \frac{n^2}{n^2} \cdot \frac{5 - \frac{6}{n}}{4 + \frac{3}{n^2}} = \frac{5}{4} \]
    \[ \lim_{n \to \infty} \frac{5n^2 + 6n}{4n^2 + 3} = \lim_{n \to \infty} \frac{n^2}{n^2} \cdot \frac{5 + \frac{6}{n}}{4 + \frac{3}{n^2}} = \frac{5}{4} \]
    By the squeeze theorem, we conclude that:
    \[ \lim_{n \to \infty} a_n = \frac{5}{4} \]
    Now, let's find the limit of the sequence \( (b_n) = \frac{\sqrt{3n^2 + 1}}{5n + 2} \). We have:
    \[
        \frac{\sqrt{3n^2 + 1}}{5n + 2} = \frac{\sqrt{3n^2} \sqrt{1 + \frac{1}{3n^2}}}{5n + 2} = \frac{n \sqrt{3} \sqrt{1 + \frac{1}{3n^2}}}{5n + 2} = \frac{\sqrt{3} \sqrt{1 + \frac{1}{3n^2}}}{5 + \frac{2}{n}}
    \]
    We can find bounds for \( b_n \):
    \[
        1 \leq \sqrt{1 + \frac{1}{3n^2}} \leq \sqrt{1 + \frac{1}{3n^2} + (\frac{1}{6n^2})^2} = \sqrt{(1 + \frac{1}{6n^2})2} = 1 + \frac{1}{6n^2}
    \]
    Thus:
    \[ \frac{\sqrt{3}}{5 + \frac{2}{n}} \leq b_n \leq \frac{\sqrt{3}(1 + \frac{1}{6n^2})}{5 + \frac{2}{n}} \]
    We can find the limits of the sequences on the left and right:
    \[ \lim_{n \to \infty} \frac{\sqrt{3}}{5 + \frac{2}{n}} = \frac{\sqrt{3}}{5} \]
    \[ \lim_{n \to \infty} \frac{\sqrt{3}(1 + \frac{1}{6n^2})}{5 + \frac{2}{n}} = \frac{\sqrt{3}}{5} \]
    By the squeeze theorem, we conclude that:
    \[ \lim_{n \to \infty} b_n = \frac{\sqrt{3}}{5} \]
\end{eg}
