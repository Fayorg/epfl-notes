\chapter{Function of Real Numbers}

\section{Functions}
\begin{definition}[Real Function]
    A function $f : E \to F$, where $E$ and $F$ are subsets of $\mathbb{R}$, is a rule that assigns to each element $x \in E$ a unique element $f(x) \in F$. The set $E = \text{D}(f)$ is called the domain of the function, and the set $F$ is called the codomain.
\end{definition}

\begin{definition}[Graph of a Function]
    The graph of a function $f : E \to F$ is the set of points in the Cartesian plane defined by:
    \[
        G(f) = \{(x, f(x)) \mid x \in E\}
    \]
\end{definition}
\begin{eg}
    The graph of the function $f(x) = x^2$ is the set of points:
    \[
        G(f) = \{(x, x^2) \mid x \in \mathbb{R}\}
    \]
    Thus:
    \begin{center}
        \begin{tikzpicture}[scale=1.5]
            \draw[->] (-2, 0) -- (2, 0) node[right] {$x$};
            \draw[->] (0, -0.5) -- (0, 2.75) node[above] {$y$};
            \draw[thick, primary, domain=-1.5:1.5, samples=50] plot (\x, {(\x)^2});
        \end{tikzpicture}
    \end{center}
\end{eg}

\subsection{Properties of Functions}
% \begin{definition}[Growing Function]
%     A function $f : E \to \mathbb{R}$ is said to be increasing ($f(x) \uparrow$) on an interval $I \subseteq E$ if for all $x_1, x_2 \in I$ such that $x_1 < x_2$, we have:
%     \[
%         f(x_1) \leq f(x_2)
%     \]
%     If the inequality is strict, i.e., $f(x_1) < f(x_2)$, then $f$ is said to be strictly increasing on $I$.
% \end{definition}

% \begin{definition}[Decreasing Function]
%     A function $f : E \to \mathbb{R}$ is said to be decreasing ($f(x) \downarrow$) on an interval $I \subseteq E$ if for all $x_1, x_2 \in I$ such that $x_1 < x_2$, we have:
%     \[
%         f(x_1) \geq f(x_2)
%     \]
%     If the inequality is strict, i.e., $f(x_1) > f(x_2)$, then $f$ is said to be strictly decreasing on $I$.
% \end{definition}

\begin{definition}[Increasing and Decreasing Functions]
    A function $f : E \to \mathbb{R}$ is said to be increasing on an interval $I \subseteq E$ if for all $x_1, x_2 \in I$ such that $x_1 < x_2$, we have:
    \[
        f(x_1) \leq f(x_2)
    \]
    If the inequality is strict, i.e., $f(x_1) < f(x_2)$, then $f$ is said to be strictly increasing on $I$. \\
    Similarly, a function $f : E \to \mathbb{R}$ is said to be decreasing on an interval $I \subseteq E$ if for all $x_1, x_2 \in I$ such that $x_1 < x_2$, we have:
    \[
        f(x_1) \geq f(x_2)
    \]
    If the inequality is strict, i.e., $f(x_1) > f(x_2)$, then $f$ is said to be strictly decreasing on $I$.
\end{definition}
Note that a function (strictly) increasing or (strictly) decreasing on an interval $I$ is also called (strictly) monotonic on $I$.

\begin{definition}[Even and Odd Functions]
    A function $f : E \to \mathbb{R}$ is said to be even if for all $x \in E$, we have:
    \[
        f(-x) = f(x)
    \]
    A function $f : E \to \mathbb{R}$ is said to be odd if for all $x \in E$, we have:
    \[
        f(-x) = -f(x)
    \]
    % TODO: Add graphs for examples of even and odd functions
\end{definition}
\begin{eg}
    The function $f(x) = x^2$ is even since for all $x \in \mathbb{R}$, we have:
    \[
        f(-x) = (-x)^2 = x^2 = f(x)
    \]
    Graphically, even functions are symmetric with respect to the y-axis.
    \begin{center}
        \begin{tikzpicture}[scale=1.5]
            \draw[->] (-2, 0) -- (2, 0) node[right] {$x$};
            \draw[->, secondary, thick] (0, -0.5) -- (0, 2.75) node[above] {$y$};
            \draw[thick, primary, domain=-1.5:1.5, samples=50] plot (\x, {(\x)^2});
        \end{tikzpicture}
    \end{center}
    The function $g(x) = x^3$ is odd since for all $x \in \mathbb{R}$, we have:
    \[
        g(-x) = (-x)^3 = -x^3 = -g(x)
    \]
    Graphically, odd functions are symmetric with respect to the origin.
    \begin{center}
        \begin{tikzpicture}[scale=1.5]
            \draw[->] (-2, 0) -- (2, 0) node[right] {$x$};
            \draw[->] (0, -2) -- (0, 2) node[above] {$y$};
            \draw[thick, primary, domain=-1.2:1.2, samples=50] plot (\x, {(\x)^3});
            \draw[secondary, thick, domain=-1.7:1.7, samples=2] plot (\x, {\x});
        \end{tikzpicture}
    \end{center}
\end{eg}

\begin{definition}[Periodic Function]
    A function $f : E \to \mathbb{R}$ is said to be periodic with period $T > 0$ if for all $x \in E$ such that $x \pm T \in E$, we have:
    \[
        f(x \pm T) = f(x)
    \]
\end{definition}
\begin{eg}
    The function $f(x) = \sin(x)^2$ is periodic with period $2\pi$, since we have:
    \[
        \sin(x)^2 = \left(\frac{e^{ix} - e^{-ix}}{2i}\right)^2 = \frac{e^{2ix} - 2 + e^{-2ix}}{-4} = \frac{1}{2} - \frac{1}{2}\cos(2x)
    \]
    Since $\cos(x)$ is periodic with period $2\pi$, we have that $\sin(x)^2$ is also periodic with a period of $\pi$ (or $\sin(x)^2$ is $\pi$-periodic).
    Graphically, we have:
    \begin{center}
        \begin{tikzpicture}[scale=1.5]
            \draw[->] (-2, 0) -- (2, 0) node[right] {$x$};
            \draw[->] (0, -0.5) -- (0, 1.5) node[above] {$y$};
            \draw[thick, primary, domain=-1.5:1.5, samples=150] plot (\x, {sin(deg(\x * 4))^2});

            \draw[<->, secondary, thick] (-0.8, -0.1) -- (0, -0.1) node[midway, below] {$\pi$};
            \draw[<->, secondary, thick] (0.8, -0.1) -- (0, -0.1) node[midway, below] {$\pi$};
        \end{tikzpicture}
    \end{center}
\end{eg}
\begin{eg}
    Some functions are periodic but it's impossible to find the smallest period. For example, the function:
    \[
        f(x) = \begin{cases}
            0 & \text{if } x \in \mathbb{Q} \\
            1 & \text{if } x \notin \mathbb{Q}
        \end{cases}
    \]
    In other words, for any rational number $P$:
    \[
        \begin{cases}
            \text{rational} + P = \text{rational} \\
            \text{irrational} + P = \text{irrational}
        \end{cases}
    \]
    Thus, $f(x + P) = f(x)$ for any rational number $P$, making $f$ a periodic function without a smallest period.
\end{eg}

\subsection{Boundedness and Extrema of Functions}
\begin{definition}[Bounded Function]
    A function $f : E \to \mathbb{R}$ is said to be bounded above on a set $A \subseteq E$ if the set $f(A) \subset \mathbb{R}$ is bounded above, i.e., there exists a real number $M$ such that for all $x \in A$, we have:
    \[
        f(x) \leq M
    \]
    Similarly, $f$ is said to be bounded below on $A$ if the set $f(A) \subset \mathbb{R}$ is bounded below, i.e., there exists a real number $m$ such that for all $x \in A$, we have:
    \[
        f(x) \geq m
    \]
    If $f$ is both bounded above and bounded below on $A$, then it is said to be bounded on $A$.
\end{definition}

\begin{definition}[Supremum and Infimum]
    A function $f : E \to \mathbb{R}$ has a supremum (least upper bound) on a set $A \subseteq E$, denoted by $\sup_{x \in A} f(x)$, if the set $f(A) \subset \mathbb{R}$ has a supremum. This means that:
    \[
        \sup_{x \in A} f(x) = \sup \{f(x), x \in A\}
    \]
    Similarly, $f$ has an infimum (greatest lower bound) on $A$, denoted by $\inf_{x \in A} f(x)$, if the set $f(A) \subset \mathbb{R}$ has an infimum. This means that:
    \[
        \inf_{x \in A} f(x) = \inf \{f(x), x \in A\}
    \]
\end{definition}
\begin{eg}
    Let $f: (0,1) \to \mathbb{R}$ defined by $f(x) = x^2 + 3$ be bounded on $A = (0,1)$. Then:
    \[
        \sup_{x \in A} f(x) = \sup \{x^2 + 3, x \in A\} = 4 \notin f(A)
    \]
    Similarly:
    \[
        \inf_{x \in A} f(x) = \inf \{x^2 + 3, x \in A\} = 3 \notin f(A)
    \]
\end{eg}

\begin{definition}[Local Maximum and Minimum]
    A function $f: E \to \mathbb{R}$ has a local maximum at a point $x_0 \in E$ if there exists $\delta > 0$ such that for all $x \in E$ with $|x - x_0| < \delta$, we have:
    \[
        f(x) \leq f(x_0)
    \]
    Similarly, $f$ has a local minimum at a point $x_0 \in E$ if there exists $\delta > 0$ such that for all $x \in E$ with $|x - x_0| < \delta$, we have:
    \[
        f(x) \geq f(x_0)
    \]
\end{definition}

\begin{definition}[Global Maximum and Minimum]
    A function $f: E \to \mathbb{R}$ has a global maximum at a point $x_0 \in E$ if for all $x \in E$, we have:
    \[
        f(x) \leq f(x_0)
    \]
    Similarly, $f$ has a global minimum at a point $x_0 \in E$ if for all $x \in E$, we have:
    \[
        f(x) \geq f(x_0)
    \]
\end{definition}
\begin{eg}
    Let's show the difference between local and global extrema graphically:
    \begin{center}
        \begin{tikzpicture}[scale=1.3]
            \draw[->] (-0.5, 0) -- (3.5, 0) node[right] {$x$};
            \draw[->] (0, -0.5) -- (0, 2.5) node[above] {$y$};
            \draw[thick, primary, domain=0:2.5, samples=100] plot (\x, {-(\x - 1)^2 + 2});

            \filldraw[secondary, thick] (1, 2) circle (1.5pt) node[above right] {Global Max};
            \filldraw[secondary, thick] (0, 1) circle (1.5pt) node[above left] {Local Min};
            \filldraw[secondary, thick] (2.49, -0.2) circle (1.5pt) node[below right] {Global Min};
        \end{tikzpicture}
    \end{center}
\end{eg}
If the $\max_{x \in E} f(x)$ (or $\min_{x \in E} f(x)$) exists, then $f$ is bounded above (or below) on $E$ and $\sup_{x \in E} f(x) = \max_{x \in E} f(x)$ (or $\inf_{x \in E} f(x) = \min_{x \in E} f(x)$). \\
A bounded function on $E$ does not necessarily reach its bounds, i.e., the maximum or minimum may not exist.
\begin{eg}
    Let $f: [0, 1) \to \mathbb{R}$ defined by $f(x) = x^2 + 3$. We clearly see that $f$ is bounded but:
    \[
        \max_{x \in [0, 1)} f(x) = 4 \notin f([0, 1))
    \]
    i.e. $f$ does not reach its upper bound and:
    \[
        \min_{x \in [0, 1)} f(x) = 3 \in f([0, 1))
    \]
    i.e. $f$ reaches its lower bound at $x = 0$:
    \[
        f(0) = 3 = \min_{x \in [0, 1)} f = \inf_{x \in [0, 1)} f
    \]
\end{eg}

\subsection{Types of Functions}
\begin{definition}[Surjectivity]
    A function $f : E \to F$ is said to be surjective (onto) if for every $y \in F$, there exists at least one $x \in E$ such that $f(x) = y$.
\end{definition}
\begin{definition}[Injectivity]
    A function $f : E \to F$ is said to be injective (one-to-one) if for every $x_1, x_2 \in E$, whenever $f(x_1) = f(x_2)$, it follows that $x_1 = x_2$ (i.e. there exists at most one $x \in E$ for each $y \in F$ such that $f(x) = y$).
\end{definition}
If $f: E \to F$ is not injective, it can be made injective by restricting its domain $E$ and if $f$ is not surjective, it can be made surjective by adjusting its codomain $F$.

\begin{definition}[Bijectivity]
    A function $f : E \to F$ is said to be bijective if it is both injective and surjective. In this case, for every $y \in F$, there exists a unique $x \in E$ such that $f(x) = y$.
\end{definition}

\begin{definition}[Inverse Function]
    Let $f : E \to F$ be a bijective function. The inverse function of $f$, denoted by $f^{-1} : F \to E$, is defined by:
    \[
        f^{-1}(y) = x \quad \text{where} \quad f(x) = y
    \]
    for every $y \in F$.
\end{definition}
By convention, for the trigonometric functions, the inverse sine, cosine, and tangent functions are defined on restricted domains to ensure bijectivity:
\begin{itemize}[itemsep=1pt,label=$\circ$]
    \item $\sin : \left[-\frac{\pi}{2}, \frac{\pi}{2}\right] \to [-1, 1]$, reciprocally $\arcsin : [-1, 1] \to \left[-\frac{\pi}{2}, \frac{\pi}{2}\right]$
    \item $\cos : [0, \pi] \to [-1, 1]$, reciprocally $\arccos : [-1, 1] \to [0, \pi]$
    \item $\tan : \left(-\frac{\pi}{2}, \frac{\pi}{2}\right) \to \mathbb{R}$, reciprocally $\arctan : \mathbb{R} \to \left(-\frac{\pi}{2}, \frac{\pi}{2}\right)$
    \item $\cot : (0, \pi) \to \mathbb{R}$, reciprocally $\text{arccot} : \mathbb{R} \to (0, \pi)$
\end{itemize}
\begin{eg}
    Graphically, the inverse function $f^{-1}$ of a bijective function $f$ can be obtained by reflecting the graph of $f$ across the line $y = x$:
    \begin{center}
        \begin{tikzpicture}[scale=1]
            \draw[->] (-0.5, 0) -- (3.5, 0) node[right] {$x$};
            \draw[->] (0, -0.5) -- (0, 3.5) node[above] {$y$};
            \draw[thick, primary, domain=0:2.5, samples=100] plot (\x, {0.5 * (\x)^2}) node[above] {$f$};
            \draw[thick, secondary, domain=0:3, samples=100] plot (\x, {sqrt(2 * \x)}) node[right] {$f^{-1}$};
            \draw[dashed] (0, 0) -- (3.5, 3.5) node[above right] {$y = x$};
        \end{tikzpicture}
    \end{center}
\end{eg}
\begin{eg}
    Let $f = \frac{1}{\cos(x)^2 + 1}$ defined on $\mathbb{R}$. Let's find the biggest interval containing $x= 1$ where $f$ is bijective. \\
    We have:
    \[
        \frac{1}{y + 1} \quad \text{is injective} \quad \forall y \in [0, +\infty)
    \]
    and 
    \[
        \cos(x)^2 \quad \text{is injective} \quad  \forall x \in [0, \frac{\pi}{2}]
    \]
    Since $1 \in [0, \frac{\pi}{2}]$, thus $f$ is injective on the interval $\left[0, \frac{\pi}{2}\right]$. We also have:
    \[
        f(x) = \frac{1}{\cos(x)^2 + 1} \quad x \in \left[0, \frac{\pi}{2}\right]
    \]
    is increasing since $\cos(x)^2$ is decreasing on $\left[0, \frac{\pi}{2}\right]$ and thus:
    \[        
        \inf_{x \in \left[0, \frac{\pi}{2}\right]} f(x) = f\left(\frac{\pi}{2}\right) = \frac{1}{2} \quad \text{and} \quad \sup_{x \in \left[0, \frac{\pi}{2}\right]} f(x) = f(0) = 1
    \]
    Therefore, $f : \left[0, \frac{\pi}{2}\right] \to \left[\frac{1}{2}, 1\right]$ is bijective and its inverse function is given by:
    \[
        f^{-1}(y) = \arccos\left(\sqrt{\frac{1}{y} - 1}\right) \quad y \in \left[\frac{1}{2}, 1\right]
    \]
\end{eg}

\subsection{Composite Functions}
\begin{definition}[Composite Function]
    Let $f : E \to F$ and $g : F \to G$ be two functions. The composite function of $f$ and $g$, denoted by $g \circ f : E \to G$, is defined by:
    \[
        (g \circ f)(x) = g(f(x))
    \]
    for every $x \in E$.
\end{definition}
\begin{eg}
    Let $f : \mathbb{R} \to \mathbb{R}$ be defined by $f(x) = 2x + 3$ and let $g : \mathbb{R} \to \mathbb{R}$ be defined by $g(x) = x^2$. Then, the composite function $g \circ f : \mathbb{R} \to \mathbb{R}$ is given by:
    \[
        (g \circ f)(x) = g(f(x)) = g(2x + 3) = (2x + 3)^2 = 4x^2 + 12x + 9
    \]
    Note that the order of composition matters, as $f \circ g$ would yield a different result:
    \[
        (f \circ g)(x) = f(g(x)) = f(x^2) = 2x^2 + 3
    \]
\end{eg}
\begin{eg}
    Let $f: E \to F$ be a bijective function with inverse $f^{-1} : F \to E$. Then, the composite functions $f \circ f^{-1} : F \to F$ and $f^{-1} \circ f : E \to E$ are given by:
    \[
        (f \circ f^{-1})(y) = f(f^{-1}(y)) = y \quad \forall y \in F
    \]
    and
    \[
        (f^{-1} \circ f)(x) = f^{-1}(f(x)) = x \quad \forall x \in E
    \]
    Thus, composing a function with its inverse yields the identity function on the respective domains.
\end{eg}

\section{Exercices}
This section gathers a selection of exercises related to Chapter \thechapter, taken from weekly assignments, past exams, textbooks, and other sources. The origin of each exercise will be indicated at its beginning.

\begin{exercise}[Quizz of Lecture 12]
    Let $f(x) = 2 \sin(1 - x^2)$ on the biggest interval where it is bijective and containing $x= 1$. Then the set of the definition of $f^{-1}$ and its image is:
    \begin{itemize}[itemsep=1pt,label=$\circ$]
        \item $f^{-1}: \left[-2 \sin(1), 2 \sin(1)\right] \to \left[\sqrt{1 - \frac{\pi}{2}}, \sqrt{1 + \frac{\pi}{2}}\right]$
        \item $f^{-1}: \left[0, 2 \sin(1)\right] \to \left[0, \sqrt{1 + \pi}\right]$
        \item $f^{-1}: \left[-2, 2 \sin(1)\right] \to \left[0, \sqrt{1 + \frac{\pi}{2}}\right]$
        \item $f^{-1}: \left[-2, 2 \sin(1)\right] \to \left[-\sqrt{1 + \frac{\pi}{2}}, \sqrt{1 + \frac{\pi}{2}}\right]$
        \item $f^{-1}: \left[-2 \sin(1), 0\right] \to \left[0, \sqrt{1 + \pi}\right]$
    \end{itemize}
    \Answer
    The correct answer is the 3rd proposition because we have:
    \[
        f(x) = 2 \sin(1 - x^2) \quad \implies \quad - \frac{\pi}{2} + 2k\pi \leq 1 - x^2 \leq \frac{\pi}{2} + 2k\pi
    \]
    for $k = 0$ since $x = 1$ is in the interval. Thus:
    \[
        1 - \frac{\pi}{2} \leq x^2 \leq 1 + \frac{\pi}{2} \quad \iff \quad 0 \leq x^2 \leq 1 + \frac{\pi}{2}
    \]
    since $1 - \frac{\pi}{2} < 0$ and $x^2 \geq 0$. Therefore:
    \[
        0 \leq x \leq \sqrt{1 + \frac{\pi}{2}}
    \]
    Thus the domain of $f(x)$ is $\left[0, \sqrt{1 + \frac{\pi}{2}}\right] = E$ for it to be bijective and containing $x = 1$. Furthermore, we see that $f$ is decreasing on $E$ thus:
    \[
        \inf_{x \in E} f(x) = f\left(\sqrt{1 + \frac{\pi}{2}}\right) = -2 \quad \text{and} \quad \sup_{x \in E} f(x) = f(0) = 2 \sin(1)
    \]
    Therefore, the image of $f$ is $\left[-2, 2 \sin(1)\right]$.
\end{exercise}