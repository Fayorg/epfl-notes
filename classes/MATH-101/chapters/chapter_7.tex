\chapter{Power Series}

\begin{definition}[Convergence Radius]
    Let $\sum_{n=0}^{\infty} a_n (x - x_0)^n$ ($x_0 \in \mathbb{R}, a_k \in \mathbb{R} \ \forall k \in \mathbb{N}$) be a power series. The domain of convergence is given by:
    \[
        D = \{x \in \mathbb{R}: \sum_{n=0}^{\infty} a_n (x - x_0)^n \text{ converges}\}
    \]
    The function $f: D \to \mathbb{R}, f(x) = \sum_{n=0}^{\infty} a_n (x - x_0)^n$ is called the function defined by the power series.
\end{definition}

\begin{eg}
    We previously defined the exponential function as:
    \[
        \exp(x) = \sum_{n=0}^{\infty} \underbrace{\frac{x^n}{n!}}_{u_n}
    \]
    a power series with $a_n = \frac{1}{n!}$ and $x_0 = 0$. Let's determine its domain of convergence using d'Alembert's ratio test:
    \[
        \lim_{n \to \infty} \left| \frac{u_{n+1}}{u_n} \right| = \lim_{n \to \infty} \left| \frac{\frac{x^{n+1}}{(n+1)!}}{\frac{x^n}{n!}} \right| = \lim_{n \to \infty} \left| \frac{x}{n+1} \right| = 0
    \]
    Since this limit is $0$ for all $x \in \mathbb{R}$, the series converges for all real numbers. Thus, the domain of convergence is $D = \mathbb{R}$.
\end{eg}

\begin{eg}
    Let's consider the power series:
    \[
        \sum_{k = 0}^{\infty} a^k (x - x_0)^k 
    \]
    where $a > 0$. Using d'Alembert's ratio test, we have:
    \[
        \lim_{k \to \infty} \left| \frac{a^{k+1} (x - x_0)^{k+1}}{a^k (x - x_0)^k} \right| = \lim_{k \to \infty} |a (x - x_0)| = |a (x - x_ 0)| = a |x - x_0|
    \]
    For the series to converge, we need:
    \[
        |a (x - x_0)| < 1 \implies |x - x_0| < \frac{1}{a}
    \]
    Therefore, the domain of convergence is:
    \[
        D = \left( x_0 - \frac{1}{a}, x_0 + \frac{1}{a} \right)
    \]
    Note that if $|x - x_0| = \frac{1}{a}$, the series becomes:
    \[
        \sum_{k = 0}^{\infty} a^k \left(\pm \frac{1}{a}\right)^k = \sum_{k = 0}^{\infty} (\pm 1)^k
    \]
    which diverges. Thus, the endpoints are not included in the domain of convergence and if $x = x_0 \pm \frac{1}{a}$, the series diverges.
\end{eg}

\begin{theorem}[Convergence Radius of Power Series]
    Let $\sum_{n=0}^{\infty} a_n (x - x_0)^n$ be a power series. Then, there exists a unique $R \in [0, +\infty]$ such that the domain of convergence is:
    \[
        D = \{x \in \mathbb{R}: |x - x_0| < R\}
    \]
    with the following properties:
    \begin{itemize}[itemsep=1pt,label=$\circ$]
        \item If $R = 0$, the series converges only at $x = x_0$.
        \item If $R = +\infty$, the series converges for all $x \in \mathbb{R}$.
        \item If $0 < R < +\infty$, the series converges for all $x$ such that $|x - x_0| < R$ and diverges for all $x$ such that $|x - x_0| > R$. The behavior at the endpoints $x = x_0 \pm R$ must be examined separately.
    \end{itemize}
\end{theorem}
Remark that the convergence radius $R$ is symmetric around $x_0$ and remark that if $R \in \mathbb{R}^*_+$, the domain of convergence has one of the following forms:
\begin{itemize}[itemsep=1pt,label=$\circ$]
    \item $(x_0 - R, x_0 + R)$
    \item $[x_0 - R, x_0 + R)$
    \item $(x_0 - R, x_0 + R]$
    \item $[x_0 - R, x_0 + R]$
\end{itemize}

\subsection{Finding the Convergence Radius}
\begin{theorem}
    Let $\sum_{n=0}^{\infty} a_n (x - x_0)^n$ ($a_n \neq 0 \forall n \in \mathbb{N}$) be a power series. The convergence radius $R$ is given by:
    \begin{itemize}[itemsep=1pt,label=$\circ$]
        \item If the limit $\lim_{n \to \infty} \left| \frac{a_{n+1}}{a_n} \right|$ exists and equals $l$, then:
        \[
            R = \frac{1}{l}
        \]
        \item If the limit $\lim_{n \to \infty} \sqrt[n]{|a_n|}$ exists and equals $l$, then:
        \[
            R = \frac{1}{l}
        \]
    \end{itemize}
    With the convention that $\frac{1}{0} = +\infty$ and $\frac{1}{+\infty} = 0$ since $0 \leq l \leq +\infty$.
\end{theorem}
\begin{proof}
    Let $x \in \mathbb{R}, x \neq x_0$. Applying d'Alembert's ratio test to the series $\sum_{n=0}^{\infty} a_n (x - x_0)^n$ gives:
    \[
        \lim_{n \to \infty} \left| \frac{a_{n+1} (x - x_0)^{n+1}}{a_n (x - x_0)^n} \right| = \lim_{n \to \infty} \left| \frac{a_{n+1}}{a_n} \right| |x - x_0|
    \]
    If the limit $\lim_{n \to \infty} \left| \frac{a_{n+1}}{a_n} \right|$ exists and equals $l$, then the above limit becomes $l |x - x_0|$. According to d'Alembert's ratio test, the series converges if $l |x - x_0| < 1$, which is equivalent to $|x - x_0| < \frac{1}{l}$. Thus, the convergence radius is $R = \frac{1}{l}$.
    Now, by applying Cauchy's root test to the same series:
    \[
        \lim_{n \to \infty} \sqrt[n]{|a_n (x - x_0)^n|} = \lim_{n \to \infty} \sqrt[n]{|a_n|} |x - x_0|
    \]
    If the limit $\lim_{n \to \infty} \sqrt[n]{|a_n|}$ exists and equals $l$, then the above limit becomes $l |x - x_0|$. According to Cauchy's root test, the series converges if $l |x - x_0| < 1$, which is equivalent to $|x - x_0| < \frac{1}{l}$. Thus, the convergence radius is again $R = \frac{1}{l}$.
\end{proof}

\begin{eg}
    Let's compute the convergence radius of the power series:
    \[
        \sum_{n = 1}^{\infty} \frac{(n + 1)^2}{5^n} x^n
    \]
    Using d'Alembert's ratio test, we have:
    \[
        \lim_{n \to \infty} \left| \frac{\frac{(n + 2)^2}{5^{n+1}} x^{n+1}}{\frac{(n + 1)^2}{5^n} x^n} \right| = \lim_{n \to \infty} \left| \frac{(n + 2)^2}{(n + 1)^2} \cdot \frac{x}{5} \right| = \left| \frac{x}{5} \right|
    \]
    For the series to converge, we need:
    \[
        \left| \frac{x}{5} \right| < 1 \implies |x| < 5
    \]
    Let's now check the behavior at the endpoints $x = \pm 5$:
    \begin{itemize}[itemsep=1pt,label=$\circ$]
        \item For $x = 5$:
        \[
            \sum_{n = 1}^{\infty} \frac{(n + 1)^2}{5^n} \cdot 5^n = \sum_{n = 1}^{\infty} (n + 1)^2
        \]
        which diverges.
        \item For $x = -5$:
        \[
            \sum_{n = 1}^{\infty} \frac{(n + 1)^2}{5^n} \cdot (-5)^n = \sum_{n = 1}^{\infty} (n + 1)^2 (-1)^n
        \]
        which also diverges.
    \end{itemize}
    Therefore, the domain of convergence is:
    \[
        D = (-5, 5)
    \]
\end{eg}

\begin{eg}
    Let's compute the convergence radius of the power series:
    \[
        \sum_{n = 1}^{\infty} \frac{(n + 1)^2}{5^n} x^{3n}
    \]
    Since multiples of $a_k = 0$, we can only use Cauchy's root test here:
    \[
        \lim_{n \to \infty} \sqrt[n]{\left| \frac{(n + 1)^2}{5^n} x^{3n} \right|} = \lim_{n \to \infty} \sqrt[n]{\frac{(n + 1)^2}{5^n}} \cdot \sqrt[n]{|x^{3n}|} = \frac{|x^3|}{5} = \frac{|x|^3}{5}
    \]
    For the series to converge, we need:
    \[
        \frac{|x|^3}{5} < 1 \implies |x| < \sqrt[3]{5}
    \]
    Let's now check the behavior at the endpoints $x = \pm \sqrt[3]{5}$:
    \begin{itemize}[itemsep=1pt,label=$\circ$]
        \item For $x = \sqrt[3]{5}$:
        \[
            \sum_{n = 1}^{\infty} \frac{(n + 1)^2}{5^n} \cdot (\sqrt[3]{5})^{3n} = \sum_{n = 1}^{\infty} (n + 1)^2
        \]
        which diverges.
        \item For $x = -\sqrt[3]{5}$:
        \[
            \sum_{n = 1}^{\infty} \frac{(n + 1)^2}{5^n} \cdot (-\sqrt[3]{5})^{3n} = \sum_{n = 1}^{\infty} (n + 1)^2
        \]
        which also diverges.
    \end{itemize}
    Therefore, the domain of convergence is:
    \[
        D = \left( -\sqrt[3]{5}, \sqrt[3]{5} \right)
    \]
\end{eg}

\begin{eg}
    Let's compute the convergence radius of the power series:
    \[
        \sum_{n = 1}^{\infty} \frac{(3n)!}{(n!)^3 b_n}x^n
    \]
    where $b_n$ is a sequence such that $\lim_{n \to \infty} \frac{b_{n + 1}}{b_n} = 3$ and $b_n \neq 0 \ \forall n \in \mathbb{N}$. Using d'Alembert's ratio test, we have:
    \begin{align*}
        \lim_{n \to \infty} \left| \frac{\frac{(3(n + 1))!}{((n + 1)!)^3 b_{n + 1}} x^{n + 1}}{\frac{(3n)!}{(n!)^3 b_n} x^n} \right| &= \lim_{n \to \infty} \frac{(3n + 3)! b_n (n!)^3}{((n + 1)!)^3 b_{n + 1} (3n)!} |x| \\
        &= \lim_{n \to \infty} \frac{(3n + 3)(3n + 2)(3n + 1)}{(n + 1)^3} \cdot \frac{b_n}{b_{n + 1}} |x| \\
        &= 27 \cdot \frac{1}{3} |x| = 9 |x|
    \end{align*}
    For the series to converge, we need:
    \[
        9 |x| < 1 \implies |x| < \frac{1}{9}
    \]
    (We could check the endpoints here, but they would lead to more complex series.) Therefore, the domain of convergence is at least:
    \[
        D \supseteq \left( -\frac{1}{9}, \frac{1}{9} \right)
    \]
\end{eg}

\begin{theorem}[Generalized Cauchy Root Test]
    Let $\sum_{n=0}^{\infty} a_n$ be a series. Then:
    \[
        \limsup_{n \to \infty} \sqrt[n]{|a_n|} = l
    \]
    where $0 \leq l \leq +\infty$, then:
    \begin{itemize}[itemsep=1pt,label=$\circ$]
        \item If $l < 1$, the series converges absolutely.
        \item If $l > 1$, the series diverges.
        \item If $l = 1$, the test is inconclusive.
    \end{itemize}
    
\end{theorem}

\begin{theorem}
    Let $\sum_{n=0}^{\infty} a_n (x - x_0)^n$ be a power series, then if:
    \[
        \limsup_{n \to \infty} \sqrt[n]{|a_n|} = l
    \]
    where $0 \leq l \leq +\infty$, the convergence radius $R$ is given by:
    \[
        R = \frac{1}{l}
    \]
    with the convention that $\frac{1}{0} = +\infty$ and $\frac{1}{+\infty} = 0$.
\end{theorem}

\begin{eg}
    Let's compute the convergence radius of the power series:
    \[
        \sum_{n = 1}^{\infty} a_n x^n \quad \text{where} \quad a_n = \begin{cases}
            \frac{1}{n} & \text{if } n \text{ is even} \\
            \frac{1}{n^n} & \text{if } n \text{ is odd}
        \end{cases}
    \]
    Remark that the limit for the sequence $(a_n)$ does not exists. Using the generalized Cauchy root test, we have:
    \[
        \limsup_{n \to \infty} \sqrt[n]{|a_n|} = \limsup_{n \to \infty} \sqrt[n]{\begin{cases}
            \frac{1}{n} & \text{if } n \text{ is even} \\
            \frac{1}{n^n} & \text{if } n \text{ is odd}
        \end{cases}}
    \]
    We can separate the limit superior into two subsequences:
    \[
        \limsup_{n \to \infty} \sqrt[2n]{\frac{1}{2n}} = 1 \quad \text{and} \quad \limsup_{n \to \infty} \sqrt[2n + 1]{\frac{1}{(2n + 1)^{2n + 1}}} = 0
    \]
    Therefore:
    \[
        \limsup_{n \to \infty} \sqrt[n]{|a_n|} = 1
    \]
    Thus, the convergence radius is:
    \[
        R = \frac{1}{1} = 1
    \]
    Let's now check the behavior at the endpoints $x = \pm 1$:
    \begin{itemize}[itemsep=1pt,label=$\circ$]
        \item For $x = 1$:
        \[
            \sum_{n = 1}^{\infty} a_n = \sum_{n = 1}^{\infty} \begin{cases}
                \frac{1}{n} & \text{if } n \text{ is even} \\
                \frac{1}{n^n} & \text{if } n \text{ is odd}
            \end{cases}
        \]
        The series diverges since the even terms form the harmonic series.
        \item For $x = -1$:
        \[
            \sum_{n = 1}^{\infty} a_n (-1)^n = \sum_{n = 1}^{\infty} \begin{cases}
                \frac{1}{n} & \text{if } n \text{ is even} \\
                -\frac{1}{n^n} & \text{if } n \text{ is odd}
            \end{cases}
        \]
        The series also diverges since the even terms form the harmonic series.
    \end{itemize}
    Therefore, the domain of convergence is:
    \[
        D = (-1, 1)
    \]
\end{eg}

\subsection{Taylor Series}
\begin{definition}[Taylor Series]
    Let $f: I \to \mathbb{R}$ be a function that is of class $C^{\infty}(I)$ (i.e. infinitely differentiable on an open interval $I \subseteq \mathbb{R}$) and let $x_0 \in I$. The Taylor series of $f$ centered at $x_0$ is the power series given by:
    \[
        \sum_{n=0}^{\infty} \frac{f^{(n)}(x_0)}{n!} (x - x_0)^n
    \]
    where $f^{(n)}$ denotes the $n$-th derivative of $f$.
\end{definition}
Remark that if $x_0 = 0$, the Taylor series is also called the Maclaurin series. \\
It's possible to compute the domain $D$ of convergence of a Taylor series but more intrestingly, one can ask if the function $f$ is equal to the function defined by its Taylor series on its domain of convergence ($E \subset D$ is the set for which this equality holds). Remember that for a given function:
\[
    f(x) = \sum_{k = 0}^{n} \frac{f^{(k)}(x_0)}{k!} (x - x_0)^k + \frac{f^{n + 1}(u)}{(n + 1)!}(x - x_0)^{n + 1}
\]
Taking the limit $n \to \infty$ leads to:
\[
    f(x) = \sum_{k = 0}^{\infty} \frac{f^{(k)}(x_0)}{k!} (x - x_0)^k + \lim_{n \to \infty} \frac{f^{n + 1}(u)}{(n + 1)!}(x - x_0)^{n + 1}
\]
Thus, if the remainder term:
\[
    R_n(x) = \frac{f^{(n + 1)}(u)}{(n + 1)!}(x - x_0)^{n + 1}
\]
tends to $0$ as $n \to \infty$, then $f$ is equal to the function defined by its Taylor series on its domain of convergence.

\begin{eg}
    Let's consider the function:
    \[
        f(x) = e^x
    \]
    Which by definition:
    \[
        e^x = \sum_{n=0}^{\infty} \frac{x^n}{n!}
    \]
    Thus it converges for all $x \in \mathbb{R}$ and $D = \mathbb{R}, E = D = \mathbb{R}$
\end{eg}

\begin{eg}
    Let's consider the function:
    \[
        f(x) = \ln(x)
    \]
    Note that $f \in C^{\infty}((0, +\infty))$. Let's compute its Taylor series centered at $x_0 = 1$:
    \[
        f^{(n)}(x) = (-1)^{n - 1} (n - 1)! x^{-n}
    \]
    Thus:
    \[
        f(x) = \sum_{n=1}^{\infty} \frac{(-1)^{n - 1} (n - 1)!}{n!} (x - 1)^n = \sum_{n=1}^{\infty} \frac{(-1)^{n - 1}}{n} (x - 1)^n
    \]
    If we make a change of variable $t = x - 1$, we get:
    \[
        \ln(1 + t) = \sum_{n=1}^{\infty} \frac{(-1)^{n - 1}}{n} t^n
    \]
    Let's determine the domain of convergence using the convergence radius formula:
    \[
        \lim_{n \to \infty} \left| \frac{a_{n + 1}}{a_n} \right| = \lim_{n \to \infty} \left| \frac{\frac{(-1)^n}{n + 1}}{\frac{(-1)^{n - 1}}{n}} \right| = \lim_{n \to \infty} \frac{n}{n + 1} = 1
    \]
    Thus, the convergence radius is $R = 1$. Let's now check the behavior at the endpoints $t = \pm 1$:
    \begin{itemize}[itemsep=1pt,label=$\circ$]
        \item For $t = 1$:
        \[
            \sum_{n=1}^{\infty} \frac{(-1)^{n - 1}}{n} = 1 - \frac{1}{2} + \frac{1}{3} - \frac{1}{4} + \ldots
        \]
        This series converges (conditionally) to $\ln(2)$.
        \item For $t = -1$:
        \[
            \sum_{n=1}^{\infty} \frac{(-1)^{n - 1}}{n} (-1)^n = -1 + \frac{1}{2} - \frac{1}{3} + \frac{1}{4} - \ldots
        \]
        This series diverges.
    \end{itemize}
    Therefore, the domain of convergence is:
    \[
        D_t = (-1, 1] \quad \implies \quad D_x = (0, 2]
    \]
    Finally, let's check if $f$ is equal to the function defined by its Taylor series on $D_x$. We have:
    \[
        \lim_{n \to \infty} \left|R_n(x)\right| = \lim_{n \to \infty} \left|\frac{(-1)^{n + 2}}{u^{n + 1}} n! \frac{(x-1)^{n + 1}}{(n + 1)!}\right| = \lim_{n \to \infty} \frac{1}{n + 1} \left| \frac{x - 1}{u} \right|^{n + 1}
    \]
    Which is difficule to compute directly. We can still take the case where $x = 2$:
    \[
        \lim_{n \to \infty} \left|R_n(2)\right| = \lim_{n \to \infty} \underbrace{\left|\frac{1}{u}\right|^{n + 1}}_{1 < u < 2} = 0
    \] 
    Thus, in this case, $f$ is equal to the function defined by its Taylor series at $x = 2$. Later in this course, we will see that $D = E = (0, 2]$.
\end{eg}

\begin{eg}
    Let's consider the function:
    \[
        f(x) = \frac{1}{1-x}
    \]
    If we remark that:
    \[
        \frac{1}{1-x} = \sum_{n=0}^{\infty} x^n \quad \text{for } |x| < 1
    \]
    we can see that this is the Taylor series of $f$ centered at $x_0 = 0$. Thus, we have $D = E = (-1, 1)$.
\end{eg}

\begin{eg}
    Let's consider the function:
    \[
        f(x) = \sin(x)
    \]
    Its Taylor series centered at $x_0 = 0$ is given by:
    \[
        \sum_{n=0}^{\infty} \frac{(-1)^n}{(2n + 1)!} x^{2n + 1}
    \]
    Using d'Alembert's ratio test, we have (note that since we don't have that all terms are non-zero, we can't use the convergence radius formula directly):
    \[
        \lim_{n \to \infty} \left| \frac{\frac{(-1)^{n + 1}}{(2(n + 1) + 1)!} x^{2(n + 1) + 1}}{\frac{(-1)^{n}}{(2n + 1)!} x^{2n + 1}} \right| = \lim_{n \to \infty} \left| \frac{x^2}{(2n + 2)(2n + 3)} \right| = 0
    \]
    Thus, $D = \mathbb{R}$. Let's now check if $f$ is equal to the function defined by its Taylor series on $D$. We have:
    \[
        \lim_{n \to \infty} \left|R_n(x)\right| = \lim_{n \to \infty} \left|\frac{\pm \sin(u) \text{ or } \pm \cos(u)}{(n + 1)!}\right| |x|^{n + 1} \leq \lim_{n \to \infty} \frac{|x|^{n + 1}}{(n + 1)!} = 0 \quad \forall x \in \mathbb{R}
    \]
\end{eg}
Similarly, we can show for $f(x) = \cos(x)$ that $D = E = \mathbb{R}$. Also remark the following Taylor series:
\[
    \sin(x) = \sum_{n=0}^{\infty} \frac{(-1)^n}{(2n + 1)!} x^{2n + 1}
\]
\[
    \cos(x) = \sum_{n=0}^{\infty} \frac{(-1)^n}{(2n)!} x^{2n}
\]
\[
    e^x = \sum_{n=0}^{\infty} \frac{x^n}{n!}
\]
\[
    \sinh(x) = \sum_{n=0}^{\infty} \frac{1}{(2n + 1)!} x^{2n + 1}
\]
\[
    \cosh(x) = \sum_{n=0}^{\infty} \frac{1}{(2n)!} x^{2n}
\]
for all $x \in \mathbb{R}$.

\begin{eg}
    Let's consider the function:
    \[
        f(x) = \begin{cases}
            e^{-\frac{1}{x^2}} & \text{if } x \neq 0 \\
            0 & \text{if } x = 0
        \end{cases}
    \]
    It's possible to show that $f \in C^{\infty}(\mathbb{R})$ and that $f^{(n)}(0) = 0$ for all $n \in \mathbb{N}$. Thus, the Taylor series of $f$ centered at $x_0 = 0$ is:
    \[
        \sum_{n=0}^{\infty} \frac{f^{(n)}(0)}{n!} x^n = \sum_{n=0}^{\infty} 0 = 0
    \]
    Therefore, the Taylor series converges to $0$ for all $x \in \mathbb{R}$, but $f(x) \neq 0$ for $x \neq 0$. Thus, in this case, we have $D = \mathbb{R}$ but $E = \{0\}$.
\end{eg}

\subsection{Primitive and Derivative of a Function defined by a Power Series}

\begin{definition}[Primitive]
    Let $f: [a, b] \to \mathbb{R}$ be a continuous function. A function $F: [a, b] \to \mathbb{R}$ is called a primitive of $f$ on $[a, b]$ if:
    \[
        F'(x) = f(x) \quad \forall x \in [a, b]
    \]
\end{definition}
Remark that if $F_1$ and $F_2$ are two primitives of $f$ on $[a, b]$, then there exists a constant $c \in \mathbb{R}$ such that:
\[
    F_1(x) - F_2(x) = c \quad \forall x \in [a, b]
\]
Note the following important primitives:
\vskip0.3cm
\begin{center}
    \begin{tabular}{p{0.45\textwidth} | p{0.45\textwidth}}
        \\ {\centering \textbf{$f(x)$} \par} & {\centering \textbf{$F(x)$} \par} \\ \\ \hline \\
        {\centering $\sin(x)$ \par} & {\centering $- \cos(x) + C$ \par} \\ \\
        {\centering$\cos(x)$ \par} & {\centering $\sin(x) + C$ \par} \\ \\
        {\centering$\frac{1}{x}$ \par} & {\centering $\ln(x) + C$ ($x > 0$)\par} \\ \\
        {\centering$x^k$ \par} & {\centering $\frac{x^{k+1}}{k+1} + C$ ($k \neq -1$) \par} \\ \\
        {\centering$e^x$ \par} & {\centering $e^x + C$ \par} \\ \\
        % {\centering$\arcsin x$ \par} & {\centering $\frac{1}{\sqrt{1 - x^2}}$ \par} \\ \\
        % {\centering$\arccos x$ \par} & {\centering $-\frac{1}{\sqrt{1 - x^2}}$ \par} \\ \\
        % {\centering$\arctan x$ \par} & {\centering $\frac{1}{1 + x^2}$ \par} \\ \\
    \end{tabular}
\end{center}

\begin{theorem}
    Let $\sum_{k=0}^{\infty} b_k (x- x_0)^k$ and $\sum_{k = 0}^{\infty} \frac{b_k}{k + 1} (x-x_0)^{k + 1}$ be two power series. The following holds:
    \begin{itemize}[itemsep=1pt,label=$\circ$]
        \item Both series have the same convergence radius $R$.
        \item If $R > 0$, then $f(x) = \sum_{k=0}^{\infty} b_k (x- x_0)^k$ is continuous on $(x_0 - R, x_0 + R)$.
        \item If $R > 0$, then the function $F(x) = \sum_{k = 0}^{\infty} \frac{b_k}{k + 1} (x-x_0)^{k + 1}$ is a primitive of $f$ on $(x_0 - R, x_0 + R)$ such that $F(x_0) = 0$.
    \end{itemize}
\end{theorem}
Remark that if $\sum_{k = 0}^{\infty} a_k(x - x_0)^k$ and $\sum_{k = 1}^{\infty} k a_k (x - x_0)^{k-1}$ two power series, then they also have the same convergence radius $R$ and if $R > 0$, then $f(x) = \sum_{k = 0}^{\infty} a_k(x - x_0)^k$ is continuously differentiable on $(x_0 - R, x_0 + R)$ and its derivative is given by:
\[
    f'(x) = \sum_{k = 1}^{\infty} k a_k (x - x_0)^{k-1}
\]

\begin{eg}
    Let's consider the power series:
    \[
        \frac{1}{1 - z} = \sum_{k = 0}^{\infty} z^k \quad \forall |z| < 1
    \]
    Let $x = 1-z$ ($z = 1-x$), then we have:
    \[
        \frac{1}{x} = \sum_{k = 0}^{\infty} (1 - x)^k = \sum_{k = 0}^{\infty} (-1)^k  (x-1)^k \quad \forall x \in (0,2)
    \]
    Which has a convergence radius $R = 1$ centered at $x_0 = 1$. Now let's consider its primitive:
    \[
        \sum_{k = 0}^{\infty} \frac{(-1)^k}{k + 1} (x-1)^{k + 1} = \sum_{k = 1}^{\infty} \frac{(-1)^{k + 1}}{k}(x-1)^k
    \]
    Which is the Taylor series of $\ln(x)$ centered at $x_0 = 1$. Thus, by the previous theorem, we have that $\sum_{k = 1}^{\infty} \frac{(-1)^{k + 1}}{k}(x-1)^k$ is a primitive of $\frac{1}{x}$ and both series have the same convergence radius $R = 1$. Therefore, we have:
    \begin{align*}
        F(x) &= \left(\text{ the primitive of } \frac{1}{x} \text{ such that } F(1) = 0\right) \\
        &= \ln(x) = \sum_{k = 1}^{\infty} \frac{(-1)^{k + 1}}{k}(x-1)^k \quad \forall x \in (0,2)
    \end{align*}
    In a previous example, we showed that this Taylor series converges to $\ln(2)$ at $x = 2$ as well, thus we have $D = E = (0, 2]$.
\end{eg}
From the example, it can be extracted that:
\[
    \ln(x) = \sum_{k = 1}^{\infty} \frac{(-1)^{k + 1}}{k}(x-1)^k \quad \forall x \in (0,2]
\]
\[
    \frac{1}{x} = \sum_{k = 0}^{\infty} (-1)^k  (x-1)^k \quad \forall x \in (0,2)
\]
Remark that the domain of convergence of both series are not the same in this case, but they share the same convergence radius.