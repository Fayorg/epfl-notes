\chapter{Integration}

\section{Integration of a Continuous Function}

\begin{definition}[Darboux Sums]
    Let $f: [a, b] \to \mathbb{R}$ be a continuous function. Let $\sigma = \{x_0 = a < x_1 < x_2 < \cdots < x_n = b\}$ be a partition of the interval $[a, b]$.
\end{definition}

\begin{eg}
    Let $\sigma_{\text{reg}} = \{a, a + \frac{b - a}{n}, a + 2\frac{b - a}{n}, \ldots, b\}$ be a regular partition of the interval $[a, b]$.
\end{eg}

\begin{definition}[Step of a Partition]
    Let $\sigma = \{x_0 = a < x_1 < x_2 < \cdots < x_n = b\}$ be a partition of the interval $[a, b]$. The step of the partition $\sigma$ is defined as:
    \[
        \mathcal{P}(\sigma) = \max_{1 \leq i \leq n} (x_i - x_{i-1})
    \]
\end{definition}

\begin{definition}[Upper and Lower Darboux Sums]
    Let $f: [a, b] \to \mathbb{R}$ be a continuous function and $\sigma = \{x_0 = a < x_1 < x_2 < \cdots < x_n = b\}$ be a partition of the interval $[a, b]$.
    The upper Darboux sum $\bar{S}_{\sigma}(f)$ of $f$ relative to $\sigma$ and lower Darboux sum $\underline{S}_{\sigma}(f)$ of $f$ relative to $\sigma$ are defined as:
    \[
        \bar{S}_{\sigma}(f) = \sum_{i=1}^{n} M_i (x_i - x_{i-1}), \quad \underline{S}_{\sigma}(f) = \sum_{i=1}^{n} m_i (x_i - x_{i-1})
    \]
    where $M_i = \max_{x \in [x_{i-1}, x_i]} f(x)$ and $m_i = \min_{x \in [x_{i-1}, x_i]} f(x)$.
\end{definition}
% TODO: add graphical representation of upper and lower sums
Remark that the following inequality can be extracted from the definitions:
\[
    m (b-a) \leq \underline{S}_{\sigma}(f) \leq \bar{S}_{\sigma}(f) \leq M (b-a)
\]
where $m = \min_{x \in [a, b]} f(x)$ and $M = \max_{x \in [a, b]} f(x)$. Also note that if $\sigma_1 \subset \sigma_2$, then:
\[
    \underline{S}_{\sigma_1}(f) \leq \underline{S}_{\sigma_2}(f) \quad \text{and} \quad \bar{S}_{\sigma_1}(f) \geq \bar{S}_{\sigma_2}(f)
\]
Remark that the supremum and infimum of the sets of lower and upper Darboux sums can be defined as:
\[
    \underline{S}(f) = \inf \{\underline{S}_\sigma, \sigma \text{ subdivision of } [a,b]\} \quad \text{and} \quad \bar{S}(f) = \sup \{\bar{S}_\sigma, \sigma \text{ subdivision of } [a,b]\}
\]

\begin{theorem}
    Let $f: [a, b] \to \mathbb{R}$ be a continuous function. Then $\underline{S}(f) = \bar{S}(f)$.
\end{theorem}

\begin{definition}[Riemann Integral]
    Let $f: [a, b] \to \mathbb{R}$ be a continuous function such that $\underline{S}(f) = \bar{S}(f)$. The Riemann integral of $f$ over the interval $[a, b]$ is defined as:
    \[
        \int_a^b f(x) \, dx = \underline{S}(f) = \bar{S}(f)
    \]
\end{definition}
Note that if $b < a$ then we define:
\[
    \int_a^b f(x) \, dx = -\int_b^a f(x) \, dx
\]
and if $a = b$ then:
\[
    \int_a^a f(x) \, dx = 0
\]

\begin{eg}
    Let's compute, using Darboux sums, the integral:
    \[
        \int_0^3 x \, dx
    \]
    Let $\sigma_n = \{0, 3/n, 6/n, \ldots, 3\}$ be a regular partition of the interval $[0, 3]$ with step $\mathcal{P}(\sigma_n) = 3/n$. We have:
    \[
        M_i = \max_{x \in [\frac{3(i-1)}{n}, \frac{3i}{n}]} x = \frac{3i}{n} \quad \text{and} \quad m_i = \min_{x \in [\frac{3(i-1)}{n}, \frac{3i}{n}]} x = \frac{3(i-1)}{n}
    \]
    Thus:
    \[
        \bar{S}_{\sigma_n}(f) = \sum_{i=1}^{n} \frac{3i}{n} \cdot \frac{3}{n} = \frac{9}{n^2} \sum_{i=1}^{n} i = \frac{9}{n^2} \cdot \frac{n(n+1)}{2} = \frac{9(n+1)}{2n}
    \]
    and
    \[        \underline{S}_{\sigma_n}(f) = \sum_{i=1}^{n} \frac{3(i-1)}{n} \cdot \frac{3}{n} = \frac{9}{n^2} \sum_{i=1}^{n} (i-1) = \frac{9}{n^2} \cdot \frac{(n-1)n}{2} = \frac{9(n-1)}{2n}
    \]
    Therefore, we have:
    \[        \underline{S}(f) = \lim_{n \to \infty} \underline{S}_{\sigma_n}(f) = \lim_{n \to \infty} \frac{9(n-1)}{2n} = \frac{9}{2}
    \]
    and:
    \[
        \bar{S}(f) = \lim_{n \to \infty} \bar{S}_{\sigma_n}(f) = \lim_{n \to \infty} \frac{9(n+1)}{2n} = \frac{9}{2}
    \]
    Thus, we conclude that:
    \[        \int_0^3 x \, dx = \frac{9}{2}
    \]
\end{eg}

\subsection{Average Value Theorem for Integrals}
Let $f: [a,b] \to \mathbb{R}$ be a continuous function. The following properties of the Riemann integral can be easily derived from its definition:
\begin{itemize}[itemsep=1pt,label=$\circ$]
    \item Let $c \in [a,b]$. Then:
    \[
        \int_a^b f(x) \, dx = \int_a^c f(x) \, dx + \int_c^b f(x) \, dx
    \]
    \item Since $m(b-a) \leq \underline{S}(f) \leq \overline{S}(f) \leq M (b-a)$, then:
    \[
        m(b-a) \leq \int_a^b f(x) \, dx \leq M(b-a)
    \]
\end{itemize}

\begin{theorem}[Average Value Theorem for Integrals]
    Let $f: [a,b] \to \mathbb{R}$ be a continuous function. Then, there exists $c \in [a,b]$ such that:
    \[
        \int_a^b f(x) \, dx = f(c) (b-a)
    \]
\end{theorem}
\begin{proof}
    Since $f$ is continuous on the interval $[a,b]$, by the Extreme Value Theorem, there exist points $x_m, x_M \in [a,b]$ such that:
    \[
        f(x_m) = m = \min_{x \in [a,b]} f(x) \quad \text{and} \quad f(x_M) = M = \max_{x \in [a,b]} f(x)
    \]
    From the properties of the Riemann integral, we have:
    \[
        m(b-a) \leq \int_a^b f(x) \, dx \leq M(b-a)
    \]
    Dividing the entire inequality by $(b-a)$ (assuming $b > a$), we get:
    \[
        m \leq \frac{1}{b-a} \int_a^b f(x) \, dx \leq M
    \]
    By the Intermediate Value Theorem, since $f$ is continuous on $[a,b]$, there exists $c \in [a,b]$ such that:
    \[
        f(c) = \frac{1}{b-a} \int_a^b f(x) \, dx
    \]
    Multiplying both sides by $(b-a)$ gives:
    \[
        \int_a^b f(x) \, dx = f(c) (b-a)
    \]
    This completes the proof.
\end{proof}

\subsection{Relation Between Integral and Primitive}
The previous chapter defined the notion of primitive of a function, now let's add these notions on top of it.

\begin{theorem}
    Let $a < b$ and $f$ a continuous function on $[a,b]$. Then:
    \[
        F(x) = \int_a^x f(t) \, dt
    \]
    is a primitive of $f$ on $[a,b]$ such that $F(a) = 0$.
\end{theorem}
\begin{proof}
    Let $x \in [a,b]$ and $h \neq 0$ such that $x + h \in [a,b]$. We have:
    \begin{align*}
        \frac{F(x+h) - F(x)}{h} &= \frac{1}{h} \left( \int_a^{x+h} f(t) \, dt - \int_a^x f(t) \, dt \right) \\
        &= \frac{1}{h} \left( \int_a^{x} f(t) \, dt + \int_x^{x+h} f(t) \, dt - \int_a^x f(t) \, dt \right) \\
        &= \frac{1}{h} \int_x^{x+h} f(t) \, dt
    \end{align*}
    By the Average Value Theorem for Integrals, there exists $c_h \in [x, x+h]$ such that:
    \[
        \frac{1}{h} \int_x^{x+h} f(t) \, dt = f(c_h)
    \]
    As $h$ approaches $0$, $c_h$ approaches $x$. Since $f$ is continuous at $x$, we have:
    \[
        \lim_{h \to 0} f(c_h) = f(x)
    \]
    Therefore:
    \[
        F'(x) = \lim_{h \to 0} \frac{F(x+h) - F(x)}{h} = f(x)
    \]
    This shows that $F$ is a primitive of $f$ on $[a,b]$. Additionally, by definition:
    \[
        F(a) = \int_a^a f(t) \, dt = 0
    \]
    Thus, the theorem is proved.
\end{proof}

\begin{theorem}[Fundamental Theorem of Calculus]
    Let $f: [a,b] \to \mathbb{R}$ be a continuous function and $F$ be any primitive of $f$ on $[a,b]$. Then:
    \[
        \int_a^b f(x) \, dx = F(b) - F(a)
    \]
\end{theorem}
\begin{proof}
    Let $G(x) = \int_a^x f(t) \, dt$. From the previous theorem, we know that $G$ is a primitive of $f$ on $[a,b]$. Since $F$ is also a primitive of $f$, there exists a constant $C$ such that:
    \[
        F(x) = G(x) + C
    \]
    for all $x \in [a,b]$. Evaluating this at $x = a$, we have:
    \[
        F(a) = G(a) + C = 0 + C \implies C = F(a)
    \]
    Therefore, we can express $F(x)$ as:
    \[
        F(x) = G(x) + F(a)
    \]
    Now, evaluating at $x = b$, we get:
    \[
        F(b) = G(b) + F(a)
    \]
    Rearranging this gives:
    \[
        G(b) = F(b) - F(a)
    \]
    But by definition of $G$, we have:
    \[
        G(b) = \int_a^b f(t) \, dt
    \]
    Thus, we conclude that:
    \[
        \int_a^b f(x) \, dx = F(b) - F(a)
    \]
    This completes the proof.
\end{proof}

\subsection{List of Basic Integrals}
Here is a list, based on the one from the previous chapter, of some basic integrals:
\vskip0.3cm
\begin{center}
    \begin{tabular}{p{0.45\textwidth} | p{0.45\textwidth}}
        \\ {\centering \textbf{$f(x)$} \par} & {\centering \textbf{$F(x)$} \par} \\ \\ \hline \\
        {\centering$e^x$ \par} & {\centering $e^x + C$ \par} \\ \\
        {\centering $\sin(x)$ \par} & {\centering $- \cos(x) + C$ \par} \\ \\
        {\centering$\cos(x)$ \par} & {\centering $\sin(x) + C$ \par} \\ \\
        {\centering$\cosh(x)$ \par} & {\centering $\sinh(x) + C$ \par} \\ \\
        {\centering$\sinh(x)$ \par} & {\centering $\cosh(x) + C$ \par} \\ \\
        {\centering$a^x$, ($a>0, a \neq 1$) \par} & {\centering $\frac{a^x}{\ln(a)} + C$ \par} \\ \\
        {\centering$\frac{1}{x}$ ($x \neq 0$) \par} & {\centering $\ln|x| + C$\par} \\ \\
        {\centering$x^k$ \par} & {\centering $\frac{x^{k+1}}{k+1} + C$ ($k \neq -1$) \par} \\ \\
        {\centering$\frac{1}{\cos^2(x)}$ \par} & {\centering $\tan(x) + C$ \par} \\ \\
        {\centering$\frac{1}{\sin^2(x)}$ \par} & {\centering $- \cot(x) + C$ \par} \\ \\
        {\centering$\frac{1}{1 + x^2}$ \par} & {\centering $\arctan(x) + C$ \par} \\ \\
        {\centering$\frac{1}{\sqrt{1 - x^2}}$ \par} & {\centering $\arcsin(x) + C$ \par} \\ \\
    \end{tabular}
\end{center}

\begin{eg}
    Let compute the integral:
    \[
        \int_0^\pi \sin(x) \, dx
    \]
    We know that a primitive of $\sin(x)$ is $-\cos(x) + C$. Thus, by the Fundamental Theorem of Calculus, we have:
    \[        \int_0^\pi \sin(x) \, dx = -\cos(x) \Big|_0^\pi = -\cos(\pi) + \cos(0) = 2
    \]
\end{eg}

\subsection{Properties of Integrals}
Let $f, g: [a,b] \to \mathbb{R}$ be two continuous functions. The following properties hold:
\begin{itemize}[itemsep=1pt,label=$\circ$]
    \item Linearity:
    \[
        \int_a^b (\alpha f(x) + \beta g(x)) \, dx = \alpha \int_a^b f(x) \, dx + \beta \int_a^b g(x) \, dx
    \]
    \item If $f(x) \geq 0$ for all $x \in [a,b]$ and $c \in (a,b)$ then:
    \[
        \int_a^b f(x) \, dx \geq \int_a^c f(x) \, dx \geq 0
    \]
    \item If $f(x) \leq g(x)$ for all $x \in [a,b]$ then:
    \[
        \int_a^b f(x) \, dx \leq \int_a^b g(x) \, dx
    \]
    \item Let $h,m: I \to [a, b]$ differentiable functions on $I$ then:
    \[
        \frac{d}{dx} \left( \int_{h(x)}^{m(x)} f(t) \, dt \right) = f(m(x)) \cdot m'(x) - f(h(x)) \cdot h'(x)
    \]
\end{itemize}

\begin{eg}
    Let's compute:
    \[
        \frac{d}{dx} \left( \int_1^{3x^2 + 3} e^{t^2} \, dt \right)
    \]
    By the last property, we have:
    \[
        \frac{d}{dx} \left( \int_1^{3x^2 + 3} e^{t^2} \, dt \right) = e^{(3x^2 + 3)^2} \cdot 6x
    \]
\end{eg}

\section{Techniques of Integration}
This section presents some techniques to compute integrals.

\subsection{Integration by Change of Variables}
\begin{theorem}[Integration by Change of Variables]
    Let $f: [a,b] \to \mathbb{R}$ be a continuous function and $\varphi: [c,d] \to [a,b]$ be a differentiable function such that $\varphi([c,d]) \subseteq [a,b]$. Then:
    \[
        \int_{\varphi(c)}^{\varphi(d)} f(x) \, dx = \int_c^d f(\varphi(t)) \cdot \varphi'(t) \, dt
    \]
    where $x = \varphi(t)$ is the change of variable.
\end{theorem}
\begin{proof}
    Let $F$ be a primitive of $f$ on $[a,b]$. By the Fundamental Theorem of Calculus, we have:
    \[
        \int_{\varphi(c)}^{\varphi(d)} f(x) \, dx = F(\varphi(d)) - F(\varphi(c))
    \]
    Now, consider the function $G(t) = F(\varphi(t))$. By the Chain Rule, we have:
    \[
        G'(t) = F'(\varphi(t)) \cdot \varphi'(t) = f(\varphi(t)) \cdot \varphi'(t)
    \]
    Thus, $G$ is a primitive of the function $f(\varphi(t)) \cdot \varphi'(t)$ on $[c,d]$. Applying the Fundamental Theorem of Calculus again, we get:
    \[
        \int_c^d f(\varphi(t)) \cdot \varphi'(t) \, dt = G(d) - G(c) = F(\varphi(d)) - F(\varphi(c))
    \]
    Therefore, we conclude that:
    \[
        \int_{\varphi(c)}^{\varphi(d)} f(x) \, dx = \int_c^d f(\varphi(t)) \cdot \varphi'(t) \, dt
    \]
    This completes the proof.
\end{proof}

\begin{eg}
    Let's compute the integral:
    \[
        \int_1^2 x e^{x^2} \, dx
    \]
    Let $x = \varphi(t) = \sqrt{t}$, thus $t = \varphi^{-1}(x) = x^2$ and $\varphi'(t) = \frac{1}{2\sqrt{t}}$. When $x = 1$, $t = 1$ and when $x = 2$, $t = 4$. Therefore:
    \[        \int_1^2 x e^{x^2} \, dx = \int_1^4 \sqrt{t} \cdot e^t \cdot \frac{1}{2\sqrt{t}} \, dt = \frac{1}{2} \int_1^4 e^t \, dt = \frac{1}{2} (e^4 - e)
    \]
\end{eg}
Another more direct approaches:

\begin{eg}
    Let's compute the same integral:
    \[
        \int_1^2 x e^{x^2} \, dx
    \]
    Let $u = x^2$, thus $du = 2x \, dx$ or $\frac{du}{2} = x \, dx$. When $x = 1$, $u = 1$ and when $x = 2$, $u = 4$. Therefore:
    \[        \int_1^2 x e^{x^2} \, dx = \int_1^4 e^u \cdot \frac{du}{2} = \frac{1}{2} \int_1^4 e^u \, du = \frac{1}{2} (e^4 - e)
    \]
\end{eg}

\begin{eg}
    Let's compute the integral:
    \[
        \int_{1/8}^{1/3} \sqrt{\frac{x + 1}{x}} \cdot \frac{1}{x^2} \, dx
    \]
    Let $u = \frac{x + 1}{x} = 1 + \frac{1}{x}$, thus $du = -\frac{1}{x^2} \, dx$ or $-du = \frac{1}{x^2} \, dx$. When $x = \frac{1}{8}$, $u = 9$ and when $x = \frac{1}{3}$, $u = 4$. Therefore:
    \[        \int_{1/8}^{1/3} \sqrt{\frac{x + 1}{x}} \cdot \frac{1}{x^2} \, dx = \int_9^4 \sqrt{u} \cdot (-du) = \int_4^9 \sqrt{u} \, du = \frac{2}{3} u^{3/2} \Big|_4^9 = \frac{2}{3} (27 - 8) = \frac{38}{3}
    \]
\end{eg}

\begin{eg}
    Let's compute the indefinite integral:
    \[
        \int \frac{x}{\sqrt{1-x^2}} \, dx
    \]
    Let $u = 1 - x^2$, thus $du = -2x \, dx$ or $-\frac{du}{2} = x \, dx$. Therefore:
    \[        \int \frac{x}{\sqrt{1-x^2}} \, dx = \int \frac{-\frac{du}{2}}{\sqrt{u}} = -\frac{1}{2} \int u^{-1/2} \, du = -\frac{1}{2} \cdot 2 u^{1/2} + C = -\sqrt{1 - x^2} + C
    \]
\end{eg}

\begin{eg}
    Let's compute the indefinite integral:
    \[
        \int \frac{1}{\sin x} \, dx
    \]
    Let's multiply by $\frac{\sin x}{\sin x}$:
    \[        \int \frac{1}{\sin x} \, dx = \int \frac{\sin x}{\sin^2 x} \, dx = \int \frac{\sin x}{1 - \cos^2 x} \, dx
    \]
    Let $u = \cos x$, thus $du = -\sin x \, dx$ or $-du = \sin x \, dx$. Therefore:
    \[        \int \frac{1}{\sin x} \, dx = \int \frac{-du}{1 - u^2} = -\int \frac{1}{(1 - u)(1 + u)} \, du
    \]
    Using partial fraction decomposition, we have:
    \[        \frac{1}{(1 - u)(1 + u)} = \frac{A}{1 - u} + \frac{B}{1 + u}
    \]
    Solving for $A$ and $B$, we find $A = \frac{1}{2}$ and $B = \frac{1}{2}$. Thus:
    \[        \int \frac{1}{\sin x} \, dx = -\frac{1}{2} \int \frac{1}{1 - u} \, du - \frac{1}{2} \int \frac{1}{1 + u} \, du
    \]
    \[        = -\frac{1}{2} (-\ln|1 - u|) - \frac{1}{2} (\ln|1 + u|) + C = \frac{1}{2} \ln|1 - \cos x| - \frac{1}{2} \ln|1 + \cos x| + C
    \]
    \[        = \frac{1}{2} \ln \left| \frac{1 - \cos x}{1 + \cos x} \right| + C = \frac{1}{2} \ln \left| \tan^2 \left( \frac{x}{2} \right) \right| + C = \ln \left| \tan \left( \frac{x}{2} \right) \right| + C
    \]
\end{eg}

\begin{eg}
    Taking the same integral as the previous example:
    \[        \int \frac{1}{\sin x} \, dx
    \]
    Using trigonometric identities, we have:
    \[        \int \frac{1}{\sin x} \, dx = \int \frac{\left(\sin \left(\frac{x}{2}\right)\right)^2 + \left(\cos \left(\frac{x}{2}\right)\right)^2}{2 \sin \left(\frac{x}{2}\right) \cos \left(\frac{x}{2}\right)} \, dx
    \]
    \[        = \frac{1}{2} \int \frac{\sin \left(\frac{x}{2}\right)}{\cos \left(\frac{x}{2}\right)} \, dx + \frac{1}{2} \int \frac{\cos \left(\frac{x}{2}\right)}{\sin \left(\frac{x}{2}\right)} \, dx
    \]
    Let's compute each integral separately. For the first one, let $u = \cos \left(\frac{x}{2}\right)$, thus $du = -\frac{1}{2} \sin \left(\frac{x}{2}\right) \, dx$ or $-2 du = \sin \left(\frac{x}{2}\right) \, dx$. Therefore:
    \[        \frac{1}{2} \int \frac{\sin \left(\frac{x}{2}\right)}{\cos \left(\frac{x}{2}\right)} \, dx = \frac{1}{2} \int \frac{-2 du}{u} = -\int \frac{1}{u} \, du = -\ln|u| + C = -\ln \left| \cos \left(\frac{x}{2}\right) \right| + C
    \]
    For the second integral, let $v = \sin \left(\frac{x}{2}\right)$, thus $dv = \frac{1}{2} \cos \left(\frac{x}{2}\right) \, dx$ or $2 dv = \cos \left(\frac{x}{2}\right) \, dx$. Therefore:
    \[        \frac{1}{2} \int \frac{\cos \left(\frac{x}{2}\right)}{\sin \left(\frac{x}{2}\right)} \, dx = \frac{1}{2} \int \frac{2 dv}{v} = \int \frac{1}{v} \, dv = \ln|v| + C = \ln \left| \sin \left(\frac{x}{2}\right) \right| + C
    \]
    Combining both results, we have:
    \[        \int \frac{1}{\sin x} \, dx = -\ln \left| \cos \left(\frac{x}{2}\right) \right| + \ln \left| \sin \left(\frac{x}{2}\right) \right| + C = \ln \left| \tan \left(\frac{x}{2}\right) \right| + C
    \]
\end{eg}
Remark that some integrals can be computed using multiple techniques, and thus the resulting expressions for the primitive may look different but are equivalent up to an additive constant i.e:
\[
    \ln 5x + C_1 \quad \text{and} \quad \ln x + \ln 5 + C_2
\]
\[
    \frac{1}{\left(\cos x\right)^2} + C_1 \quad \text{and} \quad \left(\tan x\right)^2 + C_2 \quad \left(\left(\tan x\right)^2 = \frac{1 - \left(\cos x\right)^2}{\left(\cos x\right)^2} = \frac{1}{\left(\cos x\right)^2} - 1\right)
\]

\subsection{Integration by Parts}
\begin{theorem}[Integration by Parts]
    Let $f, g: [a,b] \to \mathbb{R}$ be two differentiable functions. Then:
    \[
        \int_a^b f(x) g'(x) \, dx = f(x) g(x) \Big|_a^b - \int_a^b f'(x) g(x) \, dx
    \]
\end{theorem}
\begin{proof}
    By the Product Rule, we have:
    \[
        \frac{d}{dx} [f(x) g(x)] = f'(x) g(x) + f(x) g'(x)
    \]
    Integrating both sides from $a$ to $b$, we get:
    \[
        \int_a^b \frac{d}{dx} [f(x) g(x)] \, dx = \int_a^b f'(x) g(x) \, dx + \int_a^b f(x) g'(x) \, dx
    \]
    The left-hand side simplifies to:
    \[
        f(x) g(x) \Big|_a^b = f(b) g(b) - f(a) g(a)
    \]
    Therefore, we have:
    \[
        f(b) g(b) - f(a) g(a) = \int_a^b f'(x) g(x) \, dx + \int_a^b f(x) g'(x) \, dx
    \]
    Rearranging this equation gives:
    \[
        \int_a^b f(x) g'(x) \, dx = f(b) g(b) - f(a) g(a) - \int_a^b f'(x) g(x) \, dx
    \]
    This completes the proof.
\end{proof}

\begin{eg}
    Let's compute the integral:
    \[
        \int_1^2 x^2 \ln x \, dx
    \]
    Let $f(x) = \ln x$ and $g'(x) = x^2$. Thus, $f'(x) = \frac{1}{x}$ and $g(x) = \frac{x^3}{3}$. By the Integration by Parts theorem, we have:
    \[        \int_1^2 x^2 \ln x \, dx = \ln x \cdot \frac{x^3}{3} \Big|_1^2 - \int_1^2 \frac{1}{x} \cdot \frac{x^3}{3} \, dx
    \]
    \[        = \frac{8}{3} \ln 2 - 0 - \frac{1}{3} \int_1^2 x^2 \, dx = \frac{8}{3} \ln 2 - \frac{1}{3} \cdot \frac{x^3}{3} \Big|_1^2 = \frac{8}{3} \ln 2 - \frac{1}{3} \left( \frac{8}{3} - \frac{1}{3} \right) = \frac{8}{3} \ln 2 - \frac{7}{9}
    \]
\end{eg}
Remark that the Integration by Parts usually works well for:
\begin{itemize}[itemsep=1pt,label=$\circ$]
    \item $\int e^{ax} P(n) \, dx$
    \item $\int \left(\ln x\right)^k P(n) \, dx$
    \item $\int \left\{\sin x, \cos x\right\} P(n) \, dx$
    \item $\int e^{ax} \left\{\sin x, \cos x\right\} \, dx$
\end{itemize}
where $P(n)$ is a polynomial.

\begin{eg}
    Let's compute the indefinite integral:
    \[
        \int \sin x e^x \, dx
    \]
    Let $f(x) = \sin x$ and $g'(x) = e^x$. Thus, $f'(x) = \cos x$ and $g(x) = e^x$. By the Integration by Parts theorem, we have:
    \[        \int \sin x e^x \, dx = \sin x \cdot e^x - \int \cos x \cdot e^x \, dx
    \]
    Now, let's compute $\int \cos x \cdot e^x \, dx$ using Integration by Parts again. Let $f(x) = \cos x$ and $g'(x) = e^x$. Thus, $f'(x) = -\sin x$ and $g(x) = e^x$. We have:
    \[        \int \cos x \cdot e^x \, dx = \cos x \cdot e^x - \int -\sin x \cdot e^x \, dx = \cos x \cdot e^x + \int \sin x \cdot e^x \, dx
    \]
    Therefore, we have:
    \[        \int \sin x e^x \, dx = \sin x \cdot e^x - \left( \cos x \cdot e^x + \int \sin x \cdot e^x \, dx \right)
    \]
    \[        = \sin x \cdot e^x - \cos x \cdot e^x - \int \sin x \cdot e^x \, dx
    \]
    Rearranging this equation gives:
    \[        2 \int \sin x e^x \, dx = e^x (\sin x - \cos x)
    \]
    Thus, we conclude that:
    \[        \int \sin x e^x \, dx = \frac{e^x (\sin x - \cos x)}{2} + C
    \]
\end{eg}

\begin{eg}
    Let's compute the indefinite integral:
    \[
        \int \ln (1 + \sqrt{1 + x}) \, dx
    \]
    Let $f(x) = \ln (1 + \sqrt{1 + x})$ and $g'(x) = 1$. Thus, $f'(x) = \frac{1}{(1 + \sqrt{1 + x})} \cdot \frac{1}{2\sqrt{1 + x}}$ and $g(x) = x$. By the Integration by Parts theorem, we have:
    \[        \int \ln (1 + \sqrt{1 + x}) \, dx = x \ln (1 + \sqrt{1 + x}) - \int x \cdot \frac{1}{(1 + \sqrt{1 + x})} \cdot \frac{1}{2\sqrt{1 + x}} \, dx
    \]
    \[        = x \ln (1 + \sqrt{1 + x}) - \frac{1}{2} \int \frac{x}{\sqrt{1 + x} (1 + \sqrt{1 + x})} \, dx
    \]
    Let $u = \sqrt{1 + x}$, thus $x = u^2 - 1$ and $dx = 2u \, du$. Therefore:
    \[        \int \ln (1 + \sqrt{1 + x}) \, dx = x \ln (1 + \sqrt{1 + x}) - \frac{1}{2} \int \frac{u^2 - 1}{u (1 + u)} \cdot 2u \, du
    \]
    \[        = x \ln (1 + \sqrt{1 + x}) - \int \frac{u^2 - 1}{1 + u} \, du = x \ln (1 + \sqrt{1 + x}) - \int (u - 1) \, du
    \]
    \[        = x \ln (1 + \sqrt{1 + x}) - \left( \frac{u^2}{2} - u \right) + C = x \ln (1 + \sqrt{1 + x}) - \left( \frac{1 + x}{2} - \sqrt{1 + x} \right) + C
    \]
\end{eg}

\subsection{Integrals of Rational Functions}
Remark that any rational function $\frac{P(x)}{Q(x)}$ where $P$ and $Q$ are polynomials can be expressed in terms of elementary functions i.e:
\[
    \frac{f(x)}{g(x)} = p(x) + \sum_{i} \frac{f_i(x)}{g_i(x)}
\]
where $p(x)$ is a polynomial and:
\[
    g_i(x) = \begin{cases}
        (ax + b)^k & \text{if } f_i(x) = A_i \\
        (ax^2 + bx + c)^k & \text{if } f_i(x) = A_i x + B_i, \quad b^2 - 4ac < 0
    \end{cases}
\]

\begin{eg}
    Let's compute the indefinite integral:
    \[
        \int \frac{1}{ax + b} \, dx
    \]
    where $a \neq 0$. Let $u = ax + b$, thus $du = a \, dx$ or $\frac{du}{a} = dx$. Therefore:
    \[        \int \frac{1}{ax + b} \, dx = \int \frac{1}{u} \cdot \frac{du}{a} = \frac{1}{a} \int \frac{1}{u} \, du = \frac{1}{a} \ln|u| + C = \frac{1}{a} \ln|ax + b| + C
    \]
\end{eg}

\begin{eg}
    Let's compute the indefinite integral:
    \[
        \int \frac{(cx + d)}{(x - a)(x - b)} \, dx
    \]
    where $a \neq b$. Using partial fraction decomposition, we have:
    \[        \frac{cx + d}{(x - a)(x - b)} = \frac{A}{x - a} + \frac{B}{x - b}
    \]
    Solving for $A$ and $B$, we find:
    \[        A = \frac{ca + d}{a - b} \quad \text{and} \quad B = \frac{cb + d}{b - a}
    \]
    Thus:
    \[        \int \frac{(cx + d)}{(x - a)(x - b)} \, dx = \int \frac{ca + d}{a - b} \cdot \frac{1}{x - a} \, dx + \int \frac{cb + d}{b - a} \cdot \frac{1}{x - b} \, dx
    \]
    \[        = \frac{ca + d}{a - b} \ln|x - a| + \frac{cb + d}{b - a} \ln|x - b| + C
    \]
\end{eg}

\begin{eg}
    Let's compute the indefinite integral:
    \[
        \int \frac{1}{(ax + b)^k} \, dx
    \]
    where $a \neq 0$ and $k \geq 2$. Let $u = ax + b$, thus $du = a \, dx$ or $\frac{du}{a} = dx$. Therefore:
    \[        \int \frac{1}{(ax + b)^k} \, dx = \int \frac{1}{u^k} \cdot \frac{du}{a} = \frac{1}{a} \int u^{-k} \, du = \frac{1}{a} \cdot \frac{u^{-k+1}}{-k + 1} + C = \frac{1}{a(1 - k)} \cdot \frac{1}{(ax + b)^{k - 1}} + C
    \]
\end{eg}

\begin{eg}
    Let's compute the indefinite integral:
    \[
        \int \frac{1}{x^2 + c^2} \, dx
    \]
    where $c \neq 0$. Let $u = \frac{x}{c}$, thus $x = cu$ and $dx = c \, du$. Therefore:
    \begin{align*}
        \int \frac{1}{x^2 + c^2} \, dx &= \int \frac{1}{c^2 u^2 + c^2} \cdot c \, du = \int \frac{1}{c(u^2 + 1)} \, du \\
        &= \frac{1}{c} \int \frac{1}{u^2 + 1} \, du = \frac{1}{c} \arctan(u) + C \\
        &= \frac{1}{c} \arctan\left(\frac{x}{c}\right) + C
    \end{align*}
\end{eg}
Remark that for:
\[
    \int \frac{1}{x^2 + px + q} \, dx
\]
where $p^2 - 4q < 0$, we can complete the square:
\[    x^2 + px + q = \left(x + \frac{p}{2}\right)^2 + \left(q - \frac{p^2}{4}\right)
\]
and then use the previous result.

\begin{eg}
    Let's compute the indefinite integral:
    \[
        \int \frac{x}{x^2 + c^2} \, dx
    \]
    where $c \neq 0$. Let $u = x^2 + c^2$, thus $du = 2x \, dx$ or $\frac{du}{2} = x \, dx$. Therefore:
    \[        \int \frac{x}{x^2 + c^2} \, dx = \int \frac{1}{u} \cdot \frac{du}{2} = \frac{1}{2} \int \frac{1}{u} \, du = \frac{1}{2} \ln|u| + C = \frac{1}{2} \ln|x^2 + c^2| + C
    \]
\end{eg}

\begin{eg}
    Let's compute the integral:
    \[
        \int_0^1 \frac{1}{x^2 + x - 6} \, dx
    \]
    We can factor the denominator:
    \[        x^2 + x - 6 = (x - 2)(x + 3)
    \]
    Using partial fraction decomposition, we have:
    \[        \frac{1}{(x - 2)(x + 3)} = \frac{A}{x - 2} + \frac{B}{x + 3}
    \]
    Solving for $A$ and $B$, we find:
    \[        A = \frac{1}{5} \quad \text{and} \quad B = -\frac{1}{5}
    \]
    Thus:
    \begin{align*}
        \int_0^1 \frac{1}{x^2 + x - 6} \, dx &= \int_0^1 \frac{1}{5} \cdot \frac{1}{x - 2} \, dx - \int_0^1 \frac{1}{5} \cdot \frac{1}{x + 3} \, dx \\
        &= \frac{1}{5} \ln|x - 2| \Big|_0^1 - \frac{1}{5} \ln|x + 3| \Big|_0^1 = \frac{1}{5} (\ln 1 - \ln 2) - \frac{1}{5} (\ln 4 - \ln 3) \\
        &= \frac{1}{5} (\ln 3 - \ln 8) = \frac{1}{5} \ln \left( \frac{3}{8} \right)
    \end{align*}
\end{eg}

\begin{eg}
    Let's compute the integral:
    \[
        \int_{-2}^0 \frac{1}{x^2 + 4x + 8} \, dx
    \]
    We can complete the square in the denominator:
    \[        x^2 + 4x + 8 = (x + 2)^2 + 4
    \]
    Let $u = \frac{x + 2}{2}$, thus $x = 2u - 2$ and $dx = 2 \, du$. When $x = -2$, $u = 0$ and when $x = 0$, $u = 1$. Therefore:
    \begin{align*}
        \int_{-2}^0 \frac{1}{x^2 + 4x + 8} \, dx &= \int_0^1 \frac{1}{4u^2 + 4} \cdot 2 \, du = \int_0^1 \frac{1}{2(u^2 + 1)} \, du \\
        &= \frac{1}{2} \int_0^1 \frac{1}{u^2 + 1} \, du = \frac{1}{2} \arctan(u) \Big|_0^1 = \frac{1}{2} \left( \frac{\pi}{4} - 0 \right) = \frac{\pi}{8}
    \end{align*}
\end{eg}